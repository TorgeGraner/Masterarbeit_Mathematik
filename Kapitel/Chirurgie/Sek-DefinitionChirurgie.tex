Sei \(\mathcal{S}^i\hookrightarrow\mathring{\mathcal{M}}\) eine eingebettete Sph\"are mit trivialem Normalenb\"undel. Die letzte Forderung bedeutet hierbei, dass \({N\mathcal{S}\cong\underline{\mathbb{R}}^{j+1}}\) gilt, sodass eine Tubenumgebung von \(\mathcal{S}\) zu \(\underline{\mathbb{R}}^{j+1}\) isomorph ist. Als Anklebeabbildung sei also eine Einbettung \(\Phi\colon\underline{\mathbb{R}}^{j+1}\hookrightarrow\mathring{\mathcal{M}}\) gew\"ahlt. Da das Normalenb\"undel der eingebetteten Sph\"are \({\mathbb{S}^i\times\{0\}^j\subset\mathbb{S}^n}\) ebenso trivial ist, lassen sich \(\mathcal{M}\) und \(\mathbb{S}^n\) entlang \(\mathcal{S}\cong\mathbb{S}^i\) verkleben. Die entstehende Mannigfaltigkeit gehe durch \textbf{Chirurgie} an \(\mathcal{S}\) aus \(\mathcal{M}\) hervor. Die g\"angigste Bezeichnung ist hierbei wohl \(\chi(\mathcal{M},\Phi)\). Im Folgenden sei stattdessen
\[\mathcal{M}\multimap\mathbb{S}^i:=\mathcal{M}\mathop{+}^{\Phi}\mathbb{S}^n=\chi(\mathcal{M},\Phi)\]
eben jene Chirurgie.
\section{Verbindung zu Henkeln}
    Ein nah mit Chirurgie verbundener Vorgang ist das Anbringen eines Henkels. Sei hierzu \(\mathcal{W}^{n+1}\) eine Mannigfaltigkeit und \(\mathcal{S}^i\hookrightarrow\partial\mathcal{W}\) eine eingebettete Sph\"are mit trivialem Normalenb\"undel in \(\mathcal{W}\). Erneut ist \(N\mathcal{S}\cong\underline{\mathbb{R}}^{n+1}\) (nicht \(\underline{\mathbb{R}}^n\times\mathbb{R}_+\)). Eine Tubenumgebung ist also eine Einbettung \(\tilde{\Phi}\colon\underline{\mathbb{R}}^n\times\mathbb{R}_+\hookrightarrow\mathcal{W}\), die eine Tubenumgebung \(\Phi\colon\underline{\mathbb{R}}^n\hookrightarrow\partial\mathcal{W}\) fortsetzt. Analoges gilt f\"ur \(\mathbb{S}^i\subset\partial\mathbb{D}^{n+1}\). Die Verklebung 
    \[\mathcal{W}\multimap\mathbb{D}^{i+1}:=\mathcal{W}\mathop{+}^{\tilde{\Phi}}\mathbb{D}^{n+1}\,,\]
    gehe durch das \textbf{Anbringen eines Henkels} an \(\mathcal{S}\) aus \(\mathcal{W}\) hervor. Dieser Vorgang geschieht immer am Rand einer Mannigfaltigkeit. Naheliegenderweise gilt
    \[\partial(\mathcal{W}\multimap\mathbb{D}^{i+1})=\partial(\mathcal{W}\mathop{+}^{\tilde{\Phi}}\mathbb{D}^{n+1})\cong\partial\mathcal{W}\mathop{+}^{\Phi}\partial\mathbb{D}^{n+1}=\partial\mathcal{W}\multimap\mathbb{S}^i\,,\]
    der Rand der Anbringung eines Henkels an \(\mathcal{W}\) ist also die Chirurgie an \(\partial\mathcal{W}\). Diese Konstruktion sei in Abbildung \ref{fig:surgery} verdeutlicht. Dies vereinfacht die Sitation f\"ur geschlossene Mannigfaltigkeiten \(\mathcal{M}\), da in diesem Fall \(\mathcal{M}\times\mathbb{I}\) eine \((n+1)\)-Mannigfaltikeit mit \(\partial\mathcal{W}\cong\mathcal{M}\sqcup(-\mathcal{M})\) ist. Es gilt also auch
    \[\partial(\mathcal{M}\times\mathbb{I}\multimap\mathbb{D}^{i+1})\cong\mathcal{M}\sqcup-(\mathcal{M}\multimap\mathbb{S}^i)\]
    sodass \(\mathcal{M}\multimap\mathbb{S}^i\) und \(\mathcal{M}\) kobordant sind. Der Trick die Chirurgie derart als Rand darzustellen vereinfacht einige \"Uber\-le\-gun\-gen. Es sei jedoch Obacht geboten, da \(\mathcal{M}\times\mathbb{I}\) f\"ur glatte Mannigfaltigkeiten \(\mathcal{M}\) mit nicht leerem Rand keine glatte Mannigfaltigkeit ergibt. Entlang der Strata \(\partial\mathcal{M}\times\partial\,\mathbb{I}\) l\"asst sich keine glatte Struktur angeben, die mit der glatten Struktur von \(\mathcal{M}\) \"ubereinstimmt. Da im Folgenden ledliglich Mannigfaltigkeiten \(\mathcal{M}\) betrachtet werden, deren Rand leer oder eine Homotopiesph\"are ist, kann dem folgenderma\ss en Abhilfe geschaffen werden.
    \begin{figure}
    \centering
    \begin{tikzpicture}[scale = 0.6]

        \path (-50:3) arc (-50:230:3)
        \foreach\i in {0, 0.05, 0.1, 0.15, 0.2, 0.25, 0.3, 0.35, 0.4, 0.45, 0.5, 0.55, 0.6, 0.65, 0.7, 0.75, 0.8, 0.85, 0.9, 0.95, 1} { node [pos = \i] (A\i) {}};

        % Right tube backframe
        \foreach\i in {0, 0.05, 0.1, 0.15, 0.2, 0.25, 0.3, 0.35, 0.4, 0.45} { 
            \draw [thick, densely dotted, rotate = -50 + \i * 280] ({A\i}.center) ++(180:1 and 0.5) arc (180:360:1 and 0.3 - 0.875 * \i);
        }
        % Right core
        \draw [thick, orange]
            (-50:3) node {\tiny\textbullet} arc (-50:90:3);

        % Small ball
        \begin{scope}[yshift = 3cm] 
            % Cocore
            \draw [fill, color = darkgreen, opacity = 0.5] (0:0.1375 and 1) arc (0:360:0.1375 and 1);
            % Transversal sphere
            \draw [thick, densely dotted, color = {rgb,255:red,0;green,100;blue,0}] 
                (90:0.1375 and 1) arc (90:270:0.1375 and 1);
            \draw
                (1, 0) arc (0:360:1)
                (180:1 and 0.5) arc (180:360:1 and 0.5);
            \draw [densely dotted]
                (0:1) arc (0:180:1 and 0.5);
        \end{scope}

        % Right tube topframe
        \foreach\i in {0, 0.05, 0.1, 0.15, 0.2, 0.25, 0.3, 0.35, 0.4} { 
            \draw [thick, rotate = -50 + \i * 280] ({A\i}.center) ++(0:1 and 0.5) arc (0:180:1 and 0.3 - 0.875 * \i);
        }

        \path [thick, rotate = -50, color = red] ({A0}.center) ++(0:1 and 0.5) arc (0:180:1 and 0.3) node [pos = 0.7] {\tiny\textbullet};

        % Right tube
        \draw [thick]
            (-50:4) arc (-50:90:4) 
            (-50:2) arc (-50:90:2);

        % Big ball
        \begin{scope}[yshift = -3.75cm, scale = 3]
            \draw [densely dotted]
                (90:0.5 and 1) arc (90:270:0.5 and 1) 
                (0:1 and 0.5) arc (0:180:1 and 0.5);
            \shade[ball color = blue!10!white, opacity = 0.5] (1, 0) arc (0:360:1) -- cycle;
            \draw [thick]
                (0:1) arc (0:360:1) 
                (-90:0.5 and 1) arc (-90:90:0.5 and 1) 
                (180:1 and 0.5) arc (180:360:1 and 0.5);
        \end{scope}

        % Left tube
        \draw [thick]
            (90:4) arc (90:230:4) 
            (90:2) arc (90:230:2);

        % Left tube backframe
        \foreach\i in {0.55, 0.6, 0.65, 0.7, 0.75, 0.8, 0.85, 0.9, 0.95} { 
            \draw [thick, densely dotted, rotate = -50 + \i * 280] ({A\i}.center) ++(0:1 and 0.5) arc (0:180:1 and 0.3 - 0.875 * \i);
        }
        % Meridian backframe
        \draw [thick, densely dotted, rotate = -50 + 280, color = blue] ({A1}.center) ++(0:1 and 0.5) arc (0:180:1 and 0.3 - 0.875) node [pos = 0.6, red] {\tiny\textbullet};
        % Left core
        \draw [thick, orange]
            (90:3) arc (90:230:3) node {\tiny\textbullet};
        
        % Small ball shading
        \shade[ball color = blue!10!white, opacity = 0.5, yshift = 3cm] (1, 0) arc (0:360:1) -- cycle;
        
        % Transversal sphere top frame
        \draw [thick, color = darkgreen, yshift = 3cm] 
            (-90:0.1375 and 1) arc (-90:90:0.1375 and 1);

        % Last right tube topframe 
        \draw [thick, rotate = -50 + 0.45 * 280] ({A0.45}.center) ++(0:1 and 0.5) arc (0:180:1 and 0.3 - 0.875 * 0.45);

        % Left tube topframe
        \foreach\i in {0.55, 0.6, 0.65, 0.7, 0.75, 0.8, 0.85, 0.9, 0.95} { 
            \draw [thick, rotate = -50 + \i * 280] ({A\i}.center) ++(180:1 and 0.5) arc (180:360:1 and 0.3 - 0.875 * \i);
        }
        % Meridian topframe
        \draw [thick, rotate = -50 + 280, color = blue] ({A1}.center) ++(180:1 and 0.5) arc (180:360:1 and 0.3 - 0.875);
        
    \end{tikzpicture}
    \caption{Eine Vollkugel mit einem Henkel. Farblich markiert sind \textcolor{orange}{Kern}, \textcolor{darkgreen}{Kokern}, \textcolor{blue}{Meridian} und \textcolor{red}{\"Aquator}.}\label{fig:surgery}
\end{figure}
    \newpage
    \begin{lemma}\label{lem:smooth_man}
        Zu jeder Mannigfaltigkeit \(\mathcal{M}^n\) existiert eine Mannigfaltigkeit
        \[\mathcal{W}^{n+1}\subseteq\mathcal{M}\times\mathbb{I}\quad\text{mit}\quad\partial\mathcal{W}\cong\mathcal{M}\mathop{+}^{\partial\mathcal{M}}\mathcal{M}\,.\]
    \end{lemma}
    \begin{proof}
        Eine Kragenumgebung von \(\partial\mathcal{M}\) in \(\mathcal{M}\) kann zu einer topologischen Einbettung \(\partial\mathcal{M}\times\mathbb{R}_+^2\hookrightarrow\mathcal{M}\times\mathbb{I}\) fortgesetzt werden. Dann kann \(\mathbb{R}_+^2\) durch eine geeignete glatte Mannigfaltigkeit mit Rand ersetzt werden. Eine M\"oglichkeit diese zu definieren w\"are als jenes Gebiet \(Q\subseteq\mathbb{R}_+^2\), welches durch die Kurve \(\gamma\) abgegrenzt wird, die durch
        \[\gamma(t):=\left(1+t^2\right)\left(\cos(g(t)),\sin(g(t))\right)\quad\text{mit}\quad g(t)=\frac{h\left(t+\frac{1}{2}\right)}{h\left(t+\frac{1}{2}\right)+h\left(\frac{1}{2}-t\right)}\]
        und
        \[h(t)=\begin{cases}
            e^{-\frac{1}{x}} & x\geq0\\
            0 & \text{sonst}
        \end{cases}\]
        definiert ist. Siehe hierzu Abbildung \ref{fig:smooth_man}. Bezeichne die resultierende Mannigfaltigkeit durch \(\mathcal{W}\). Der Rand dieser Gl\"attung ist gerade eine Gl\"attung von
        \[\partial(\mathcal{M}\times\mathbb{I})=\partial\mathcal{M}\times\mathbb{I}\cup\mathcal{M}\times\partial\,\mathbb{I}\,.\]
        Aus der Definition l\"asst sich erkennen, dass diese zu \(\mathcal{M}+_{\partial\mathcal{M}}\mathcal{M}\) diffeomorph ist.
    \end{proof}
    \begin{figure}
        \centering
        \includegraphics[scale = 2]{Kapitel/Chirurgie/geogebra-export.png}
        \caption{Eine glatte Kurve, sodass das oben rechts berandete Gebiet eine glatte Mannigfaltigkeit mit Rand ist, die zum \(\mathbb{R}_+^2\) hom\"oomorph ist.}\label{fig:smooth_man}
    \end{figure}
    Sei \(\mathcal{W}\) gem\"a\ss{} Lemma \ref{lem:smooth_man}. Wird ein Henkel an dem Rand von \(\partial\mathcal{W}\) angebracht, ergibt sich
    \[\partial\left(\mathcal{W}\multimap\mathbb{D}^{i+1}\right)\cong\mathcal{M}\mathop{+}^{\partial\mathcal{M}}\left(\mathcal{M}\multimap\mathbb{S}^i\right)\,.\]
    Wenn \(\partial\mathcal{M}\) zus\"atzlich leer oder eine Homotopiesph\"are ist, folgt f\"ur \(0<i<n\) 
    \[H_i(\partial\mathcal{W})\cong H_i(\mathcal{M})\oplus H_i(\mathcal{M})\,.\]
    Diese Mannigfaltigkeit verh\"alt sich deshalb \"ahnlich genug zu \(\mathcal{M}\times\mathbb{I}\).

\subsection{Kombinatorische Chirurgie}
    Besonders im Rahmen homologischer \"Uberlegungen ist die glatte Struktur der Mannigfaltigkeit nicht vonn\"oten, sodass in diesem Fall eine topologisch \"aqui\-va\-len\-te und einfachere Notation genutzt werden kann. Sei eine glatte Einbettung \(\Phi\colon\mathbb{S}^i\times\mathbb{D}^{j+1}\hookrightarrow\mathring{\mathcal{M}}\) mit \(\mathcal{D}:=\im\Phi\) gegeben. Setze \(\mathcal{M}_0:=\mathcal{M}\setminus\mathring{\mathcal{D}}\) und betrachte das topologische Pushout
    \[\mathcal{M}^{\prime}:=\mathcal{M}_0\cup_{\partial\mathcal{D}}\left(\mathbb{D}^{i+1}\times\mathbb{S}^j\right)\,.\]
    Ebenso l\"asst sich das Anbringen eines Henkels an \(\mathcal{W}^{n+1}\) durch
    \[\mathcal{W}+H^i:=\mathcal{W}\cup_{\mathcal{D}}\left(\mathbb{D}^{i+1}\times\mathbb{D}^{j+1}\right)\]
    definieren. Die entstehende Mannigfaltigkeit sind hierbei hom\"oomorph zu ihren oberen glatten Definitionen (\cite{kosinski1992differential} Kapitel VI Proposition 8.1). 
    
    \subsubsection{Meridian und \"Aquator einer Chirurgie}
        Diese Definition legt nahe, warum \(\mathbb{S}^i\times\mathbb{S}^j\) bei dem Untersuchen von einer Chirurgie eine wichtige Rolle spielt. F\"ur ein \(x\in\mathbb{S}^i\) und ein \(y\in\mathbb{S}^j\) sei \(\Phi\left(\{x\}\times\mathbb{S}^j\right)\) ein \textbf{Meridian} und \(\Phi\left(\mathbb{S}^i\times\{y\}\right)\) ein \textbf{\"Aquator}. 

    \subsubsection{Kern und Kokern eines Henkels}
        Der Anteil von \(\mathcal{W}\multimap\mathbb{D}^{i+1}\), der mit \(\mathbb{D}^{i+1}\times0\) korrespondiert hei\ss e \textbf{Kern} und der Anteil \(0\times\mathbb{D}^{j+1}\) hei\ss e \textbf{Kokern}. Der Rand des Kernes ist gerade die Anklebesph\"are, der Rand des Kokerns hei\ss e transversale Sph\"are. Diese Begriffe sind insbesondere im Rahmen des H-Kobordismus-Satzes wichtig.
