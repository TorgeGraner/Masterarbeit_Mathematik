Sei \(\gamma\colon\mathbb{S}^n\to\operatorname{SO}(k)\) glatt und \((\mathbb{S}^n,E)\) die Standardsph\"are mit der Standardrahmung des Normalenb\"undels im \(\mathbb{R}^{n+k}\). Dann existiert genau eine normale Rahmung \(\Gamma\) von \(\mathbb{S}^n\), sodass der Rahmenwechsel von \(E\) zu \(\Gamma\) gerade \(\gamma\) ist. F\"ur \(\Gamma\) kann also
\[\Gamma_i(p)=\sum_{j=1}^k\gamma_{ij}(p)E_j(p)\]
als Definition genutzt werden. Dies liefert die normal gerahmte Mannigfaltigkeit \((\mathbb{S}^n,\Gamma)\in\Omega_n^{\text{\tiny Fr}}(\mathbb{S}^{n+k})\) und (auch wenn noch nicht klar ist dass diese wohldefiniert ist) eine Abbildung
\[J_n^k\colon\pi_n(\operatorname{SO}(k))\to\Omega_n^{\text{\tiny Fr}}(\mathbb{S}^{n+k})\,.\]
Sei nun \(\eqcl{\gamma}\in\ker J_n^k\), es gelte also \(J_n^k\eqcl{\gamma}=0\). Dann ist \((\mathbb{S}^n,\Gamma)\) gerahmt nullbordant. Das ist genau dann der Fall, wenn sich \(\Gamma\) auf eine Mannigfaltigkeit \(\mathcal{M}^{n+1}\subseteq\mathbb{R}_+^{n+k+1}\) mit \(\partial\mathcal{M}\cong\mathbb{S}^n\) fortsetzen l\"asst. F\"ur hinreichend gro\ss es \(k\), also f\"ur \(k\geq n+2\), stabilisiert sich diese Definition, und es ergibt sich der \textbf{Hopf-Whitehead-J-Homomorphismus}
\[J_n\colon\pi_n(\operatorname{SO})\to\Pi_n\,.\]
\begin{proposition}\label{prop:adams_order}
    Das Bild \({\im J_{4m-1}}\) ist eine zyklische Gruppe des Ranges
    \[\operatorname{Rang}\operatorname{Im}J_{4m-1}=\operatorname{Nenner}\left(\frac{B_{2m}}{4m}\right)\]
    und ein direkter Summand von \(\Pi_{4m-1}\).
\end{proposition}
\begin{proof}
    Siehe \cite{adams1966groups} Satz 1.5.
\end{proof}