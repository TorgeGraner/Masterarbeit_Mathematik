Um einen \(\mathbb{R}\)-Vektorr\"aum \(V\) effektiv betrachten zu k\"onnen, ist es stets n\"otig eine Basis \(\mathcal{B}\) zu w\"ahlen. Diese etabliert eine Vektorraumisomorphie \(\mathbb{R}^n\cong V\) und reduziert die Komplexit\"at der abstrakten Struktur von \(V\) auf den einfachst m\"oglichen Fall und etabliert einen \textit{Referenzrahmen}. Die Verallgemeinerung einer Basis auf Vektorr\"aume ist naheliegend, und besteht darin, dass auf stetige Art und Weise jeder Faser eine Basis zugeordnet wird. Hierbei existieren drei \"aquivalente Definitionen
\begin{definition}[Rahmen eines Vektorb\"undels]
    Sei \(\xi\colon E\to B\) ein Vektorb\"undel. Ein Rahmen ist 
    \begin{itemize}
        \item[i] eine Trivialisierung \(\xi\cong\underline{\mathbb{R}}^k\),
        \item[ii] die Wahl \(k\) linear unabh\"angiger Schnitte \(X_i\colon B\to E\) oder
        \item[iii] ein Schnitt im Rahmenb\"undel von \(\xi\).
    \end{itemize}
\end{definition}
Im Gegensatz zu Vektorr\"aumen, muss ein Rahmen keineswegs existieren. Viel eher ist die Existenz eines Rahmens eine starke Einschr\"ankung, da dies bereits impliziert, dass \(\xi\) trivial ist. Weiter sind unterschiedliche Rahmen nicht unbedingt zueinander \"aquivalent. Ein Vektorb\"undel zusammen mit einem Rahmen hei\ss e \textbf{gerahmtes Vektorb\"undel}. Die Wahl einer Rahmung von \(\xi\oplus\underline{\mathbb{R}}\) liefert eine \textbf{stabile Rahmung}. Wenn \(\xi\) zus\"atzlich orientierbar ist, ist es m\"oglich von orientierten Rahmungen zu sprechen. Im Folgenden seien alle Rahmungen orientiert. Seien \(F\) und \(G\) zwei Rahmungen eines Vektorb\"undels \(\xi\colon E\to B\). Dann sind sowohl \(F(p)\) als auch \(G(p)\) Basen von \(\xi^{-1}(p)\), und es kann der Basiswechsel \(M_{G(p)}^{F(p)}\in\operatorname{GL}^+(n)\) von \(F(p)\) zu \(G(p)\) betrachtet werden. Dies liefert eine stetige Funktion
\[M_G^F\colon B\to\operatorname{GL}^+(n),\,p\mapsto M_{G(p)}^{F(p)}\,,\]
den \textbf{Rahmenwechsel} von \(F\) zu \(G\). 
\begin{example}[Rahmenb\"undel der Sph\"are]\label{ex:framebundle_sphere}
    Der Tangentialvektorraum von \(\mathbb{S}^n\) an \(p\) kann als Untervektorraum des \(\mathbb{R}^{n+1}\) durch \(T_p\mathbb{S}^n\cong\{p\}^{\perp}\) identifiziert werden. Eine gegebene Orthonormalbasis \(b_i\) von \(T_p\mathbb{S}^n\subset\mathbb{R}^{n+1}\) kann also stets durch \(p\) zu einer Orthonormalbasis des \(\mathbb{R}^{n+1}\) erg\"anzt werden. Somit liegt die Matrix \(A:=(p,b_1,\dots,b_n)^{\intercal}\) in \(\operatorname{SO}(n+1)\) und es gilt \(A\cdot e_1=p\). Umgekehrt beschreibt eine Matrix \(B\in\operatorname{SO}(n+1)\) mit \(B\cdot e_1=p\) eine Orthonormalbasis von \(T_p\mathbb{S}^n\). Diese Korrespondenz liefert einen Hauptfaserb\"undelisomorphismus des orientiert orthonormalen Rahmenb\"undels der Sph\"are zu \(\operatorname{SO}(n+1)\). Weiter stiftet die Einh\"angung
    \[SA=\begin{pmatrix}
        A & 0\\
        0 & 1
    \end{pmatrix}\]
    eine Inklusion \(S\colon\operatorname{SO}(n)\hookrightarrow\operatorname{SO}(n+1)\) und f\"uhrt zu der im Folgenden au\ss erordentlich wichtigen Faserung
    \[\operatorname{SO}(n)\xrightarrow{S}\operatorname{SO}(n+1)\xrightarrow{\psi}\mathbb{S}^n\,.\]
    Eine tangentiale Rahmung der Sph\"are existiert nur f\"ur \(n\in\{1,3,7\}\), da nur in diesem Fall nullteilerfreie Multiplikationen auf dem \(\mathbb{R}^{n+1}\) existieren.
\end{example}

\subsection{Parallelisierbarkeit}
    Sei \(\mathcal{V}\) eine Mannigfaltigkeit. Dann sind das Tangentialb\"undel und gegebenenfalls auch das Normalenb\"undel definiert. Die Frage nach der Trivialit\"at dieser B\"undel erm\"oglicht einige Aussagen \"uber \(\mathcal{V}\) und m\"oglichweise den umliegenden Raum. \(\mathcal{V}\) hei\ss e (stabil) tangential oder normal gerahmt, wenn \(T\mathcal{V}\) oder \(N\mathcal{V}\) mit einer (stabilen) Rahmung versehen sind. Eine Mannigfaltigkeit mit stabil trivialem Tangentialb\"undel hei\ss e auch \textbf{\(\pi\)-Mannigfaltigkeit}. Die Signifikanz von \(\pi\)-Mannigfaltigkeiten der Dimension \(n\) f\"ur die Chirurgietheorie besteht darin, dass eingebettete \(k\)-Sph\"aren mit \(n\geq2k\) ein stabil triviales Normalenb\"undel besitzen, die Frage nach der eigentlichen Trivialit\"at etwas vereinfacht.

    \begin{theorem}\label{thm:vec_dim_triv}
        Sei \(\xi\) ein Vektorb\"undel vom Rang \(n\) \"uber einem \(k\)-dimensionalen CW-Komplex \(X\) mit \(n>k\). Dann ist \(\xi\) genau dann stabil trivial, wenn \(\xi\) bereits trivial ist.
    \end{theorem}
    \begin{proof}
        Sei \(F\colon\xi\oplus\underline{\mathbb{R}}\to\underline{\mathbb{R}}^{n+1}\) eine Trivialisierung. Diese liefert in jeder Faser eine Einbettung \(\xi^{-1}(p)\hookrightarrow\mathbb{R}^{n+1}\), sei etwa
        \[\xi^{-1}(p)\cong\{f(p)\}^{\perp}\cong T_{f(p)}\mathbb{S}^n\]
        f\"ur eine stetige Abbildung \(f\colon X\to\mathbb{S}^n\), die wegen \(n>k\) nullhomotop sein muss. Andererseits gilt nun \(\xi\cong f^*(T\mathbb{S}^n)\), sodass \(\xi\) bereits trivial sein muss.
    \end{proof}

    \begin{theorem}\label{thm:sub_pi_triv}
        Eine Untermannigfaltigkeit \(\mathcal{V}^k\) einer \(\pi\)-Mannigfaltigkeit \(\mathcal{M}^n\) mit \(n\geq2k\) ist genau dann eine \(\pi\)-Mannigfaltigkeit, wenn ihr Normalenb\"undel stabil trivial ist.
    \end{theorem}
    \begin{proof}
        Aus einer Zerlegung 
        \[T\mathcal{V}\oplus N\mathcal{V}\cong T\mathcal{M}|_{\mathcal{V}}\,,\]
        folgt
        \[T\mathcal{V}\oplus N\mathcal{V}\oplus\underline{\mathbb{R}}\cong T\mathcal{M}|_{\mathcal{V}}\oplus\underline{\mathbb{R}}\cong\underline{\mathbb{R}}^{n+1}\,.\]
        Somit ist einerseits \(T\mathcal{V}\oplus\underline{\mathbb{R}}^{n-k+1}\) trivial, wenn \(N\mathcal{V}\oplus\underline{\mathbb{R}}\) es ist, und andererseits \(N\mathcal{V}\oplus\underline{\mathbb{R}}^{k+1}\) trivial, wenn \(T\mathcal{V}\oplus\underline{\mathbb{R}}\) es ist. Aus Satz \ref{thm:vec_dim_triv} folgt, dass dann jeweils \(T\mathcal{V}\) und \(N\mathcal{V}\) stabil trivial sind.
    \end{proof}

    \begin{corollary}\label{cor:subsphere_stable}
        Jede in eine \(\pi\)-Mannigfaltigkeit eingebettete Sph\"are \(\mathcal{S}^k\hookrightarrow\mathcal{M}^n\) mit \(n\geq2k\) besitzt ein stabil triviales Normalenb\"undel.
    \end{corollary}
    \begin{proof}
        Das folgt aus Satz \ref{thm:sub_pi_triv}, da alle Sph\"aren \(\pi\)-Mannigfaltigkeiten sind \ref{cor:hom_pi}.
    \end{proof}
    