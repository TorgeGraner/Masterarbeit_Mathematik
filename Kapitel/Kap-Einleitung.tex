Der zentrale Begriff, der ben\"otigt wird, um die bekannte Differentialrechnung im \(\mathbb{R}^n\) auf Mannigfaltigkeiten zu verallgemeinern, ist durch die differenzierbare Struktur gegeben. Erst durch diese ist es m\"oglich die Differenzierbarkeit von Funktionen zwischen Mannigfaltigkeiten \(f\colon\mathcal{M}^m\to\mathcal{N}^n\) auf die bekannte Differenzierbarkeit zu reduzieren. Umso verbl\"uffender ist es, dass bereits auf dem \(\mathbb{R}^4\), aufgefasst als topologische Mannigfaltigkeit, unterschiedliche, nicht-\"aquivalente differenzierbare Strukturen existieren. Gleicherma\ss en l\"asst sich die \(n\)-Sph\"are f\"ur gewisse \(n\) mit nicht-\"aquivalenten glatten Atlanten versehen. Auf diese Art und Weise ergeben sich die \textit{exotischen Sph\"aren}. Diese sind also gerade glatte Mannigfaltigkeiten, die hom\"oomorph zu der \(n\)-Sph\"are sind, und stehen \"uber die verallgemeinerte Poincar\'e-Vermutung in einem engen Zusammenhang mit den Homotopiesph\"aren, also jenen Mannigfaltigkeiten, die zu der \(n\)-Sph\"are homotopie\"aquivalent sind. Es gilt:
\begin{proposition}[Verallgemeinerte Poincar\'e-Vermutung]
    Jede \(n\)-Homotopiesph\"are ist zu der \(n\)-Sph\"are hom\"oomorph.
\end{proposition}
\noindent Die Menge der exotischen Sph\"aren ist somit gleich der Menge der Diffeomorphieklassen von Homotopiesph\"aren. In hohen Dimensionen, also f\"ur \(n\geq5\) gilt zus\"atzlich der \(H\)-Kobordismus-Satz, also ist
\begin{proposition}
    F\"ur \(n\geq5\) ist jeder einfach zusammenh\"angende Kobordismus von \(\mathcal{M}^n\) zu \(\mathcal{N}^n\), der \(\mathcal{M}\) und \(\mathcal{N}\) als Deformationsretrakte enth\"alt, zu \(\mathcal{M}\times\mathbb{I}\) diffeomorph. Insbesondere sind \(\mathcal{M}\) und \(\mathcal{N}\) diffeomorph.
\end{proposition}
\noindent Ein derartiger Kobordismus hei\ss e \(H\)-Kobordismus. Die Menge der \(H\)-Ko\-bor\-dis\-mus\-klassen von Homotopiesph\"aren ist somit f\"ur \(n\geq5\) gerade die Menge der exotischen Sph\"aren und sei durch \(\Theta_n\) bezeichnet. Die Frage nach einer geeigneten Gruppenstruktur und der Gr\"o\ss e von \(\Theta_n\) solle im Folgenden gekl\"art werden.
Obgar das Ziel dieser Arbeit ist eine \textit{m\"oglichst} geschlossene Form zu finden, ist dies in diesem Rahmen aufgrund der Komplexit\"at nicht m\"oglich. Insbesondere werden grundlegende Kenntnisse der algebraischen Topologie (siehe Allen Hatcher \glqq Algebraic Topology\grqq{} \cite{hatcher2002algebraic}) und Differentialgeometrie vorausgesetzt. Der Arbeit zugrundeliege liegt das Paper \glqq Groups of homotopy spheres: I\grqq{} von Michel Kervaire und John Milnor \cite{kervaire1963homotopy}, dessen Inhalt jedoch in vielen anderen Werken aufbearbeitet wurde. Teil II wurde leider nie ver\"offentlicht. Viele Ideen entstammen Antoni Kosinskis \glqq Differential manifolds\grqq{} \cite{kosinski1992differential} und Jerome Levines \glqq Lectures on groups of homotopy spheres\grqq{} \cite{levine1985lectures}. Jegliche ben\"otigte Theorie zu Vektorb\"undeln l\"asst sich in Karlheinz Knapps \glqq Vektorb\"undel\grqq{} \cite{knapp2013vektorbuendel} oder John Milnors und James Stasheffs \glqq Characteristic classes\grqq{} \cite{milnor1974characteristic} finden. Ein besonderes Augenmerk wurde darauf gelegt, Mannigfaltigkeiten mit Ecken zu vermeiden, auch wenn es \textit{offensichtlich} m\"oglich ist diese zu gl\"atten. 