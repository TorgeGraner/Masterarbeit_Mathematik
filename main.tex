\documentclass[oneside]{book}
\usepackage[utf8]{inputenc}
\usepackage{units}    % useful for settings units, \unit[23]{m}
\usepackage{nicefrac} % for setting fractions esp. within text, \nicefrac{km}{h}
\usepackage[german]{babel}
\usepackage{tikz}
\usetikzlibrary{positioning, automata, arrows, patterns}
\usepackage{tikz-3dplot}
\usepackage{pgf, pgfplots}
\pgfplotsset{compat=1.18} 
\usepackage{algorithm, algorithmicx, algpseudocode} % Pseudocode
\usepackage{amsmath, amssymb, amsthm} % definitions
\usepackage{bbm}
\usepackage{pict2e} % angle symbol
\usepackage[pdfpagelabels,colorlinks,allcolors=black]{hyperref} % Hyperlink referencing
\usepackage{subcaption} % subfigure captions
\usepackage{bm} % Bold math letters
\usepackage{tipa, stmaryrd} % eta symbol
\usepackage{utfsym}
\usepackage{multirow} % Multirow table
\usepackage{booktabs} % Better tables
\usepackage{csquotes}
\usepackage{mleftright} % Fix left/right formatting
\usepackage{xcolor}
\usepackage[style=alphabetic]{biblatex}
\bibliography{sources.bib}

\definecolor{darkgreen}{RGB}{0, 100, 0}

\algrenewcommand\algorithmicrequire{\textbf{Input:}}
\algrenewcommand\algorithmicensure{\textbf{Output:}}
\algnewcommand\AND{\textbf{and }}
\algnewcommand\OR{\textbf{or }}
\algnewcommand\NOT{\textbf{not }}

%%%%%%%%%%%%%%%%%%%%%%%%%%%%%%%%%%%%%%%%%%%%%%%%%%%%%%%%%%%%%%%%%%%%%%%%%%%%%%%%%%%%

\renewcommand{\geq}{\geqslant}
\renewcommand{\leq}{\leqslant}
\renewcommand{\phi}{\varphi}
\renewcommand{\epsilon}{\varepsilon}
%\renewcommand{\theta}{\vartheta}
\renewcommand{\eta}{%
    \raisebox{-0.7mm}[0.8\height][\width]{
        \parbox{0.5mm}{
            \makebox[0.1mm]{
            \begin{tikzpicture}%
                \draw node at (-0.04, 0) {\scalebox{0.9}{\textlhtlongi}}; 
                \draw node at (0.09, -0.08) {\rotatebox[origin=c]{15}{\scalebox{1}[-1]{\(\Rbag\)}}};
            \end{tikzpicture}%
            }
        }
    }\hspace*{1pt}
}

\DeclareSymbolFont{myletters}{OML}{ztmcm}{m}{it}
\DeclareMathSymbol{\uplambda}{\mathord}{myletters}{"15}
\renewcommand{\lambda}{\uplambda}
% Fix left/right formatting
\renewcommand{\left}{\mleft}
\renewcommand{\right}{\mright}
\renewcommand{\ker}{\operatorname{Ker}}
\newcommand{\im}{\operatorname{Im}}
\newcommand{\dx}{\mathrm{d}}
\newcommand{\abs}[1]{\left|#1\right|}
\newcommand{\norm}[1]{\abs{\abs{#1}}}
\def\i{\boldsymbol{i}}
\newcommand{\eqcl}[1]{[\,#1\,]}
\newcommand{\overeq}[1]{\mathop{=}^{\text{\tiny #1}}}
\newcommand{\overcong}[1]{\mathop{\cong}^{\text{\tiny #1}}}
\newcommand{\overto}[1]{\mathop{\to}^{\text{\tiny #1}}}
\newcommand{\surgery}{\,\usym{2702}\,}
\newcommand{\commend}[1]{}

\newtheoremstyle{break}
  {\topsep}{\topsep}%
  {\itshape}{}%
  {\bfseries}{}%
  {\newline}{}%
\theoremstyle{break}
\newtheorem{theorem}{Satz}[section]
\newtheorem{lemma}[theorem]{Lemma}
\newtheorem{corollary}[theorem]{Korollar}
\newtheorem{proposition}[theorem]{Proposition}
\newtheorem{definition}{Definition}[section]
\theoremstyle{definition}
\newtheorem{example}{Beispiel}[section]
\theoremstyle{definition}
\newtheorem{remark}{Bemerkung}[section]
    
\theoremstyle{definition}

\renewcommand\qedsymbol{q.e.d.}

\begin{document}

    \begin{titlepage}
               
        \begin{center}
            \LARGE{\textsc{Institut für Mathematik}}
            
            \vfill
            
            \LARGE{\emph{Masterarbeit}}
            
            \vspace{8mm}
            
            \huge{\textbf{Eigenschaften der Menge der H-Kobordismusklassen von Homotopie-Sphären}}
            
            \vspace{8mm}
            
            \LARGE{Torge Graner}
            
            \vspace{32mm}
            
            \large{\today}
            \vfill
            
            \begin{tabular}{ll}
              \large
              Erstgutachter: & \large Prof. Dr. math. Oliver Röndigs\\
              \large
              Zweitgutachter: & \large Prof. Dr. math. Markus Spitzweck
            \end{tabular}
        \end{center} 
    \end{titlepage}

    \tableofcontents
    \thispagestyle{empty}

    \chapter{Einleitung}
        Der zentrale Begriff, der ben\"otigt wird, um die bekannte Differentialrechnung im \(\mathbb{R}^n\) auf Mannigfaltigkeiten zu verallgemeinern, ist durch die differenzierbare Struktur gegeben. Erst durch diese ist es m\"oglich die Differenzierbarkeit von Funktionen zwischen Mannigfaltigkeiten \(f\colon\mathcal{M}^m\to\mathcal{N}^n\) auf die bekannte Differenzierbarkeit zu reduzieren. Umso verbl\"uffender ist es, dass bereits auf dem \(\mathbb{R}^4\), aufgefasst als topologische Mannigfaltigkeit, unterschiedliche, nicht-\"aquivalente differenzierbare Strukturen existieren. Gleicherma\ss en l\"asst sich die \(n\)-Sph\"are f\"ur gewisse \(n\) mit nicht-\"aquivalenten glatten Atlanten versehen. Auf diese Art und Weise ergeben sich die \textit{exotischen Sph\"aren}. Diese sind also gerade glatte Mannigfaltigkeiten, die hom\"oomorph zu der \(n\)-Sph\"are sind, und stehen \"uber die verallgemeinerte Poincar\'e-Vermutung in einem engen Zusammenhang mit den Homotopiesph\"aren, also jenen Mannigfaltigkeiten, die zu der \(n\)-Sph\"are homotopie\"aquivalent sind. Es gilt:
\begin{proposition}[Verallgemeinerte Poincar\'e-Vermutung]
    Jede \(n\)-Homotopiesph\"are ist zu der \(n\)-Sph\"are hom\"oomorph.
\end{proposition}
\noindent Die Menge der exotischen Sph\"aren ist somit gleich der Menge der Diffeomorphieklassen von Homotopiesph\"aren. In hohen Dimensionen, also f\"ur \(n\geq5\) gilt zus\"atzlich der \(H\)-Kobordismus-Satz, also ist
\begin{proposition}
    F\"ur \(n\geq5\) ist jeder einfach zusammenh\"angende Kobordismus von \(\mathcal{M}^n\) zu \(\mathcal{N}^n\), der \(\mathcal{M}\) und \(\mathcal{N}\) als Deformationsretrakte enth\"alt, zu \(\mathcal{M}\times\mathbb{I}\) diffeomorph. Insbesondere sind \(\mathcal{M}\) und \(\mathcal{N}\) diffeomorph.
\end{proposition}
\noindent Ein derartiger Kobordismus hei\ss e \(H\)-Kobordismus. Die Menge der \(H\)-Ko\-bor\-dis\-mus\-klassen von Homotopiesph\"aren ist somit f\"ur \(n\geq5\) gerade die Menge der exotischen Sph\"aren und sei durch \(\Theta_n\) bezeichnet. Die Frage nach einer geeigneten Gruppenstruktur und der Gr\"o\ss e von \(\Theta_n\) solle im Folgenden gekl\"art werden.
Obgar das Ziel dieser Arbeit ist eine \textit{m\"oglichst} geschlossene Form zu finden, ist dies in diesem Rahmen aufgrund der Komplexit\"at nicht m\"oglich. Insbesondere werden grundlegende Kenntnisse der algebraischen Topologie (siehe Allen Hatcher \glqq Algebraic Topology\grqq{} \cite{hatcher2002algebraic}) und Differentialgeometrie vorausgesetzt. Der Arbeit zugrundeliege liegt das Paper \glqq Groups of homotopy spheres: I\grqq{} von Michel Kervaire und John Milnor \cite{kervaire1963homotopy}, dessen Inhalt jedoch in vielen anderen Werken aufbearbeitet wurde. Teil II wurde leider nie ver\"offentlicht. Viele Ideen entstammen Antoni Kosinskis \glqq Differential manifolds\grqq{} \cite{kosinski1992differential} und Jerome Levines \glqq Lectures on groups of homotopy spheres\grqq{} \cite{levine1985lectures}. Jegliche ben\"otigte Theorie zu Vektorb\"undeln l\"asst sich in Karlheinz Knapps \glqq Vektorb\"undel\grqq{} \cite{knapp2013vektorbuendel} oder John Milnors und James Stasheffs \glqq Characteristic classes\grqq{} \cite{milnor1974characteristic} finden. Ein besonderes Augenmerk wurde darauf gelegt, Mannigfaltigkeiten mit Ecken zu vermeiden, auch wenn es \textit{offensichtlich} m\"oglich ist diese zu gl\"atten. 

    \chapter{Mannigfaltigkeiten}
        Bevor damit begonnen werden kann \"uberhaupt eine Gruppenstruktur auf \(\Theta_n\) zu definieren, m\"ussen zun\"achst einige Vorbereitungen getroffen werden. Im Folgenden seien alle Mannigfaltigkeiten, soweit nicht anders spezifiziert, zweit\-ab\-z\"ahl\-bar, kompakt und glatt (m\"oglicherweise mit Rand). Die Annahme der Glattheit ist keine Einschr\"ankung gegen\"uber der \(\mathcal{C}^k\)-Differenzierbarkeit, da zu jeder differenzierbaren Mannigfaltigkeit mit einer \(\mathcal{C}^k\)-Struktur \(\mathfrak{A}\) eine \(\mathcal{C}^{\infty}\)-Struktur \(\mathfrak{B}\) und ein \(\mathcal{C}^k\)-Diffeomorphismus \((\mathcal{M},\mathfrak{A})\to(\mathcal{M},\mathfrak{B})\) existiert \cite{hirsch2012difftop} Satz 3.4. Eine kompakte Mannigfaltigkeit mit leerem Rand hei\ss e \textbf{geschlossen}. 

\section{Orientierbarkeit und Poincar\'e-Dualit\"at}
    Sei \(\mathcal{M}^n\) eine kompakte, topologische Mannigfaltigkeit. Ist diese geschlossen, ist sie genau dann orientierbar, wenn \(H_n(\mathcal{M})\cong\mathbb{Z}\) gilt. Ist sie berandet, muss \(H_n(\mathcal{M},\partial\mathcal{M})\cong\mathbb{Z}\) gelten. Eine Wahl eines Erzeugers dieser Gruppen korrespondiert mit der Wahl einer Orientierung von \(\mathcal{M}\), hei\ss e \textbf{Fundamentalklasse} von \(\mathcal{M}\) und wird gelegentlich unter leichtem Notationsmissbrauch mit \(\eqcl{\mathcal{M}}\) bezeichnet. Ist \(\mathcal{M}\) nicht zusammenh\"angend, sei eine Orientierung eine Wahl von Fundamentalklassen aller Komponenten. Im Folgenden werden alle Mannigfaltigkeit als orientierbar angenommen. Wenn \(\iota\colon\mathcal{M}\hookrightarrow\mathcal{W}\) eine Einbettung ist, kann \(\eqcl{\mathcal{M}}\) als Element von \(H_n(\mathcal{W})\) aufgefasst werden. Schreibe f\"ur die inkludierte Homologieklasse
\[\eqcl{\mathcal{M}\mathrel{|}\mathcal{W}}:=\iota_*\eqcl{\mathcal{M}}\in H_n(\mathcal{W})\,.\]
Ist \(\mathcal{M}\) geschlossen, sagt die \textbf{Poincar\'e-Dualit\"at} nun gerade aus, dass der Homomorphismus 
\[H^k(\mathcal{M})\to H_{n-k}(\mathcal{M}),\,p\mapsto p\frown\eqcl{\mathcal{M}}\]
ein Isomorphismus ist (\cite{hatcher2002algebraic} Satz 3.30). Ist der Rand nicht leer, nimmt diese Isomorphie die Formen
\[H^k(\mathcal{M})\mathop{\cong}H_{n-k}(\mathcal{M},\partial\mathcal{M})\quad\text{und}\quad H^k(\mathcal{M},\partial\mathcal{M})\mathop{\cong}H_{n-k}(\mathcal{M})\]
an. Im Folgenden sei das duale Element von \(x\) auch durch \(x^*\) gekennzeichnet. Ist \(\mathcal{W}^n\) ein Kobordismus mit \(\partial\mathcal{W}\cong\partial_-\mathcal{W}\sqcup\partial_+\mathcal{W}\). Dann gilt die \textbf{Lefschetz-Dualit\"at}, also ist 
\[H^k(\mathcal{W},\partial_-\mathcal{W})\to H_{n-k}(\mathcal{W},\partial_+\mathcal{W}),\,\sigma\mapsto\sigma\frown\eqcl{\mathcal{W}}\,.\]
ist ein Isomorphismus (\cite{hatcher2002algebraic} Satz 3.43). Die langen exakten Folgen der Homomologie und der Kohomologie sind dabei durch folgendes bis auf Vorzeichen kommutative Diagramm verbunden (\cite{dold1980lectures} Abschnitt VIII Satz 9.1), in welchem \(i+j=n\) gelte.
\begin{equation}\label{diag:poin_exact}
    \begin{tikzpicture}[baseline=(current  bounding  box.center)]
        \draw 
            (0, 0) node (A) {\(H_{i+1}(\mathcal{W},\partial\mathcal{W})\)}
            (3, 0) node (B) {\(H_i(\partial\mathcal{W})\)}
            (5.6, 0) node (C) {\(H_i(\mathcal{W})\)}
            (8.4, 0) node (D) {\(H_i(\mathcal{W},\partial\mathcal{W})\)}
            (0, 2) node (E) {\(H^{j-1}(\mathcal{W})\)}
            (3, 2) node (F) {\(H^{j-1}(\partial\mathcal{W})\)}
            (5.6, 2) node (G) {\(H^j(\mathcal{W},\partial\mathcal{W})\)}
            (8.4, 2) node (H) {\(H^j(\mathcal{W})\)}
            (A) edge [-stealth] node [above] {\(\partial\)} (B)
            (B) edge [-stealth] node [above] {\(\iota_*\)} (C)
            (C) edge [-stealth] node [above] {\(q_*\)} (D)
            (E) edge [-stealth] node [above] {\(\iota^*\)} (F)
            (F) edge [-stealth] node [above] {\(\delta\)} (G)
            (G) edge [-stealth] node [above] {\(q^*\)} (H)
            (E) edge [-stealth] node [right] {\(\frown\eqcl{\mathcal{W}}\)} (A)
            (F) edge [-stealth] node [right] {\(\frown\eqcl{\partial\mathcal{W}}\)} (B)
            (G) edge [-stealth] node [right] {\(\frown\eqcl{\mathcal{W}}\)} (C)
            (H) edge [-stealth] node [right] {\(\frown\eqcl{\mathcal{W}}\)} (D)
            ;
    \end{tikzpicture}
\end{equation}
    
\section{Die Schnittpaarung}
    Sei \(\mathcal{W}^n\) eine Mannigfaltigkeit und \(i+j=n\). Dann l\"asst sich \"uber die Kronecker-Paarung und Poincar\'e-Dualit\"at eine Bilinearform
\[\mathrel{-}\cdot\mathrel{-}\,\colon H_i(\mathcal{W})\otimes H_j(\mathcal{W})\to\mathbb{Z},\,x\cdot y:=\left\langle x^*\smile y^*,\eqcl{\mathcal{W}}\right\rangle\]
definieren. Aus den Eigenschaften des Cup-Produktes folgt einerseits
\begin{equation}\label{eq:intersect_prop_1}
    x\cdot y=\left\langle x^*\smile y^*,\eqcl{\mathcal{W}}\right\rangle=(-1)^{ij}\left\langle y^*\smile x^*,\eqcl{\mathcal{W}}\right\rangle=(-1)^{ij}(y\cdot x)\,,
\end{equation}
aus den Eigenschaften der Kroneckerpaarung ergibt sich
\begin{equation}\label{eq:intersect_prop_2}
    x\cdot y=\langle x^*\smile y^*,\eqcl{\mathcal{W}}\rangle=\langle q^*x^*,y^*\frown\eqcl{\mathcal{W}}\rangle=\langle q^*x^*,y\rangle\,.
\end{equation}
Da \(\mathcal{M}\) eine orientierbare Mannigfaltigkeit ist, besitzt die Auswertungsabbildung eine besonders einfache Form. Es gilt:
\begin{lemma}\label{lem:intersect_factor}
    Die Abbildung \({\operatorname{Ad}\colon H_i(\mathcal{M})\to\operatorname{Hom}(H_j(\mathcal{M}),\mathbb{Z}),\,x\mapsto x\cdot(\mathrel{-})}\) ist gleich der Komposition
    \[H_i(\mathcal{M})\mathop{\longrightarrow}^{q_*}H_i(\mathcal{M},\partial\mathcal{M})\mathop{\longrightarrow}^{\text{\tiny P.D.}}H^j(\mathcal{M})\mathop{\longrightarrow}^{\text{\tiny U.K.}}\operatorname{Hom}(H_j(\mathcal{M}),\mathbb{Z})\,.\]
\end{lemma}
\begin{proof}
    Die Komposition bildet \({x\in H_i(\mathcal{M})}\) auf \({\langle\left(q_*(x)\right)^*,\mathrel{-}\rangle}\) ab, wobei aus Diagramm \ref{diag:poin_exact} folgt, dass die Gleichung \({(q_*(x))^*=q^*x^*}\) gilt. Dies zeigt
    \[\langle(q_*(x))^*,\mathrel{-}\rangle=\langle q^*x^*,\mathrel{-}\rangle\mathop{=}^{\text{\tiny\eqref{eq:intersect_prop_2}}}x\cdot(\mathrel{-})=\operatorname{Ad}(x)\,.\]
\end{proof}
\newpage
\begin{corollary}\label{crl:intersect_uni}
    Sei \(\mathcal{M}^{i+j}\) derart, dass \({H_i(\partial\mathcal{M})=H_{i-1}(\partial\mathcal{M})=0}\) gilt und \(H_{j-1}(\mathcal{M})\) frei ist. Dann ist \(\mathrel{-}\cdot\mathrel{-}\) unimodular.
\end{corollary}
\begin{proof}
    Betrachte die Faktorisierung
    \[\operatorname{Ad}\colon H_i(\mathcal{M})\mathop{\longrightarrow}^{q_*}H_i(\mathcal{M},\partial\mathcal{M})\mathop{\longrightarrow}^{\text{\tiny P.D.}}H^j(\mathcal{M})\mathop{\longrightarrow}^{\text{\tiny U.K.}}\operatorname{Hom}(H_j(\mathcal{M}),\mathbb{Z})\]
    gem\"a\ss{} Lemma \ref{lem:intersect_factor}. Aus den Annahmen folgt, dass alle Abbildungen isomorphismen sind.
\end{proof}
Die derart definierte Schnittform ist besonders interessant, wenn \(\mathcal{M}\) gerade Dimension besitzt und \(i=j\) gilt. In diesem Fall hei\ss e die Bilinearform \textbf{Schnittform}.
\commend{
    Seien \(\mathcal{M}\) und \(\mathcal{N}\) Untermannigfaltigkeiten komplement\"arer Dimension in \(\mathcal{W}\). Definiere ihre \textbf{Schnittzahl} durch
    \[\mathcal{M}\cdot\mathcal{N}:=\eqcl{\mathcal{M}\mathrel{|}\mathcal{W}}\cdot\eqcl{\mathcal{N}\mathrel{|}\mathcal{W}}\,.\]
    Diese Definition l\"asst sich geometrisch interpretieren. Ohne die Isotopieklasse von \(\mathcal{M}\) oder \(\mathcal{N}\) zu ver\"andern, kann angenommen werden, dass sich die beiden Mannigfaltigkeiten transversal schneiden. In diesem Fall ist \(\mathcal{V}:=\mathcal{M}\cap\mathcal{N}\) eine nulldimensionale Mannigfaltigkeit, die folglich aus endlich vielen Punkten besteht. In jedem Punkt \(p\in\mathcal{V}\) existiert nun aufgrund der Transversalit\"at eine direkte Summenzerlegung \(T_p\mathcal{W}=T_p\mathcal{M}\oplus T_p\mathcal{N}\). Eine positiv orientierte Basis von \(T_p\mathcal{M}\) zusammen mit einer positiv orientierten Basis von \(\mathcal{N}\) ergibt eine Basis von \(T_p\mathcal{W}\). Setze \(\epsilon_p:=1\), falls die zusammengesetzte Basis erneut positiv orientiert, und \(\epsilon_p:=-1\) sonst. Dann gilt
    \[\mathcal{M}\cdot\mathcal{N}\mathop{=\sum\epsilon_p}_{p\in\mathcal{M}\cap\mathcal{N}}\,.\]
    Dies folgt, da f\"ur transversale Mannigfaltigkeiten \(\mathcal{M}\) und \(\mathcal{N}\) die Identit\"at
    \[\eqcl{\mathcal{M}\cap\mathcal{N}\mathrel{|}\mathcal{W}}^*=\eqcl{\mathcal{M}\mathrel{|}\mathcal{W}}^*\smile\eqcl{\mathcal{N}\mathrel{|}\mathcal{W}}^*\in H^n(\mathcal{W})\]
    gilt, und eine Orientierung einer nulldimensionalen Mannigfaltigkeit \(\mathcal{V}\) gerade eine Wahl von \(\pm1\) f\"ur jeden Punkt \(p\in\mathcal{V}\) ist [REF].
}

\section{Immersionen und Whitneys Trick}
    Sei \(f\colon\mathcal{M}\looparrowright\mathcal{N}\) eine Immersion. Dann faktorisiert \(\dx f\) durch die kanonische Abbildung \(f^*T\mathcal{N}\to T\mathcal{N}\) und es existiert ein kommutatives Diagramm
\begin{center}
    \begin{tikzpicture}[scale = 0.8]
        \draw
            (30:1.5) node (A) {\(T\mathcal{N}\)}
            (150:1.5) node (B) {\(T\mathcal{M}\)}
            (270:1.5) node (C) {\(f^*T\mathcal{N}\)}
            ;
            \draw [-stealth] (140:1.5) arc (140:40:1.5) node [pos = 0.5, above] {\(\dx f\)};
            \draw [>-stealth] (160:1.5) arc (160:247:1.5) node [pos = 0.45, right] {\(g\)};
            \draw [-stealth] (295:1.5) arc (295:380:1.5);
    \end{tikzpicture}
\end{center}
sodass sich das Normalenb\"undel von \(f\) als \(\nu(f):=f^*T\mathcal{N}/g(T\mathcal{M})\) definieren l\"asst. Ist \(f\) eine Einbettung, entspricht dies gerade dem bekannten Normalenb\"undel. Analog zu der Schnittzahl zweier transversaler Mannigfaltigkeiten kann die Selbstschnittzahl einer Immersion \(f\colon\mathcal{M}^k\looparrowright\mathcal{N}^{2k}\) definiert werden. Es kann zun\"achst angenommen werden, dass \(f\) sich selbst transversal schneide und lediglich endlich viele Doppelpunkte \(f(p)=f(q)\) besitzt. Definiere \(\epsilon_p\) als \(1\), wenn die zusammengesetzte Orientierung von 
\[T_p\mathcal{N}=\dx_pf\left(T_p\mathbb{S}^k\right)\oplus\dx_qf\left(T_q\mathbb{S}^k\right)\]
der gew\"ahlten Orientierung von \(\mathcal{N}\) entspricht und \(-1\) sonst. Ist \(k\) gerade ist, sei die Selbstschnittzahl
\[I_f\mathop{:=\sum\epsilon_p}_{f(p)=f(q)}\,,\]
ist \(k\) ungerade sei sie eben jene Summe modulo zwei. Besonders wichtig ist diese Zahl in der Verwendung des starken Einbettungssatzes von Whitney, also auch in Whitneys Trick. Siehe Abbildung \ref{fig:whitney_trick}.

\begin{proposition}[Whitneys Trick]\label{prop:whit_trick}
    Sei \(k\geq3\), \(\mathcal{N}^{2k}\) einfach zu\-sam\-men\-h\"an\-gend und \(f\colon\mathcal{M}^k\looparrowright\mathcal{N}\) eine Immersion. \"Ubersteigt die Anzahl der Doppelpunkte von \(f\) die Zahl \(\abs{I_f}\), oder ist \(>0\) f\"ur ungerade \(k\), existiert eine regul\"ar homotope Immersion \(g\), die zwei Doppelpunkte weniger besitzt.
\end{proposition}
\begin{proof}
    Siehe \cite{whitney1944intersect} Satz 4.
\end{proof}

\begin{figure}
    \centering
    \begin{tikzpicture}
        \begin{scope}[xshift = -3cm]
            \draw 
                (2, 0) arc (10:170:2 and 1) 
                    node (A) [pos = 0.2] {\tiny\textbullet}
                    node (B) [pos = 0.8] {\tiny\textbullet}
                    node (C) [pos = 0.45] {}
                (2, 1) arc (-10:-170:2 and 1)
                    node (D) [pos = 0.55] {}
                (C.center) edge [densely dotted, -stealth] ++(0, -1.2)
                (D.center) edge [densely dotted, -stealth] ++(0, 1.2)
                ;
            \draw 
                (A) node [above] {\tiny\(q\)} node [below] {\tiny\(1\)}
                (B) node [above] {\tiny\(p\)} node [below] {\tiny\(-1\)}
            ;

            \fill [pattern = north west lines] 
                (A) arc (42:138:2 and 1) arc (-138:-42: 2 and 1) -- cycle;
        \end{scope}
        \draw (-0.75, 0.5) edge [-stealth] (0.75, 0.5);
        \begin{scope}[xshift = 3cm]
            \draw 
                (2, 0) arc (10:170:2 and 0.5) 
                    node (E) [pos = 0.45] {}
                (2, 1) arc (-10:-170:2 and 0.5)
                    node (F) [pos = 0.55] {}
                (E.center) edge [densely dotted, -stealth] ++(0, -0.75)
                (F.center) edge [densely dotted, -stealth] ++(0, 0.75)
                ;
        \end{scope}
    \end{tikzpicture}
    \caption{Die Idee von Whitneys Trick zur Eliminierung der Doppelpunkte \(p\) und \(q\) einer Immersion \(f\colon\mathbb{S}^1\looparrowright\mathcal{N}\). Der einfache Zusammenhang von \(\mathcal{N}\) garantiert, dass der schraffierte Bereich kontrahierbar ist.}\label{fig:whitney_trick}
\end{figure}

\begin{corollary}\label{cor:imm_reg_hom}
    Sei \(k\geq3\) ungerade, \(\mathcal{N}^{2k}\) einfach zusammenh\"angend und \(f\colon\mathcal{M}^k\looparrowright\mathcal{N}^{2k}\) eine Immersion. Die Selbstschnittzahl von \(f\) ist genau dann null, wenn \(f\) regul\"ar homotop zu einer Einbettung ist.
\end{corollary}

\begin{proposition}\label{prop:imm_inter_zero}
    Es existiert eine Immersion \(h\colon\mathbb{S}^k\looparrowright\mathbb{R}^{2k}\) mit Selbstschnittzahl eins.
\end{proposition}
\begin{proof}
    Siehe \cite{whitney1944intersect} Satz 3.
\end{proof}

\begin{figure}[!h]
    \centering
    \begin{tikzpicture}
        \draw (-0.707, 0.707) arc (45:315:1) -- (0.707, 0.707) arc (135:-135:1) -- cycle;
    \end{tikzpicture}
    \caption{Eine Immersion \(f\colon\mathbb{S}^1\looparrowright\mathbb{R}^2\) mit Selbstschnittzahl eins.}\label{fig:double_imm}
\end{figure}




    \chapter{Verklebung von Mannigfaltigkeiten}
        \section{Allgemeine Verklebungen}
    In der algebraischen Topologie ist es oft relativ einfach, die Homotopiegruppen gezielt zu eliminieren. Betrachte hierzu einen CW-Komplex \(X\) und ein \(\eqcl{\gamma}\in\pi_k(X)\). Dann ergibt das Ankleben einer \((k+1)\)-Zelle entlang der Anklebeabbildung \(\gamma\colon\mathbb{S}^k\to X\) einen CW-Komplex, in welchem \(\gamma\) nullhomotop ist. Die niederen Homotopiegruppen bleiben hierbei gleich, die h\"oheren Homotopiegruppen nicht zwingenderma\ss en. Dieser Prozess versagt f\"ur Mannigfaltigkeiten (insbesonders glatte) auf ganzer Linie. Eine pr\"azise Formulierung einer Verklebung f\"ur glatte Mannigfaltigkeiten l\"asst sich als \textit{etwas haarig} beschreiben. Der naive Ansatz die Verklebung zweier glatte Mannigfaltigkeiten \(\mathcal{M}\) und \(\mathcal{N}\) entlang einer eingebetteten Mannigfaltigkeit \(\mathcal{S}\) \"uber das topologisches Pushout-Diagramm
\begin{center}
    \begin{tikzpicture}
        \draw   (0, 0) node (A) {\(\mathcal{S}\)}
                (0, -1.5) node (B) {\(\mathcal{M}\)}
                (2, 0) node (C) {\(\mathcal{N}\)}
                (2, -1.5) node (D) {\(\mathcal{M}\cup_{\mathcal{S}}\mathcal{N}\)}
                
                (A) edge [-stealth] (B)
                (A) edge [-stealth] (C)
                (B) edge [-stealth] (D)
                (C) edge [-stealth] (D)
                ;
    \end{tikzpicture}
\end{center}
\noindent zu definieren birgt allgemein keinen lokal euklidischen Raum. Selbst wenn sich eine topologische Mannigfaltigkeit ergibt, ist dies keine glatte Mannigfaltigkeit sondern eine \textit{glatte Mannigfaltigkeit mit Ecken}. Diese k\"onnen zwar gegl\"attet werden (siehe Abbildung \ref{fig:smooth_corner}), was jedoch nicht sonderlich elegant ist und eine Reihe von weiteren Problemen einf\"uhrt. F\"ur die korrekte Verklebung m\"ussen unterschiedliche F\"alle unterschieden werden, die Vorgehensweise ist jedoch stets gleich. Seien \(\mathcal{M}\) und \(\mathcal{N}\) Mannigfaltigkeiten mit einer gemeinsamen Untermannigfaltigkeit \(\mathcal{V}\). Dann entstehe die Verklebung von \(\mathcal{M}\) und \(\mathcal{N}\) entlang von \(\mathcal{V}\) durch das Identifizieren von Tubenumgebungen von \(\mathcal{V}\) in \(\mathcal{M}\) und \(\mathcal{N}\) \"uber einen orientierungsumkehrenden Diffeomorphismus. Das Problem besteht darin, dass abgesehen von der Position von \(\mathcal{V}\) in \(\mathcal{M}\) und \(\mathcal{N}\) unterschiedliche Begriffe von Tubenumgebungen betrachtet werden m\"ussen. F\"ur alle n\"otigen Eindeutigkeitsbeweise siehe \cite{kosinski1992differential} Kapitel VI Sektionen 1-5.

\subsection{Tubenumgebungen}
    Sei \(\mathcal{V}^k\hookrightarrow\mathcal{M}^n\) eine Untermannigfaltigkeit. Die folgenden drei Positionen von \(\mathcal{V}\) in \(\mathcal{M}\) sind m\"oglich und am einfachsten zu handhaben.
    \begin{enumerate}
        \item \(\mathcal{V}\subseteq\mathring{\mathcal{M}}\) ist geschlossen,
        \item \(\mathcal{V}\subseteq\partial\mathcal{M}\) ist geschlossen,
        \item \((\mathcal{V},\partial\mathcal{V})\subseteq(\mathcal{M},\partial\mathcal{M})\) ist eine ordentliche Untermannigfaltigkeit.
    \end{enumerate}
    Insbesondere sind auch drei unterschiedliche Begriffe von Tubenumgebungen erforderlich. Siehe Abbildung \ref{fig:tub_neigh}.

    \begin{figure}[t]
        \centering
        \begin{minipage}[t]{.45\textwidth}
            \centering
            \begin{tikzpicture}[scale = 1.3]
                \draw [rounded corners, pattern = north west lines] (0, 0) -- (1, 0) -- (1, 1) -- (0, 1) -- cycle;
                \draw [pattern = north east lines] (1, 0.25) node {\textcolor{red}{\tiny\textbullet}} arc (-90:180:0.75) node {\textcolor{red}{\tiny\textbullet}} -- (0.75, 1) node {\textcolor{red}{\tiny\textbullet}} arc (180:-90:0.25) node {\textcolor{red}{\tiny\textbullet}} -- cycle;
                \begin{scope}[xshift = 2cm]
                    \draw [rounded corners, pattern = north east lines] (0, 0) -- (1, 0) -- (1, 0.25) arc (-90:180:0.75) -- (0, 1) -- cycle;
                    \draw [rounded corners = 0.75, draw = black, fill = white] (1, 1) -- (1, 0.75) arc (-90:180:0.25) -- cycle;
                \end{scope}
            \end{tikzpicture}
            \caption{Abrunden der Ecken durch eine Hom\"oomorphie.}\label{fig:smooth_corner}
        \end{minipage}\hfill
        \begin{minipage}[t]{.52\textwidth}
            \centering
            \begin{tikzpicture}[scale = 1.5]
                \draw [blue, thick]
                    (146.5207:1) arc (146.5207:123.448:1)
                    (56.551:1) arc (56.4509:33.27295:1)
                    (-13.522:1) arc (-13.522:-36.478:1)
                ;
                \draw 
                    (0:1) arc (0:360:1)
                    (45:1) node {\tiny\textbullet} arc (-45:-135:1) node {\tiny\textbullet}

                    (-0.1, -0.3) arc (0:360:0.3)
                    (-25:1) node {\tiny\textbullet}
                ;
                \draw [fill = blue, opacity = 0.2, dashed]
                    (33.27295:1) arc (-45.963:-134.037:1.2)
                    arc (146.5207:123.448:1)
                    arc (-133.5491:-46.4509:0.8)
                    arc (56.4509:33.27295:1) -- cycle
                ;
                \draw [fill = blue, opacity = 0.2, dashed]
                    (-13.522:1) arc (-13.522:-36.478:1)
                    arc (239.2608:70.739:0.2) --cycle
                ;
                \draw [fill = blue, opacity = 0.2, dashed] (0, -0.3) arc (0:360:0.4);
                \fill [white, dashed] (-0.2, -0.3) arc (0:360:0.2);
                \draw [opacity = 0.2, dashed] (-0.2, -0.3) arc (0:360:0.2);
            \end{tikzpicture}
            \caption{Die drei unterschiedlichen Arten von Tubenumgebungen in \(\mathbb{D}^2\).}\label{fig:tub_neigh}
        \end{minipage}
    \end{figure}


    \subsubsection{Untermannigfaltigkeiten des Inneren}
        Sei \(\iota\colon\mathcal{V}^k\hookrightarrow\mathring{\mathcal{M}}\) eine geschlossene Untermannigfaltigkeit und \(0\colon\mathcal{V}\to N\mathcal{V}\) der Nullschnitt. 
        \begin{definition}[Tubenumgebung f\"ur \(\mathcal{V}\subseteq\mathring{\mathcal{M}}\)]
            Ein riemannsches Vektor\-b\"un\-del \(\xi\colon E\to\mathcal{V}\) vom Rang \(n-k\) und eine Einbettung \(\Psi\colon E\hookrightarrow\mathring{\mathcal{M}}\) sodass \(\Psi\circ0=\iota\) ist.
        \end{definition}
        \noindent Es ist ein Standardresultat, dass derartige Tubenumgebungen existieren, bis auf Isotopie eindeutig bestimmt sind und \(\xi\) zu dem Normalenb\"undel isomorph ist.

    \subsubsection{Untermannigfaltigkeiten des Randes}
        Sei \(\iota\colon\mathcal{V}^k\hookrightarrow\partial\mathcal{M}^{n+1}\) eine geschlossene Untermannigfaltigkeit.
        \begin{definition}[Tubenumgebung f\"ur \(\mathcal{V}\subseteq\partial\mathcal{M}\)]
            Eine Tubenumgebung \({\Psi\colon E\hookrightarrow\partial\mathcal{M}}\) mit einer Fortsetzung zu einer Einbettung \({\Tilde{\Psi}\colon E\times\mathbb{R}_+\hookrightarrow\mathcal{M}}\).
        \end{definition}
        \noindent Eine derartige Tubenumgebung l\"asst sich mithilfe von einer Tubenumgebung in \(\partial\mathcal{M}\) und  einer Kragenumgebung \(\partial\mathcal{M}\times\mathbb{R}_+\hookrightarrow\mathcal{M}\) konstruieren und ist somit erneut bis auf Isotopie eindeutig bestimmt. Beachte hierbei, dass das Normalenb\"undel von \(\mathcal{V}\) in \(\mathcal{M}\) kein \textit{Halbraumb\"undel}, sondern ein normales \((n-k)\)-dimensionales Vektorb\"undel ist, weshalb nicht einfach eine Einbettung des Normalenb\"undels von \(\mathcal{V}\) in \(\mathcal{M}\) gew\"ahlt werden kann.

    \subsubsection{Ordentliche Untermannigfaltigkeiten}
        Eine Untermannigfaltigkeit \(\mathcal{V}\subseteq\mathcal{M}\) hei\ss e ordentliche Untermannigfaltigkeit (neat submanifold), wenn:
        \begin{itemize}
            \item[i] Es gelte \(\partial\mathcal{V}=\mathcal{V}\cap\partial\mathcal{M}\)
            \item[ii] F\"ur alle \(p\in\partial\mathcal{M}\) existiert eine Karte \(\alpha\colon\mathbb{R}_+^n\to U\) mit \(\alpha\mathbb{R}_+^k=U\cap\mathcal{V}\).
        \end{itemize}
        \begin{definition}[Ordentliche Tubenumgebung]
            Ein riemannsches Vektor\-b\"un\-del \(\xi\colon E\to\mathcal{V}\) mit einer Einbettung \(\tilde{\Psi}\colon E\hookrightarrow\mathcal{M}\) die mit \(\iota\) kommutiert, sodass die Einschr\"ankung \(\Psi\colon E|_{\partial\mathcal{V}}\hookrightarrow\partial\mathcal{M}\) eine Tubenumgebung von \(\partial\mathcal{V}\) in \(\partial\mathcal{M}\) ist.
        \end{definition}

\subsection{Die Randsumme}
    Die Summe zweier Mannigfaltigkeiten ist stets die gleiche, auch wenn die Tubenumgebungen etwas unterschiedliche Formen besitzen. Die Konstruktion sei hier nur an der Randsumme erl\"autert, die anderen beiden F\"alle verlaufen analog. Sei \({\xi\colon E\to\mathcal{V}}\) ein riemannsches Vektorb\"undel und f\"ur \({i\in\{1,2\}}\) jeweils \({\mathcal{M}_i}\) Mannigfaltigkeiten, \({\iota_i\colon\mathcal{V}\hookrightarrow\partial\mathcal{M}_i}\) Untermannigfaltigkeiten mit Tubenumgebungen \({\psi_i\colon E\hookrightarrow\partial\mathcal{M}_i}\) in \({\partial\mathcal{M}_i}\). Seien \({\Psi_i\colon E\times\mathbb{R}_+\hookrightarrow\mathcal{M}_i}\) Fortsetzungen der \({\psi_i}\) zu Tubenumgebungen in \({\mathcal{M}_i}\). Sei zuletzt die diffeomorphe Involution
    \[\alpha\colon(E\times\mathbb{R}_+)\setminus\mathbf{0}\to(E\times\mathbb{R}_+)\setminus\mathbf{0},\,v\mapsto\frac{v}{\norm{v}^2}\]
    gegeben, so identifiziere
    \[\Psi_1(x)\sim\Psi_2\alpha(x)\quad\forall x\in(E\times\mathbb{R}_+)\setminus\mathbf{0}\,,\]
    und setze
    \[\mathcal{M}_1\mathop{+}_{\Psi_1}^{\Psi_2}\mathcal{M}_2=\left(\mathcal{M}_1\setminus\iota_1(\mathcal{V})\sqcup-\mathcal{M}_2\setminus\iota_2(\mathcal{V})\right)/\sim\,.\]
    Diese Konstruktion ist bis auf Diffeomorphie unabh\"angig von der Wahl der Fortsetzung der \({h_i}\). Weiter ergeben isotope Einbettungen diffeomorphe Mannigfaltigkeiten. Beachte, dass unterschiedliche \(\Psi\) allgemein jedoch sehr wohl den Diffeomorphietyp der Summe ver\"andern kann.

    \subsubsection{Die verbundene Summe}
        Ein Spezialfall tritt auf, wenn \(\mathcal{V}\) ein Punkt ist. In diesem Fall wird die Summe zweier Mannigfaltigkeiten auch verbundene Summe genannt. Siehe Abbildung \ref{fig:conn_sum}.
        \begin{figure}
            \centering
            \begin{tikzpicture}[scale = 0.65]
                \begin{scope}[xshift = -2cm]
                    \draw [-stealth] (45:1) arc (45:225:1);
                    \draw [-stealth] (-135:1) arc (-135:45:1);
                    \draw (1, 0) node {\tiny\textbullet} ++(-0.3, 0) node {\(x\)};
                    
                    \draw [-stealth] (225:2) arc (225:45:2);
                    \draw [-stealth] (45:2) arc (45:-135:2);
                    \draw (2, 0) node {\tiny\textbullet} ++(0.3, 0) node {\(x\)};

                    \draw (0, 1.4) node {\(\mathcal{M}\)};
                    \draw (0, 2.5) node {\(-\mathcal{M}\)};
                \end{scope}
                
                \begin{scope}[xshift = 3.5cm]
                    \draw [-stealth] 
                        (60:0.75) arc (60:300:0.75);
                    \draw [-stealth] 
                        (225:2) arc (225:45:2);
                    \draw [-stealth]
                        (45:2) node (A) {}
                        ++(0.5, -0.5) node (B) {}
                        ++(-0.5, -0.25) node (C) {}
                        ++(-1, 0.33) node (D) {}
                        (A.center) .. controls (B.center) and (C.center) .. (D.center) node [pos = 0.4] (BB) {}
                        ;
                    \path
                        (-45:2) node (H) {}
                        ++(0.5, 0.5) node (G) {}
                        ++(-0.5, 0.25) node (F) {}
                        ++(-1, -0.33) node (E) {};
                    \draw [-stealth]
                        (E.center) .. controls (F.center) and (G.center) .. (H.center) node [pos = 0.6] (AA) {};
                    \draw [-stealth] (H) arc (-45:-135:2);
                    \draw [-stealth] (60:0.75) node (I) {}
                        ++(0.3248, -0.1875) node (J) {}
                        ++(0.5, 0) node (K) {}
                        ++(0.5, 0.1) node (L) {}
                        (L.center) .. controls (K.center) and (J.center) .. (I.center)
                    ;
                    \draw [-stealth] (-60:0.75) node (M) {}
                        ++(0.3248, 0.1875) node (N) {}
                        ++(0.5, 0) node (O) {}
                        ++(0.5, -0.1) node (P) {}
                        (M.center) .. controls (N.center) and (O.center) .. (P.center)
                    ;
                    
                    \draw [dotted] 
                        (290:0.75) -- (E.center)
                        (70:0.75) -- (D.center)
                        (AA.center) -- (P.center)
                        (BB.center) -- (L.center)
                        (-0.5, 1) node {\(\mathcal{M}\setminus x\)} 
                        (-0.5, 2.5) node {\(-\mathcal{M}\setminus x\)};
                \end{scope}
            \end{tikzpicture}\hfill
            \begin{tikzpicture}[scale = 0.5]
                    \draw 
                        (45:2) node (A) {}
                        ++(0.5, -0.5) node (B) {}
                        ++(0.5, -0.25) node (C) {}
                        ++(1, 0.33) node (D) {}
                        
                        (-45:2) node (H) {} 
                        ++(0.5, 0.5) node (G) {}
                        ++(0.5, 0.25) node (F) {}
                        ++(1, -0.33) node (E) {}

                        (0, 2.5) node {\(\mathcal{M}\setminus x\)}
                        ;
                    \draw [-stealth] (D.center) .. controls (C.center) and (B.center) .. (A.center);
                    \draw [-stealth] (A.center) arc (45:315:2);
                    \draw (315:2) .. controls (G.center) and (F.center) .. (E.center); % Large lower
                    \draw [-stealth] (0, 0) -- (0, 0.5); 
                    \draw [-stealth] (0, 0) -- (0.5, 0);
                    \fill [opacity = 0.2] (D.center) .. controls (C.center) and (B.center) .. (A.center) arc (45:315:2) .. controls (G.center) and (F.center) .. (E.center) -- cycle;
                    \begin{scope}[xshift = 5.7cm, scale = 1.2]
                        \draw [-stealth] (-135:2) arc (-135:135:2);
                        \draw (135:2) node (A1) {}
                            ++(-0.5, -0.5) node (B1) {}
                            ++(-0.5, -0.25) node (C1) {}
                            ++(-0.7, 0.3) node (D1) {}
                            
                            (-135:2) node (E1) {}
                            ++(-0.5, 0.5) node (F1) {}
                            ++(-0.5, 0.25) node (G1) {}
                            ++(-0.7, -0.3) node (H1) {}
                            
                            (0, 2.4) node {\(-\mathcal{M}\setminus x\)}
                            ;
                        \draw (A1.center) .. controls (B1.center) and (C1.center) .. (D1.center);
                        \draw [-stealth] (H1.center) .. controls (G1.center) and (F1.center) .. (E1.center);
                        
                        \draw [dashed] 
                            (D1.center) -- (H1.center) node [pos = 0.5] {\tiny\textbullet} 
                            (D.center) -- (E.center) node [pos = 0.5] {\tiny\textbullet};
                        \draw [-stealth] (0, 0) -- (0, 0.5); 
                        \draw [-stealth] (0, 0) -- (0.5, 0);
                        \fill [opacity = 0.2] (H1.center) .. controls (G1.center) and (F1.center) .. (E1.center) node (X) [pos = 0.3] {} arc (-135:135:2) .. controls (B1.center) and (C1.center) .. (D1.center) node (XX) [pos = 0.7] {} -- cycle;
                    \end{scope}
                    \draw (X.center) -- (XX.center) node [above] {\(\xi_+^1\)}; 
            \end{tikzpicture}
            \caption{Die verbundenen Summe zweier Kreise und die verbundene Rand\-sum\-me zweier Scheiben.}\label{fig:conn_sum}
        \end{figure}

    \subsection{Homologie von Verklebungen am Rand}
        Sei \(\mathcal{V}^i\) eine gemeinsame Untermannigfaltigkeit der R\"ander von \(\mathcal{M}^n\) und \(\mathcal{N}^n\) mit trivialen Normalenb\"undeln. Seien \(A_1\) und \(A_2\) die Bilder von \(\mathcal{M}\setminus\mathcal{V}\) und \(\mathcal{N}\setminus\mathcal{V}\) in \(\mathcal{M}\mathop{+}^{\mathcal{V}}\mathcal{N}\). Dann ist die zugeh\"orige Mayer-Vietoris Folge von der Form
        \[H_*(A_1\cap A_2)\to H_*(A_1)\oplus H_*(A_2)\to H_*(\mathcal{M}\mathop{+}^{\mathcal{V}}\mathcal{N})\to H_{*-1}(A_1\cap A_2)\,.\]
        Hierbei ist \(A_1\cap A_2\) zu einer Tubenumgebung von \(\mathcal{V}\) isomorph, aus welcher \(\mathcal{V}\) entfernt wurde. Da eine triviale Tubenumgebung in diesem Fall von der Form \(\mathcal{V}\times\mathbb{R}^{n-i-1}\times\mathbb{R}_+\) ist, enth\"alt \({\mathcal{V}\times\mathbb{R}^{n-i-1}\times\mathbb{R}_+\setminus\mathbf{0}}\) den Raum \({\mathcal{V}\times\mathbb{S}_+^{n-i-1}}\) als Deformationsretrakt, also folgt
        \[H_*(A_1\cap A_2)\cong H_*\left(\mathcal{V}\right)\quad\text{und per Ausschneidung}\quad H_*(\mathcal{M},\mathcal{M}\setminus\mathcal{V})=0\,.\]
        Aus \(H_*(\mathcal{M},\mathcal{M}\setminus\mathcal{V})=0\) ergibt sich \(H_*(A_1)\cong H_*(\mathcal{M})\). Die Mayer-Vietoris-Folge ist also
        \[H_*\left(\mathcal{V}\right)\to H_*(\mathcal{M})\oplus H_*(\mathcal{N})\to H_*(\mathcal{M}\mathop{+}^{\mathcal{V}}\mathcal{N})\to H_{*-1}\left(\mathcal{V}\right)\,.\]
        Beispielsweise folgt daraus f\"ur die verbundene Randsumme f\"ur \(j>0\)
        \[H_j(\mathcal{M}+\mathcal{N})\cong H_j(\mathcal{M})\oplus H_j(\mathcal{N})\,.\]
        Beachte, dass dies \textbf{nicht} f\"ur die normale Randsumme entlang des Inneren gilt. In diesem Fall gilt \(H_*(A_1\cap A_2)\cong H_*(\mathbb{S}^{n-1})\), also ist 
        \[H_j(\mathcal{M}+\mathcal{N})\cong H_j(\mathcal{M})\oplus H_j(\mathcal{N})\]
        lediglich f\"ur \(n-1>j>0\). F\"ur orientierte Mannigfaltigkeiten \(\mathcal{M}\) und \(\mathcal{N}\) gilt dies auch f\"ur \(j=n-1\), da der Homomorphismus \(H_n(\mathcal{M}+\mathcal{N})\to H_{n-1}(\mathbb{S}^{n-1})\) dann ein Isomorphismus ist.
        \begin{example}
            Seien \(\mathcal{M}^{2k}\) und \(\mathcal{N}^{2k}\) Mannigfaltigkeiten, die die gleiche Homotopiesph\"are \(\Sigma\) beranden. Dann ist die Verklebung von \(\mathcal{M}\) und \(\mathcal{N}\) entlang \(\Sigma\) eine geschlossene Mannigfaltigkeit, und aus \(H_k(\Sigma)=H_{k-1}(\Sigma)=0\) folgt
            \[H_k(\mathcal{M}\mathop{+}^{\Sigma}\mathcal{N})\cong H_k(\mathcal{M})\oplus H_k(\mathcal{N})\,.\]
        \end{example}
        

     
    \chapter{Etwas Homotopietheorie}
        Verklebungen sind von besonders einfacher Natur, wenn das Normalenb\"undel der geteilten Mannigfaltigkeit trivial ist. Um dies zu entscheiden ist der Begriff der Kupplungsfunktion n\"otig. Durch diese l\"asst sich die Frage der Trivialit\"at eines Vektorb\"undels \"uber einer Sph\"are auf ein algebraisches Element in \(\eqcl{\gamma}\in\pi_k(\operatorname{SO}(n))\) reduzieren. Somit liegt es nahe, die Struktur dieser Gruppe zu untersuchen. Das kleine Problem besteht hierbei darin, dass die Berechnung dieser Gruppen im allgemeinen spektakul\"ar aufw\"andig ist. Die niedrigdimensionalen Beispiele sind \cite{lundell1992tables}:
\begin{center}
    \begin{tabular}{c|cccccccccc}
         & \(\pi_1\)         & \(\pi_2\) & \(\pi_3\)                     & \(\pi_4\)                         & \(\pi_5\) & \(\pi_6\) & \(\pi_7\) & \(\pi_8\) & \(\pi_9\) & \(\pi_{10}\)\\\hline\\[-12pt]
        \footnotesize\(\operatorname{SO}(3)\)& \(\mathbb{Z}_2\)  & \(0\)     & \(\mathbb{Z}\)                & \(\mathbb{Z}_2\)                  & \(\mathbb{Z}_2\) & \(\mathbb{Z}_{12}\) & \(\mathbb{Z}_2\) & \(\mathbb{Z}_2\) & \(\mathbb{Z}_3\) & \(\mathbb{Z}_{15}\)\\
        \footnotesize\(\operatorname{SO}(4)\)& \(\mathbb{Z}_2\)             & \(0\)     & \(\mathbb{Z}^{\oplus2}\)& \(\mathbb{Z}_2^{\oplus2}\)& \(\mathbb{Z}_2^{\oplus2}\) & \(\mathbb{Z}_{12}^{\oplus2}\) & \(\mathbb{Z}_2^{\oplus2}\) & \(\mathbb{Z}_2^{\oplus2}\) &  \(\mathbb{Z}_3^{\oplus2}\) & \(\mathbb{Z}_{15}^{\oplus2}\)\\
        \footnotesize\(\operatorname{SO}(5)\)& \(\mathbb{Z}_2\)             & \(0\)     & \(\mathbb{Z}\)                & \(\mathbb{Z}_2\)                  & \(\mathbb{Z}_2\) & \(0\) & \(\mathbb{Z}\) & \(0\) & \(0\) & \(\mathbb{Z}_{120}\)\\
        \footnotesize\(\operatorname{SO}(6)\)& \(\mathbb{Z}_2\)             & \(0\)     & \(\mathbb{Z}\)                         & \(0\)                             & \(\mathbb{Z}\) & \(0\) & \(\mathbb{Z}\) & \(\mathbb{Z}_{24}\) & \(\mathbb{Z}_2\) & \(\mathbb{Z}_2\oplus\mathbb{Z}_{120}\)\\
        \footnotesize\(\operatorname{SO}(7)\)& \(\mathbb{Z}_2\)             & \(0\)     & \(\mathbb{Z}\)                         & \(0\)                             & \(0\) & \(0\) & \(\mathbb{Z}\) & \(\mathbb{Z}_2^{\oplus2}\) & \(\mathbb{Z}_2^{\oplus2}\) & \(\mathbb{Z}_8\)\\
    \end{tabular}
\end{center} 
Es l\"asst sich erkennen dass sich die \(\pi_k(\operatorname{SO}(n))\) f\"ur \(n\geq k+2\) \textit{stabilisieren}, es gilt also \(\pi_k(\operatorname{SO}(n))\cong\pi_k(\operatorname{SO}(n+1))\). Dies f\"uhrt zu dem Begriff der \textbf{stabilen Homotopiegruppen} von \(\operatorname{SO}(n)\). Definiere durch die nat\"urliche Inklusion
\[\operatorname{SO}:=\underrightarrow{\operatorname{colim}}\operatorname{SO}(n)\,,\]
so sind die Homotopiegruppen durch den Periodizit\"atssatz von Bott \cite{bott1959classical} komplett bestimmt. Es gilt:
\begin{center}
    \begin{tabular}{c|cccccccc}
         \(n\operatorname{mod}8\) & 0 & 1 & 2 & 3 & 4 & 5 & 6 & 7\\\hline\\[-12pt]
        \(\pi_n(\operatorname{SO})\)& \(\mathbb{Z}_2\) & \(\mathbb{Z}_2\) & \(0\) & \(\mathbb{Z}\) & \(0\) & \(0\) & \(0\) & \(\mathbb{Z}\)
    \end{tabular}
\end{center}

\newpage
\section{Vektorbündel \"uber Sph\"aren}
    Die wichtige verbleibende Frage besteht nun darin zu entscheiden, wann ein B\"undel trivial ist. F\"ur Vektorb\"undel \"uber Sph\"aren ist dies etwas einfacher als f\"ur beliebige andere R\"aume. Sei \(\xi\colon E\to\mathbb{S}^{k+1}\) ein orientiertes Vektorb\"undel vom Rang \(n\). Schreibe \({\mathbb{S}^{k+1}=H_1\cup H_2}\) als Vereinigung der oberen und unteren Hemisph\"are. Beachte hierbei \({H_1\cap H_2=\mathbb{S}^k}\). Da die \(H_i\) kontrahierbar sind, sind die Vektorb\"undel \(E|_{H_i}\) jeweils trivial. Seien
\[h_i\colon E|_{H_i}\to H_i\times\mathbb{R}^n\]
Trivialisierungen. Dann existiert eine Funktion \(h_{\xi}\colon\mathbb{S}^k\to\operatorname{GL}^+(n)\), sodass f\"ur alle \(x\in\mathbb{S}^k\) und \(y\in\mathbb{R}^n\)
\[h_2^{\phantom{}}h_1^{-1}(x,y)=(x,h_{\xi}(x)\cdot y)\]
gilt. Bezeichne diese Funktion \(h_{\xi}\) als \textbf{Kupplungsfunktion} (engl. Clutching-Func\-tion) von \(\pi\). Umgekehrt kann zu einer beliebigen Funktion \(f\colon\mathbb{S}^k\to\operatorname{GL}^+(n)\) ein orientiertes Vektorb\"undel \"uber \(\mathbb{S}^{k+1}\) gebildet werden, indem zwei triviale B\"undel \"uber den \(H_i\) mithilfe von \(f\) verklebt werden. Setze also
\[\left(H_1\times\mathbb{R}^n\sqcup H_2\times\mathbb{R}^n\right)/\left(\forall x\in\mathbb{S}^k\colon(x,y)\sim(x,f(x)\cdot y)\right)\]
mit der naheliegenden Projektion. Es l\"asst sich zeigen, dass homotope Kupplungsfunktionen mit orientiert isomorphen Vektorb\"undeln korrespondieren, sodass die Isomorphie
\begin{equation}
    \operatorname{Vekt}_n^+(\mathbb{S}^{k+1})\cong\left[\mathbb{S}^k,\operatorname{GL}^+(n)\right]\cong\pi_k\left(\operatorname{GL}^+(n)\right)\cong\pi_k\left(\operatorname{SO}(n)\right)
\end{equation}
gilt. Siehe zum Beispiel \cite{knapp2013vektorbuendel} Satz 3.1.11. Insbesondere ist ein Vektorb\"undel genau dann trivial, wenn die Kupplungsfunktion nullhomotop ist.

\begin{example}[M\"obius-Band]
    Das anschaulichste Beispiel einer nicht trivialen Kupplungsfunktion ist das (offene) M\"o\-bius\-band, auch wenn dieses nicht orientierbar ist und somit streng genommen nicht unter die obere Definition f\"allt. Hierbei werden zwei triviale Vektorb\"undel \(\underline{\mathbb{R}}\) \"uber \(\mathbb{D}^1\) entlang \(\partial\mathbb{D}^1=\mathbb{S}^0\) verklebt. Die Kupplungsfunktion ist dabei
    \[\phi\colon\mathbb{S}^0\to\operatorname{O}(1),\,x\mapsto\begin{cases}
        \mathbbm{1} & x=1\\
        -\mathbbm{1} & x=-1
    \end{cases}\]
    also nicht nullhomotop, sodass das B\"undel nicht trivial ist. 
    \[\mathbb{M}:=\left(\mathbb{D}\times\mathbb{R}\sqcup\mathbb{D}\times\mathbb{R}\right)/\left(\forall x\in\mathbb{S}^0\times\mathbb{R}\colon(x,y)_1\sim(x,\phi(x)\cdot y)_2\right)\,.\]
    Beachte, dass dieses B\"undel \textbf{nicht} orientierbar ist. Dies ver\"andert die Diskussion der Kupplungsfunktionen, da \(\operatorname{GL}\) im Gegensatz zu \(\operatorname{GL}^+\) zwei Zusammenhangskomponenten besitzt.
\end{example}

\begin{figure}
    \centering
    \begin{tikzpicture}[scale = 0.5]
        \begin{scope}[xshift = 0.35cm]
            \draw 
                (-90:1) arc (-90:90:1)
                (-90:2) arc (-90:90:2)
                (-90:3) arc (-90:90:3)
            \foreach\i in {-90,-80, ..., 90} {
                (\i:2) +(\i:-1) -- +(\i:1)
            };
        \end{scope}
        
        \begin{scope}[xshift = -0.35cm]
            \draw 
                (90:1) arc (90:270:1)
                (90:2) arc (90:270:2)
                (90:3) arc (90:270:3)
            \foreach\i in {90, 100, ..., 270} {
                (\i:2) +(\i:-1) -- +(\i:1)
            };
        \end{scope}
        \draw [dotted] 
            (0.35, 3) -- (-0.35, 3)
            (0.35, 2) -- (-0.35, 2)
            (0.35, 1) -- (-0.35, 1)
            (0.35, -1) -- (-0.35, -3)
            (0.35, -2) -- (-0.35, -2)
            (0.35, -3) -- (-0.35, -1)
            ;
    \end{tikzpicture}
    \caption{Das M\"obius-Band als nicht-triviales Vektorb\"undel \"uber der \(1\)-Sph\"are. Beachte, dass dieses nicht orientiert ist.}
\end{figure}

\section{Rahmungen}
    Um einen \(\mathbb{R}\)-Vektorr\"aum \(V\) effektiv betrachten zu k\"onnen, ist es stets n\"otig eine Basis \(\mathcal{B}\) zu w\"ahlen. Diese etabliert eine Vektorraumisomorphie \(\mathbb{R}^n\cong V\) und reduziert die Komplexit\"at der abstrakten Struktur von \(V\) auf den einfachst m\"oglichen Fall und etabliert einen \textit{Referenzrahmen}. Die Verallgemeinerung einer Basis auf Vektorr\"aume ist naheliegend, und besteht darin, dass auf stetige Art und Weise jeder Faser eine Basis zugeordnet wird. Hierbei existieren drei \"aquivalente Definitionen
\begin{definition}[Rahmen eines Vektorb\"undels]
    Sei \(\xi\colon E\to B\) ein Vektorb\"undel. Ein Rahmen ist 
    \begin{itemize}
        \item[i] eine Trivialisierung \(\xi\cong\underline{\mathbb{R}}^k\),
        \item[ii] die Wahl \(k\) linear unabh\"angiger Schnitte \(X_i\colon B\to E\) oder
        \item[iii] ein Schnitt im Rahmenb\"undel von \(\xi\).
    \end{itemize}
\end{definition}
Im Gegensatz zu Vektorr\"aumen, muss ein Rahmen keineswegs existieren. Viel eher ist die Existenz eines Rahmens eine starke Einschr\"ankung, da dies bereits impliziert, dass \(\xi\) trivial ist. Weiter sind unterschiedliche Rahmen nicht unbedingt zueinander \"aquivalent. Ein Vektorb\"undel zusammen mit einem Rahmen hei\ss e \textbf{gerahmtes Vektorb\"undel}. Die Wahl einer Rahmung von \(\xi\oplus\underline{\mathbb{R}}\) liefert eine \textbf{stabile Rahmung}. Wenn \(\xi\) zus\"atzlich orientierbar ist, ist es m\"oglich von orientierten Rahmungen zu sprechen. Im Folgenden seien alle Rahmungen orientiert. Seien \(F\) und \(G\) zwei Rahmungen eines Vektorb\"undels \(\xi\colon E\to B\). Dann sind sowohl \(F(p)\) als auch \(G(p)\) Basen von \(\xi^{-1}(p)\), und es kann der Basiswechsel \(M_{G(p)}^{F(p)}\in\operatorname{GL}^+(n)\) von \(F(p)\) zu \(G(p)\) betrachtet werden. Dies liefert eine stetige Funktion
\[M_G^F\colon B\to\operatorname{GL}^+(n),\,p\mapsto M_{G(p)}^{F(p)}\,,\]
den \textbf{Rahmenwechsel} von \(F\) zu \(G\). 
\begin{example}[Rahmenb\"undel der Sph\"are]\label{ex:framebundle_sphere}
    Der Tangentialvektorraum von \(\mathbb{S}^n\) an \(p\) kann als Untervektorraum des \(\mathbb{R}^{n+1}\) durch \(T_p\mathbb{S}^n\cong\{p\}^{\perp}\) identifiziert werden. Eine gegebene Orthonormalbasis \(b_i\) von \(T_p\mathbb{S}^n\subset\mathbb{R}^{n+1}\) kann also stets durch \(p\) zu einer Orthonormalbasis des \(\mathbb{R}^{n+1}\) erg\"anzt werden. Somit liegt die Matrix \(A:=(p,b_1,\dots,b_n)^{\intercal}\) in \(\operatorname{SO}(n+1)\) und es gilt \(A\cdot e_1=p\). Umgekehrt beschreibt eine Matrix \(B\in\operatorname{SO}(n+1)\) mit \(B\cdot e_1=p\) eine Orthonormalbasis von \(T_p\mathbb{S}^n\). Diese Korrespondenz liefert einen Hauptfaserb\"undelisomorphismus des orientiert orthonormalen Rahmenb\"undels der Sph\"are zu \(\operatorname{SO}(n+1)\). Weiter stiftet die Einh\"angung
    \[SA=\begin{pmatrix}
        A & 0\\
        0 & 1
    \end{pmatrix}\]
    eine Inklusion \(S\colon\operatorname{SO}(n)\hookrightarrow\operatorname{SO}(n+1)\) und f\"uhrt zu der im Folgenden au\ss erordentlich wichtigen Faserung
    \[\operatorname{SO}(n)\xrightarrow{S}\operatorname{SO}(n+1)\xrightarrow{\psi}\mathbb{S}^n\,.\]
    Eine tangentiale Rahmung der Sph\"are existiert nur f\"ur \(n\in\{1,3,7\}\), da nur in diesem Fall nullteilerfreie Multiplikationen auf dem \(\mathbb{R}^{n+1}\) existieren.
\end{example}

\subsection{Parallelisierbarkeit}
    Sei \(\mathcal{V}\) eine Mannigfaltigkeit. Dann sind das Tangentialb\"undel und gegebenenfalls auch das Normalenb\"undel definiert. Die Frage nach der Trivialit\"at dieser B\"undel erm\"oglicht einige Aussagen \"uber \(\mathcal{V}\) und m\"oglichweise den umliegenden Raum. \(\mathcal{V}\) hei\ss e (stabil) tangential oder normal gerahmt, wenn \(T\mathcal{V}\) oder \(N\mathcal{V}\) mit einer (stabilen) Rahmung versehen sind. Eine Mannigfaltigkeit mit stabil trivialem Tangentialb\"undel hei\ss e auch \textbf{\(\pi\)-Mannigfaltigkeit}. Die Signifikanz von \(\pi\)-Mannigfaltigkeiten der Dimension \(n\) f\"ur die Chirurgietheorie besteht darin, dass eingebettete \(k\)-Sph\"aren mit \(n\geq2k\) ein stabil triviales Normalenb\"undel besitzen, die Frage nach der eigentlichen Trivialit\"at etwas vereinfacht.

    \begin{theorem}\label{thm:vec_dim_triv}
        Sei \(\xi\) ein Vektorb\"undel vom Rang \(n\) \"uber einem \(k\)-dimensionalen CW-Komplex \(X\) mit \(n>k\). Dann ist \(\xi\) genau dann stabil trivial, wenn \(\xi\) bereits trivial ist.
    \end{theorem}
    \begin{proof}
        Sei \(F\colon\xi\oplus\underline{\mathbb{R}}\to\underline{\mathbb{R}}^{n+1}\) eine Trivialisierung. Diese liefert in jeder Faser eine Einbettung \(\xi^{-1}(p)\hookrightarrow\mathbb{R}^{n+1}\), sei etwa
        \[\xi^{-1}(p)\cong\{f(p)\}^{\perp}\cong T_{f(p)}\mathbb{S}^n\]
        f\"ur eine stetige Abbildung \(f\colon X\to\mathbb{S}^n\), die wegen \(n>k\) nullhomotop sein muss. Andererseits gilt nun \(\xi\cong f^*(T\mathbb{S}^n)\), sodass \(\xi\) bereits trivial sein muss.
    \end{proof}

    \begin{theorem}\label{thm:sub_pi_triv}
        Eine Untermannigfaltigkeit \(\mathcal{V}^k\) einer \(\pi\)-Mannigfaltigkeit \(\mathcal{M}^n\) mit \(n\geq2k\) ist genau dann eine \(\pi\)-Mannigfaltigkeit, wenn ihr Normalenb\"undel stabil trivial ist.
    \end{theorem}
    \begin{proof}
        Aus einer Zerlegung 
        \[T\mathcal{V}\oplus N\mathcal{V}\cong T\mathcal{M}|_{\mathcal{V}}\,,\]
        folgt
        \[T\mathcal{V}\oplus N\mathcal{V}\oplus\underline{\mathbb{R}}\cong T\mathcal{M}|_{\mathcal{V}}\oplus\underline{\mathbb{R}}\cong\underline{\mathbb{R}}^{n+1}\,.\]
        Somit ist einerseits \(T\mathcal{V}\oplus\underline{\mathbb{R}}^{n-k+1}\) trivial, wenn \(N\mathcal{V}\oplus\underline{\mathbb{R}}\) es ist, und andererseits \(N\mathcal{V}\oplus\underline{\mathbb{R}}^{k+1}\) trivial, wenn \(T\mathcal{V}\oplus\underline{\mathbb{R}}\) es ist. Aus Satz \ref{thm:vec_dim_triv} folgt, dass dann jeweils \(T\mathcal{V}\) und \(N\mathcal{V}\) stabil trivial sind.
    \end{proof}

    \begin{corollary}\label{cor:subsphere_stable}
        Jede in eine \(\pi\)-Mannigfaltigkeit eingebettete Sph\"are \(\mathcal{S}^k\hookrightarrow\mathcal{M}^n\) mit \(n\geq2k\) besitzt ein stabil triviales Normalenb\"undel.
    \end{corollary}
    \begin{proof}
        Das folgt aus Satz \ref{thm:sub_pi_triv}, da alle Sph\"aren \(\pi\)-Mannigfaltigkeiten sind \ref{cor:hom_pi}.
    \end{proof}
    

\section{Die Einh\"angung}
    Das Normalenb\"undel einer Mannigfaltigkeit \(\mathcal{M}^n\) ist stets von dem umliegenden Raum abh\"angig. Der Einbettungssatz von Whitney garantiert Einbettungen \(\mathcal{M}\hookrightarrow\mathbb{R}^m\) f\"ur \(m\geq2n\). F\"ur \(m\geq2n+2\) ist diese Einbettung bis aus Isotopie eindeutig bestimmt. Die Normalenb\"undel zweier solcher isotoper Einbettungen in einen \(\mathbb{R}^m\) sind isomorph, das Normalenb\"undel \textit{stabilisieriert} sich also. Dies motiviert den Begriff der Stabilisierung von Vektorb\"undeln.

\subsection{Die Einh\"angung}
    Sei \(f\in\operatorname{SO}(n)\). Dieses Element kann einerseits als stetige Abbildung
    \[f\colon\mathbb{S}^{n-1}\to\mathbb{S}^{n-1}\quad\text{mit}\quad Sf\colon S\mathbb{S}^{n-1}\to S\mathbb{S}^{n-1}\,,\]
    andererseits aber auch als lineare Abbildung
    \[f\colon\mathbb{R}^n\to\mathbb{R}^n\quad\text{mit}\quad Sf\colon\mathbb{R}^n\oplus\mathbb{R}\to\mathbb{R}^n\oplus\mathbb{R},\,x\mapsto\begin{pmatrix}
        A & 0\\
        0 & 1
    \end{pmatrix}\cdot x\,.\]
    verstanden werden. Jeweils ist \(Sf\in\operatorname{SO}(n+1)\) und definiert eine Inklusion \(S\colon\operatorname{SO}(n)\hookrightarrow\operatorname{SO}(n+1)\). Diese induziert offenbar einen Homomorphismus
    \[\pi_k(\operatorname{SO}(n))\xrightarrow{S_*}\pi_k(\operatorname{SO}(n+1))\,.\]
    Die gesamte Konstruktion erkl\"art einen Zusammenhang zwischen der topologischen Einh\"angung und der direkten Summe mit \(\mathbb{R}\). Zu einem orientierbaren Vektorb\"undel \(\xi\colon E\to\mathbb{S}^{k+1}\) mit Kupplungsfunktion \(\gamma\colon\mathbb{S}^k\to\operatorname{SO}(n)\) l\"asst sich diese Korrespondenz derart beschreiben, dass das Vektorb\"undel \(\xi\oplus\underline{\mathbb{R}}\to\mathbb{S}^{k+1}\), die Kupplungsfunktion \(S\gamma\) besitzt. 

\subsection{Die Stabilisierungsfolge der Sph\"are}\label{subsec:standard_frame_seq}
    Wie in Beispiel \ref{ex:framebundle_sphere} ist das Rahmenb\"undel der Sph\"are gerade \(\operatorname{SO}(n+1)\to\mathbb{S}^n\). Die Faser ist hierbei \(\operatorname{SO}(n)\). Dies liefert die wichtige Faserung
    \[\operatorname{SO}(n)\xrightarrow{S}\operatorname{SO}(n+1)\xrightarrow{\psi}\mathbb{S}^n\,,\]
    wobei \(S\) erneut die Einh\"angung bezeichne, und \(\psi(A)=A\cdot e_1\) sei. Zugeh\"orig zu jeder Faserung ist eine lange exakte Folge
    \[\pi_{k+1}\left(\mathbb{S}^n\right)\xrightarrow{\partial}\pi_k\left(\operatorname{SO}(n)\right)\xrightarrow{S_*}\pi_k\left(\operatorname{SO}(n+1)\right)\xrightarrow{\psi_*}\pi_k\left(\mathbb{S}^n\right)\xrightarrow{\partial}\pi_{k-1}\left(\operatorname{SO}(n)\right)\,.\]
    Diese ist besonders f\"ur \(k=n\) von Interesse. In diesem Fall wird der Generator \(\eqcl{\mathbbm{1}}\in\pi_n\left(\mathbb{S}^n\right)\) durch \(\partial\) auf die Kupplungsfunktion des Tangentialb\"undels abgebildet. Siehe auch \cite{knapp2013vektorbuendel} Korollar 3.3.4. Die Idee ist, dass sich die Identit\"at zu einer Funktion \(\mathbb{D}^n\to\operatorname{SO}(n+1)\) anheben l\"asst, die zu einer Rahmung des Tangentialb\"undels \"uber der oberen Hemisph\"are \(\mathbb{S}_+^n\cong\mathbb{D}^n\) homotop ist. Die Einschr\"ankung dieser Anhebung auf \(\mathbb{S}^{n-1}\) ist dann gerade der Rahmenwechsel dieser Rahmung zu der Standardrahmung der unteren Hemisph\"are. Ihre Homotopieklasse ist per Definition \(\eqcl{\tau_n}\). Der Kern \(\ker S_*\cong\im\partial\) wird also von \(\eqcl{\tau_n}\) erzeugt. 

\subsection{Normalenb\"undel eingebetteter Sph\"aren}\label{subsec:stable_normal_bundle}
    Sei \({n\geq2k}\), \(\mathcal{M}^n\) eine \(\pi\)-Mannigfaltigkeit und \({\mathcal{S}^k\hookrightarrow\mathcal{M}^n}\) eine eingebettete Sph\"a\-re, die gem\"a\ss{} Korollar \ref{cor:subsphere_stable} ein stabil triviales Normalenb\"undel \(\nu\) besitzt. Es gelten also 
    \[0=\eqcl{\nu\oplus\underline{\mathbb{R}}}=S_*\eqcl{\nu}\,\quad\text{beziehungsweise}\quad\eqcl{\nu}\in\ker S_*\,.\]
    Dann folgt aus dem Vorangegangenen, dass \(\eqcl{\nu}\) ein Vielfaches von \(\eqcl{\tau_k}\) ist. Weitere Kenntnisse \"uber den Rang von \(\eqcl{\tau_k}\) liefern
    \[\eqcl{\nu}\in\ker S_*\cong\begin{cases}
        \mathbb{Z} & k=2m\\
        \mathbb{Z}_2 & k=2m+1\notin\{1,3,7\}\\
        0 & k\in\{1,3,7\}
    \end{cases}\,.\]
    Siehe hierzu auch \cite{knapp2013vektorbuendel} Abschnitt 3.3.1.
    
\newpage
\section{Homotopiesph\"aren}
    Es ist nun naheliegend, die Menge \(\Theta_k\) mit der verbundenen Summe zu versehen. Es sind einige Lemmata notwendig, um zu zeigen, dass dies eine wohldefinierte Gruppenstruktur ergibt. 
\begin{lemma}\label{lem:simp_crit}
    Ein einfach zusammenh\"angender Kobordismus \(\mathcal{W}\) von \(\mathcal{M}\) zu \(\mathcal{N}\) ist genau dann ein H-Kobordismus, wenn \(H_*(\mathcal{W},\mathcal{M})=0\) gilt.
\end{lemma}
\begin{proof}
    Die Hinrichtung ist trivial. Seien nun \(H_*(\mathcal{W},\mathcal{M})=0\) und \(\iota\colon\mathcal{M}\hookrightarrow\mathcal{W}\) sowie \(j\colon\mathcal{N}\hookrightarrow\mathcal{W}\) die Inklusionen. Aus der Annahme folgt, dass  \(\iota\) ein Quasi-Isomorphismus ist. Dann folgt auch \(H^*(\mathcal{W},\mathcal{M})=0\) und aus der Lefschetz-Dualit\"at
    \[H_*(\mathcal{W},\mathcal{N})\cong H^{n+1-*}(\mathcal{W},\mathcal{M})=0\,,\]
    sodass auch \(j_*\) ein Quasi-Isomorphismus ist. Da \(\mathcal{W}\), \(\mathcal{M}\) und \(\mathcal{N}\) einfach zusammenh\"angend sind, folgt aus dem Satz von Whitehead, zusammen mit dem Satz von Hurewicz, dass die Inklusionen auch Homotopie\"aquivalenzen sind. 
\end{proof}
Das additive Inverse in \(\Theta_k\) ist offenbar die H-Kobordismus-Klasse der Standardsph\"are. Diese l\"asst sich wie folgt beschreiben.
\begin{lemma}\label{lem:h_cob_cont}
    Eine einfach zusammenh\"angende Mannigfaltigkeit \(\mathcal{M}^n\) ist genau dann Rand einer kontrahierbaren Mannigfaltigkeit \(\mathcal{W}^{n+1}\), wenn sie zu der Standardsph\"are H-kobordant ist.
\end{lemma}
\begin{proof}
    \subsubsection*{Hinrichtung}
        Sei \(\mathcal{M}\) Rand der kontrahierbaren Mannigfaltigkeit \(\mathcal{W}\). Sei \(\mathcal{D}\) eine eingebettete Scheibe und \({\mathcal{W}^{\prime}:=\mathcal{W}\setminus\mathring{\mathcal{D}}}\). Dann ist \(\partial\mathcal{W}^{\prime}\cong\mathcal{M}\sqcup-\mathcal{S}\) und \(\mathcal{W}^{\prime}\) ein Kobordismus. Per Ausschneidung folgt
        \[H_*(\mathcal{W}^{\prime},\partial\mathcal{D})\overcong{Aus.}H(\mathcal{W},\mathcal{D})\cong\Tilde{H}(\mathcal{W})=0\,,\]
        da \(\mathcal{W}\) einfach zusammenh\"angend ist, folgt aus Lemma \ref{lem:simp_crit}, dass \(\mathcal{W}^{\prime}\) ein H-Kobordismus ist.
        
    \subsubsection*{R\"uckrichtung}
        Sei \(\mathcal{M}\) mithilfe des Kobordismus \(\mathcal{W}^{\prime}\) zu der \(n\)-Sph\"are H-kobordant. Setze
        \[\mathcal{W}:=\mathcal{W}^{\prime}\mathop{+}^{\mathbb{S}^n}\mathbb{D}^{n+1}\,.\]
        Per Annahme gilt \(H_*(\mathcal{W}^{\prime},\mathbb{S}^n)=0\), sodass erneut
        \[H(\mathcal{W},\mathcal{D})\overcong{Aus.}H_*(\mathcal{W}^{\prime},\partial\mathcal{D})=0\]
        folgt. Da \(\mathcal{W}\) einfach zusammenh\"angend ist, ist sie zu dem kontrahierbaren Raum \(\mathcal{D}\) homotopie\"aquivalent und somit selbst kontrahierbar.
\end{proof}
\newpage
\begin{lemma}
    Das additive Inverse einer Homotopiesph\"are \(\mathcal{M}\) ist \(-\mathcal{M}\).
\end{lemma}
\begin{proof}
    W\"ahle eine Einbettung \(\Psi\colon\mathbb{R}^n+\mathbb{R}^n\hookrightarrow\mathbb{I}\times\mathbb{R}^n\) wie in Abbildung \ref{fig:sum_of_r}. F\"ur Details siehe \cite{kosinski1992differential} Kapitel VI Satz 1.3. Zu einer Verklebeabbildung \({\Phi\colon\mathbb{R}^n\hookrightarrow\mathcal{M}}\) ergibt dies eine Einbettung
    \[\im\Phi+\im\Phi\hookrightarrow\mathbb{I}\times\im\Phi\subseteq\mathbb{I}\times\mathcal{M}\,.\]
    Wird diese Einbettung \"uber \(\mathcal{M}\setminus\im\Phi\hookrightarrow0\times\mathcal{M}\) und \(-\mathcal{M}\setminus\im\Phi\hookrightarrow1\times\mathcal{M}\) zu einer Einbettung \(\mathcal{M}+(-\mathcal{M})\hookrightarrow\mathcal{M}\times\mathbb{I}\) fortgesetzt, ergibt sich eine Situation analog zu Abbildung \ref{fig:sum_of_r}. Das Bild dieser Einbettung, also auch \(\mathcal{M}+(-\mathcal{M})\), berandet eine Mannigfaltigkeit, die zu \(\mathcal{M}\), aus der eine offene Scheibe entfernt wurde, homotopie\"aquivalent ist. Diese muss jedoch kontrahierbar sein, da \(\mathcal{M}\) eine Homotopiesph\"are ist. Die Aussage folgt aus Lemma \ref{lem:h_cob_cont}.
\end{proof}

\begin{figure}
    \centering
    \begin{tikzpicture}
        \draw [pattern = north west lines]
            (-2.5, 1) -- node [above] {\(\mathbb{R}+\mathbb{R}\)} (-1.5, 1) arc (90:-90:1) -- (-2.5, -1);
        \draw [pattern = north west lines]
            (2.5, 1) -- (1.5, 1) arc (90:270:1) -- (2.5, -1)
            (-2.5, 0) edge [blue] (-0.5, 0) 
            (0.5, 0) edge [blue] (2.5, 0)
            (-3, 0) node {\(\mathbb{R}\setminus\mathring{\mathbb{D}}\)}
        ;
    \end{tikzpicture}
    \caption{Eine Einbettung \(\mathbb{R}+\mathbb{R}\hookrightarrow\mathbb{I}\times\mathbb{R}\). Ihr Bild berandet die schraffierte Mannigfaltigkeit, die die blaue Mannigfaltigkeit als Deformationsretrakt enth\"alt.}\label{fig:sum_of_r}
\end{figure}
\begin{lemma}
    Sind \(\mathcal{M}^n\) und \(\mathcal{M}^{\prime}\) geschlossen, einfach zusammenh\"angend und H-kobordant, sind es auch \(\mathcal{M}+\mathcal{N}\) und \(\mathcal{M}^{\prime}+\mathcal{N}\).
\end{lemma}
\begin{proof}
    Es kann angenommen werden, dass \(n\geq3\) ist. Sei \(\mathcal{W}_1\) ein H-Ko\-bor\-dis\-mus mit \(\partial\mathcal{W}_1=-\mathcal{M}^{\prime}\sqcup\mathcal{M}\). Da \(\mathcal{W}_1\) zu\-sam\-men\-h\"ang\-end ist, existiert eine Einbettung \(\iota\colon(\mathbb{I},\partial\,\mathbb{I})\hookrightarrow(\mathcal{W}_1,\partial\mathcal{W}_1)\) einer ordentlichen Untermannigfaltigkeit \(\mathcal{I}\), die Punkte \(p\in\mathcal{M}\) und \(p^{\prime}\in\mathcal{M}^{\prime}\) verbindet. Sei weiter \(\mathcal{W}_2:=\mathcal{N}\times\mathbb{I}\). Definiere den Kobordismus
    \[\mathcal{W}:=\mathcal{W}_1\mathop{+}^{\mathcal{I}}\mathcal{W}_2\quad\text{mit}\quad\partial\mathcal{W}\cong-\left(\mathcal{M}^{\prime}+\mathcal{N}\right)\sqcup\left(\mathcal{M}+\mathcal{N}\right)\,.\]
    Es bleibt zu zeigen, dass die Inklusionen \(\mathcal{M}+\mathcal{N}\hookrightarrow\mathcal{W}\) und \(\mathcal{M}^{\prime}+\mathcal{N}\hookrightarrow\mathcal{W}\) Homotopie\"aquivalenzen sind. Betrachte zun\"achst die Inklusion \(\iota\colon\mathcal{M}\hookrightarrow\mathcal{W}_1\). Da \(\mathcal{I}\cong\mathbb{I}\) kontrahierbar ist, ist das Normalenb\"undel in \(\mathcal{W}_1\) trivial, sodass eine Umgebung von \(\mathcal{I}\) zu \(\mathbb{I}\times\mathbb{R}^n\) diffeomorph ist. Das kommutative Diagramm
    \begin{equation}
        \begin{aligned}\label{eq:lem_sumHCob_01}
            \begin{tikzpicture}
                \draw
                    (0, 0) node (A) {\(H_*(\mathcal{M}\setminus p)\)}
                    (4, 0) node (B) {\(H_*(\mathbb{R}^n,\mathbb{R}^n\setminus\mathbf{0})\)}
                    (0, -1.5) node (D) {\(H_*(\mathcal{W}_1\setminus\mathcal{I})\)}
                    (4, -1.5) node (E) {\(H_*(\mathbb{I}\times\mathbb{R}^n,\mathbb{I}\times(\mathbb{R}^n\setminus\mathbf{0}))\)}

                    (A) edge [-stealth] node [above] {\(\cong\)} (B)
                    (A) edge [-stealth] node [left] {\(\iota_*\)} (D)
                    (B) edge [-stealth] node [right] {\((0\times\mathbbm{1})_*\)} node [left] {\(\cong\)} (E)
                    (D) edge [-stealth] node [above] {\(\cong\)} (E)
                    ;
            \end{tikzpicture}
        \end{aligned}
    \end{equation}
    zeigt, dass die linke Abbildung ein Isomorphismus ist. Dann folgt aus dem F\"unfer\-lemma, dass auch die rechte Abbildung folgenden Diagrammes ein Isomorphismus sein muss.
    \begin{equation}
        \begin{aligned}\label{eq:lem_sumHCob_02}
            \begin{tikzpicture}
            \draw
                (0, 0) node (A) {\(H_*(\mathcal{M}\setminus p)\)}
                (3, 0) node (B) {\(H_*(\mathcal{M})\)}
                (6, 0) node (C) {\(H_*(\mathcal{M},\mathcal{M}\setminus p)\)}
                (0, -1.5) node (D) {\(H_*(\mathcal{W}_1\setminus\mathcal{I})\)}
                (3, -1.5) node (E) {\(H_*(\mathcal{W}_1)\)}
                (6, -1.5) node (F) {\(H_*(\mathcal{W}_1,\mathcal{W}_1\setminus\mathcal{I})\)}

                (A) edge [-stealth] (B)
                (A) edge [-stealth] node [left] {\(\cong\)} node [right] {\tiny\eqref{eq:lem_sumHCob_01}} (D)
                (B) edge [-stealth] (C)
                (B) edge [-stealth] node [left] {\(\cong\)} node [right] {\tiny H.Kob.} (E)
                (C) edge [-stealth] node [left] {\(\cong\)} node [right] {\tiny Fünferlemma} (F)
                (D) edge [-stealth] (E)
                (E) edge [-stealth] (F)
                ;
            \end{tikzpicture}
        \end{aligned}
    \end{equation}
    Betrachte nun die relative Mayer-Vietoris-Folge der Zerlegung von \((\mathcal{W},\mathcal{M}+\mathcal{N})\) durch die Bilder von \((\mathcal{W}_1\setminus\mathcal{I},\mathcal{M}\setminus p)\) und \((\mathcal{W}_2\setminus\mathbb{I},\mathcal{N}\setminus q)\). Es gilt
    \[H_*\left(\mathcal{W}_1\setminus\mathcal{I}\cap\mathcal{W}_2\setminus\mathbb{I},\mathcal{M}\setminus p\cap\mathcal{N}\setminus q\right)\cong H_*\left(\mathbb{I}\times(\mathbb{R}^n\setminus\mathbf{0}),\mathbb{R}^n\setminus\mathbf{0}\right)=0\,.\]
    Aus der Exaktheit der Mayer-Vietoris-Folge
    \[0\overeq{\eqref{eq:lem_sumHCob_02}}H_*(\mathcal{W}_1\setminus\mathcal{I},\mathcal{M}\setminus p)\oplus H_*(\mathcal{W}_2\setminus\mathbb{I},\mathcal{N}\setminus p)\to H_*(\mathcal{W},\mathcal{M}+\mathcal{N})\to0\,,\]
    folgt nun, dass auch \(H_*(\mathcal{W},\mathcal{M}+\mathcal{N})=0\) ist. Da \(\mathcal{W}\) f\"ur \(n\geq3\) weiterhin einfach zusammenh\"angend ist, folgt die Aussage aus Lemma \ref{lem:simp_crit}.
\end{proof}

\begin{lemma}
    Die verbundene Summe zweier Homotopiesph\"aren ist eine Homotopiesph\"are.
\end{lemma}
\begin{proof}
    F\"ur \(0<i<n\) gilt \(H_i(\mathcal{M}+\mathcal{N})\cong H_i(\mathcal{M})\oplus H_i(\mathcal{N})=0\), sodass \(\mathcal{M}+\mathcal{N}\) wegen
    \[H_0(\mathcal{M}+\mathcal{N})\cong H_n(\mathcal{M}+\mathcal{N})\cong\mathbb{Z}\quad\text{und}\quad H_j(\mathcal{M}+\mathcal{N})=0\quad\text{f\"ur}\quad j>n\]
    die gleichen Homologiegruppen wie die Sph\"are besitzt. Aus dem Satz von Seifert-van-Kampen folgt
    \[\pi_1(\mathcal{M}+\mathcal{N})\cong\pi_1(\mathcal{M})\times\pi_1(\mathcal{N})=0\]
    sodas \(\mathcal{M}+\mathcal{N}\) einfach zusammenh\"angend, und wegen dem Satz von Whitehead auch \((n-1)\)-zusammenh\"angend ist. Aus dem Satz von Hurewicz folgt, dass ein
    \[\eqcl{f}\in\pi_n(\mathcal{M}+\mathcal{N})\cong H_n(\mathcal{M}+\mathcal{N})\cong\mathbb{Z}\,,\]
    also eine Abbildung \(f\colon\mathbb{S}^n\to\mathcal{M}+\mathcal{N}\) des Grades eins existiert. Diese induziert in allen Dimensionen Isomorphismen \(f_*\colon H_i(\mathbb{S}^n)\to H_i(\mathcal{M}+\mathcal{N})\). Erneut folgt aus dem Satz von Whitehead, dass \(f\) eine Homotopie\"aquivalenz ist.
\end{proof}


\section{Die Thom-Pontrjagin-Konstruktion}
    Zwei geschlossene, normal gerahmte Untermannigfaltigkeiten \(\mathcal{M}^n\) und \(\mathcal{V}^n\) einer geschlossenen Mannigfaltigkeit \(\mathcal{N}^{n+k}\) hei\ss en gerahmt kobordant, falls ein Kobordismus \(\mathcal{W}^{n+1}\subseteq\mathcal{N}\times\mathbb{I}\) mit einer Rahmung existiere, die mit den Rahmungen von \(\mathcal{M}\) und \(\mathcal{V}\) \"ubereinstimmt. Die Menge der \"Aquivalenzklassen zusammen mit der disunkten Vereinigung bildet f\"ur \(k\geq n+2\) eine abelsche Gruppe \cite{kosinski1992differential} Kapitel IX Satz 3.1. Das Nullelement ist hierbei trivialerweise die \"Aquivalenzklasse der leeren Mannigfaltigkeit. Ein Repr\"asentant dessen ist durch die Standardsph\"are mit der \textit{Standard-Rahmung} gegeben. Diese ist durch kanonische Identifizierungen des nach au\ss en  gerichtete Normalenvektors \(n(p)\in N_p\mathbb{S}^n\subseteq\mathbb{R}^{n+k}\) zusammen mit den Vektoren \(e_{n+2},\dots,e_{n+k}\) gegeben. Inverse Elemente sind \"Aquivalenzklassen der gleichen Mannigfaltigkeit, in welchem ein Basisvektor \(v\) mit \(-v\) ersetzt wurde. Die derartig erhaltene Gruppe sei durch \(\Omega_n^{\text{\tiny Fr}}(\mathcal{N})\) bezeichnet.
\begin{figure}
    \centering
    \tdplotsetmaincoords{60}{110}
    \tdplotsetthetaplanecoords{0}
    
    \begin{tikzpicture}[scale = 2.5, tdplot_main_coords]
        \draw[dashed,tdplot_rotated_coords] (0,-1,0) arc (-90:90:1);
        \draw[tdplot_rotated_coords]
        \foreach \i in {-60, -30,..., 60} {
            (\i:1) ++(\i:0.25) edge[stealth-] (\i:1)
        };
        \draw
        \foreach \i in {0,45, ..., 360} {
            (\i:1) ++(\i:0.25) edge[stealth-] (\i:1)
        }
        \foreach \i in {45, 135} {
            (\i:1cm and 1cm) ++(\i:0.25cm and 0.25cm) edge[stealth-] (\i:1cm and 1cm)
        };
        \draw [dashed] (180:1cm and 5mm) arc (180:0:1cm and 1cm) (1, 0, 0) arc (0:360:1);
        \shade[ball color=blue!10!white,opacity=0.2] (1cm, 0) arc (0:-180:1cm and 5mm) arc (180:0:1cm and 1cm);
    \end{tikzpicture}
    \caption{Die Standardrahmung von \(\mathbb{S}^1\subset\mathbb{R}^2\), und der gerahmte Nullbordismus \(\mathbb{S}_+^2\subset\mathbb{R}^3\).}
\end{figure}

\subsection{Thom-Pontrjagin-Kollapsabbildungen}
    Nach dem Satz von Sard besitzt jede Funktion \(f\colon\mathcal{N}^{n+k}\to\mathbb{S}^k\) einen regul\"aren Wert \(\star\in\mathbb{S}^k\), wobei das Urbild eines regul\"aren Wertes stets eine \(n\)-dimensionale Untermannigfaltigkeit \(f^{-1}(\star)\subseteq\mathcal{N}\) ergibt. Da \(\star\) eine nulldimensionale Mannigfaltigkeit ist, besitzt diese ein triviales Normalenb\"undel. Eine Rahmung dieser (also eine Wahl einer positiv orientierten Basis \(\phi\) von \(T_{\star}\mathbb{S}^k\)) l\"asst sich zu einer normalen Rahmung von \(f^{-1}(\star)\) zur\"uckziehen. Unterschiedliche regul\"are Punkte oder unterschiedliche positiv orientierte Basen ergeben hierbei zueinander gerahmt kobordante Mannigfaltigkeiten. Folglich l\"asst sich die Abbildung
    \[\Tilde{p}\colon\mathcal{C}^{\infty}(\mathcal{N},\mathbb{S}^k)\to\Omega_n^{\text{\tiny Fr}}(\mathcal{N}),\,f\mapsto\eqcl{\left(f^{-1}(\star),(\dx f)^*(\phi)\right)}\]
    definieren. Eine glatte Homotopie \(H\colon\mathcal{N}\times\mathbb{I}\to\mathbb{S}^k\) mit regul\"arem Punkt \(\star\) zwischen zwei Abbildungen \(f,g\in\mathcal{C}^{\infty}(\mathcal{N},\mathbb{S}^k)\) birgt nun den gerahmten Kobordismus \(H^{-1}(\star)\subseteq\mathcal{N}\times\mathbb{I}\) zwischen \(f^{-1}(\star)\) und \(g^{-1}(\star)\). Auf diese Art und Weise induziert \(\tilde{p}\) eine Abbildung \(p\colon\eqcl{\mathcal{N},\mathbb{S}^k}\to\Omega_n^{\text{\tiny Fr}}(\mathcal{N})\). Umgekehrt l\"asst sich einer Mannigfaltigkeit \(\mathcal{M}^n\subseteq\mathcal{N}^{n+k}\) mit einer  Trivialisierung \(\Phi\colon N\mathcal{M}\to\mathcal{M}\times\mathbb{R}^k\) folgenderma\ss en eine Funktion \(f\colon\mathcal{N}\to\mathbb{S}^k\) mit \(\mathcal{M}=f^{-1}(\star)\) zuordnen. Sei \(\Psi\colon N\mathcal{M}\hookrightarrow\mathcal{N}\) eine Tubenumgebung mit Bild \(U\subseteq\mathcal{N}\). Dann kann gem\"a\ss{} Diagramm \ref{dia:gen_map} eine glatte Abbildung \(f\colon U\to\mathbb{S}^k\setminus\{-\star\}\) konstruiert werden, die durch
    \[\Tilde{f}\colon\mathcal{N}\to\mathbb{S}^k,\,x\mapsto\begin{cases}
        f(x) & x\in U\\
        -\star & \text{sonst}
    \end{cases}\]
    auf \(\mathcal{N}\) fortgesetzt werden kann, \(\Tilde{f}^{-1}(\star)=\mathcal{M}\) erf\"ullt, und die Rahmung \(\Phi\) induziert. Bei der Konstruktion sei beachtet, dass sich unterschiedliche Basen von \(T_{\star}\mathbb{S}^n\) bez\"uglich einer Funktion zwar zu \"aquivalenten Rahmungen zur\"uckziehen, dass eine Mannigfaltigkeit zu unterschiedlichen Rahmungen jedoch sehr wohl in unterschiedlichen gerahmten Kobordanzklassen liegen kann. Von besonderem Interesse ist der Fall \({\mathcal{N}=\mathbb{S}^{n+k}}\), da \(\eqcl{\mathbb{S}^{n+k},\mathbb{S}^k}=\pi_{n+k}(\mathbb{S}^k)\) ist. Siehe f\"ur weitere Informationen \cite{milnor1965topology} \S7.

    \begin{figure}
        \centering
        \begin{tikzpicture}
            \draw   node (A) {\(U\subseteq\mathcal{N}\)}
                    node [above = of A] (B) {\(N\mathcal{M}\)}
                    node [right = of B] (C) {\(\underline{\mathbb{R}}^k\)}
                    node [right = of C] (D) {\(\mathbb{R}^k\)}
                    node [below = of D] (E) {\(\mathbb{S}^k\setminus\{-\star\}\)}
                    (B) edge [bend right = 35, ->] node [sloped, above, rotate = 180] {\(\cong\)} node [left] {\(\Psi\)} (A)
                    (B) edge [bend left = 35, ->] node [below] {\(\cong\)} node [above] {\(\Phi\)} (C)
                    (A) edge [bend right = 35, ->] node [sloped, above] {\(\cong\)} (C)
                    (C) edge [bend left = 35, ->] node [above] {\(\pi_2\)} (D)
                    (D) edge [bend left = 35, ->] node [sloped, above] {\(\cong\)} (E)
                    (A) edge [bend right = 25, ->] node [sloped, above] {\(f\)} (E);
        \end{tikzpicture}
        \caption{Die Konstruktion einer \(\mathcal{M}\) generierenden Abbildung \(f\).}
        \label{dia:gen_map}
    \end{figure}

    \begin{proposition}[Thom-Pontrjagin]
        Die Gruppen \(\Omega_n^{\textup{\tiny Fr}}(\mathbb{S}^{n+k})\) und \(\pi_{n+k}(\mathbb{S}^{k})\) sind isomorph.
    \end{proposition}
    \noindent Insbesondere wird die Folge der \(\pi_{n+k}(\mathbb{S}^k)\), also auch die Folge der \(\Omega_n^{\text{\tiny Fr}}(\mathbb{S}^{n+k})\) f\"ur \(k\geq n+2\) station\"ar. Setze
    \[\Omega_n^{\text{\tiny Fr}}:=\lim_{k\to\infty}\Omega_n^{\textup{\tiny Fr}}(\mathbb{S}^{n+k})\cong\lim_{k\to\infty}\pi_{n+k}(\mathbb{S}^k)=\Pi_n\,.\]
    Es stellt sich in der obigen Definition die Frage, warum die disjunkte Summe anstatt der verbundenen Summe als Gruppenoperation gew\"ahlt wird. Die Antwort darauf liegt in dem durch die verbundene Randsumme erhaltenen Kobordismus
    \[\mathcal{W}=\mathcal{M}\times\mathbb{I}+\mathcal{N}\times\mathbb{I}\quad\text{mit}\quad\partial\mathcal{W}\cong(-\mathcal{M})\sqcup(-\mathcal{N})\sqcup(\mathcal{M}+\mathcal{N})\,.\]
    Gegebene Rahmungen \(F\) und \(G\) von \(\mathcal{M}\) und \(\mathcal{N}\) induzieren eine Rahmung auf \(\mathcal{W}\) und damit auf \(\mathcal{M}+\mathcal{N}\). Folglich sind \({(\mathcal{M},F)\sqcup(\mathcal{N},G)}\) und \({(\mathcal{M}+\mathcal{N},H)}\) gerahmt kobordant, sodass die disjunkte Summe und die verbundene Summe tats\"achlich die gleiche Gruppenoperation bestimmen.


\section{Der \texorpdfstring{\(J\)}{TEXT}-Homomorphismus}
    Sei \(\gamma\colon\mathbb{S}^n\to\operatorname{SO}(k)\) glatt und \((\mathbb{S}^n,E)\) die Standardsph\"are mit der Standardrahmung des Normalenb\"undels im \(\mathbb{R}^{n+k}\). Dann existiert genau eine normale Rahmung \(\Gamma\) von \(\mathbb{S}^n\), sodass der Rahmenwechsel von \(E\) zu \(\Gamma\) gerade \(\gamma\) ist. F\"ur \(\Gamma\) kann also
\[\Gamma_i(p)=\sum_{j=1}^k\gamma_{ij}(p)E_j(p)\]
als Definition genutzt werden. Dies liefert die normal gerahmte Mannigfaltigkeit \((\mathbb{S}^n,\Gamma)\in\Omega_n^{\text{\tiny Fr}}(\mathbb{S}^{n+k})\) und (auch wenn noch nicht klar ist dass diese wohldefiniert ist) eine Abbildung
\[J_n^k\colon\pi_n(\operatorname{SO}(k))\to\Omega_n^{\text{\tiny Fr}}(\mathbb{S}^{n+k})\,.\]
Sei nun \(\eqcl{\gamma}\in\ker J_n^k\), es gelte also \(J_n^k\eqcl{\gamma}=0\). Dann ist \((\mathbb{S}^n,\Gamma)\) gerahmt nullbordant. Das ist genau dann der Fall, wenn sich \(\Gamma\) auf eine Mannigfaltigkeit \(\mathcal{M}^{n+1}\subseteq\mathbb{R}_+^{n+k+1}\) mit \(\partial\mathcal{M}\cong\mathbb{S}^n\) fortsetzen l\"asst. F\"ur hinreichend gro\ss es \(k\), also f\"ur \(k\geq n+2\), stabilisiert sich diese Definition, und es ergibt sich der \textbf{Hopf-Whitehead-J-Homomorphismus}
\[J_n\colon\pi_n(\operatorname{SO})\to\Pi_n\,.\]
\begin{proposition}\label{prop:adams_order}
    Das Bild \({\im J_{4m-1}}\) ist eine zyklische Gruppe des Ranges
    \[\operatorname{Rang}\operatorname{Im}J_{4m-1}=\operatorname{Nenner}\left(\frac{B_{2m}}{4m}\right)\]
    und ein direkter Summand von \(\Pi_{4m-1}\).
\end{proposition}
\begin{proof}
    Siehe \cite{adams1966groups} Satz 1.5.
\end{proof}

    \chapter{Chirurgie}
        Sei \(\mathcal{M}^n\) eine Mannigfaltigkeit und \({n=i+j+1}\). Chirurgie ist der Prozess, \(\mathcal{M}\) mit der Standardsph\"are \(\mathbb{S}^n\) entlang einer in das Innere von \(\mathcal{M}\) eingebetteten Sph\"are mit trivialem Normalenb\"undel \({\mathbb{S}^i\hookrightarrow\mathring{\mathcal{M}}}\) zu verkleben. Dieser Prozess l\"asst sich auch derart beschreiben, dass eine eingebettete Untermannigfaltigkeit \({\mathbb{S}^i\times\mathbb{D}^{j+1}}\) herausgeschnitten, und die resultierende Mannigfaltigkeit entlang \({\mathbb{S}^i\times\mathbb{S}^j}\) mit \({\mathbb{D}^{i+1}\times\mathbb{S}^j}\) verklebt wird. Dies f\"uhrt zu einer kontrollierten Ver\"anderung der Homologie.

\section{Definitionen}
    Sei \(\mathcal{S}^i\hookrightarrow\mathring{\mathcal{M}}\) eine eingebettete Sph\"are mit trivialem Normalenb\"undel. Die letzte Forderung bedeutet hierbei, dass \({N\mathcal{S}\cong\underline{\mathbb{R}}^{j+1}}\) gilt, sodass eine Tubenumgebung von \(\mathcal{S}\) zu \(\underline{\mathbb{R}}^{j+1}\) isomorph ist. Als Anklebeabbildung sei also eine Einbettung \(\Phi\colon\underline{\mathbb{R}}^{j+1}\hookrightarrow\mathring{\mathcal{M}}\) gew\"ahlt. Da das Normalenb\"undel der eingebetteten Sph\"are \({\mathbb{S}^i\times\{0\}^j\subset\mathbb{S}^n}\) ebenso trivial ist, lassen sich \(\mathcal{M}\) und \(\mathbb{S}^n\) entlang \(\mathcal{S}\cong\mathbb{S}^i\) verkleben. Die entstehende Mannigfaltigkeit gehe durch \textbf{Chirurgie} an \(\mathcal{S}\) aus \(\mathcal{M}\) hervor. Die g\"angigste Bezeichnung ist hierbei wohl \(\chi(\mathcal{M},\Phi)\). Im Folgenden sei stattdessen
\[\mathcal{M}\multimap\mathbb{S}^i:=\mathcal{M}\mathop{+}^{\Phi}\mathbb{S}^n=\chi(\mathcal{M},\Phi)\]
eben jene Chirurgie.
\section{Verbindung zu Henkeln}
    Ein nah mit Chirurgie verbundener Vorgang ist das Anbringen eines Henkels. Sei hierzu \(\mathcal{W}^{n+1}\) eine Mannigfaltigkeit und \(\mathcal{S}^i\hookrightarrow\partial\mathcal{W}\) eine eingebettete Sph\"are mit trivialem Normalenb\"undel in \(\mathcal{W}\). Erneut ist \(N\mathcal{S}\cong\underline{\mathbb{R}}^{n+1}\) (nicht \(\underline{\mathbb{R}}^n\times\mathbb{R}_+\)). Eine Tubenumgebung ist also eine Einbettung \(\tilde{\Phi}\colon\underline{\mathbb{R}}^n\times\mathbb{R}_+\hookrightarrow\mathcal{W}\), die eine Tubenumgebung \(\Phi\colon\underline{\mathbb{R}}^n\hookrightarrow\partial\mathcal{W}\) fortsetzt. Analoges gilt f\"ur \(\mathbb{S}^i\subset\partial\mathbb{D}^{n+1}\). Die Verklebung 
    \[\mathcal{W}\multimap\mathbb{D}^{i+1}:=\mathcal{W}\mathop{+}^{\tilde{\Phi}}\mathbb{D}^{n+1}\,,\]
    gehe durch das \textbf{Anbringen eines Henkels} an \(\mathcal{S}\) aus \(\mathcal{W}\) hervor. Dieser Vorgang geschieht immer am Rand einer Mannigfaltigkeit. Naheliegenderweise gilt
    \[\partial(\mathcal{W}\multimap\mathbb{D}^{i+1})=\partial(\mathcal{W}\mathop{+}^{\tilde{\Phi}}\mathbb{D}^{n+1})\cong\partial\mathcal{W}\mathop{+}^{\Phi}\partial\mathbb{D}^{n+1}=\partial\mathcal{W}\multimap\mathbb{S}^i\,,\]
    der Rand der Anbringung eines Henkels an \(\mathcal{W}\) ist also die Chirurgie an \(\partial\mathcal{W}\). Diese Konstruktion sei in Abbildung \ref{fig:surgery} verdeutlicht. Dies vereinfacht die Sitation f\"ur geschlossene Mannigfaltigkeiten \(\mathcal{M}\), da in diesem Fall \(\mathcal{M}\times\mathbb{I}\) eine \((n+1)\)-Mannigfaltikeit mit \(\partial\mathcal{W}\cong\mathcal{M}\sqcup(-\mathcal{M})\) ist. Es gilt also auch
    \[\partial(\mathcal{M}\times\mathbb{I}\multimap\mathbb{D}^{i+1})\cong\mathcal{M}\sqcup-(\mathcal{M}\multimap\mathbb{S}^i)\]
    sodass \(\mathcal{M}\multimap\mathbb{S}^i\) und \(\mathcal{M}\) kobordant sind. Der Trick die Chirurgie derart als Rand darzustellen vereinfacht einige \"Uber\-le\-gun\-gen. Es sei jedoch Obacht geboten, da \(\mathcal{M}\times\mathbb{I}\) f\"ur glatte Mannigfaltigkeiten \(\mathcal{M}\) mit nicht leerem Rand keine glatte Mannigfaltigkeit ergibt. Entlang der Strata \(\partial\mathcal{M}\times\partial\,\mathbb{I}\) l\"asst sich keine glatte Struktur angeben, die mit der glatten Struktur von \(\mathcal{M}\) \"ubereinstimmt. Da im Folgenden ledliglich Mannigfaltigkeiten \(\mathcal{M}\) betrachtet werden, deren Rand leer oder eine Homotopiesph\"are ist, kann dem folgenderma\ss en Abhilfe geschaffen werden.
    \begin{figure}
    \centering
    \begin{tikzpicture}[scale = 0.6]

        \path (-50:3) arc (-50:230:3)
        \foreach\i in {0, 0.05, 0.1, 0.15, 0.2, 0.25, 0.3, 0.35, 0.4, 0.45, 0.5, 0.55, 0.6, 0.65, 0.7, 0.75, 0.8, 0.85, 0.9, 0.95, 1} { node [pos = \i] (A\i) {}};

        % Right tube backframe
        \foreach\i in {0, 0.05, 0.1, 0.15, 0.2, 0.25, 0.3, 0.35, 0.4, 0.45} { 
            \draw [thick, densely dotted, rotate = -50 + \i * 280] ({A\i}.center) ++(180:1 and 0.5) arc (180:360:1 and 0.3 - 0.875 * \i);
        }
        % Right core
        \draw [thick, orange]
            (-50:3) node {\tiny\textbullet} arc (-50:90:3);

        % Small ball
        \begin{scope}[yshift = 3cm] 
            % Cocore
            \draw [fill, color = darkgreen, opacity = 0.5] (0:0.1375 and 1) arc (0:360:0.1375 and 1);
            % Transversal sphere
            \draw [thick, densely dotted, color = {rgb,255:red,0;green,100;blue,0}] 
                (90:0.1375 and 1) arc (90:270:0.1375 and 1);
            \draw
                (1, 0) arc (0:360:1)
                (180:1 and 0.5) arc (180:360:1 and 0.5);
            \draw [densely dotted]
                (0:1) arc (0:180:1 and 0.5);
        \end{scope}

        % Right tube topframe
        \foreach\i in {0, 0.05, 0.1, 0.15, 0.2, 0.25, 0.3, 0.35, 0.4} { 
            \draw [thick, rotate = -50 + \i * 280] ({A\i}.center) ++(0:1 and 0.5) arc (0:180:1 and 0.3 - 0.875 * \i);
        }

        \path [thick, rotate = -50, color = red] ({A0}.center) ++(0:1 and 0.5) arc (0:180:1 and 0.3) node [pos = 0.7] {\tiny\textbullet};

        % Right tube
        \draw [thick]
            (-50:4) arc (-50:90:4) 
            (-50:2) arc (-50:90:2);

        % Big ball
        \begin{scope}[yshift = -3.75cm, scale = 3]
            \draw [densely dotted]
                (90:0.5 and 1) arc (90:270:0.5 and 1) 
                (0:1 and 0.5) arc (0:180:1 and 0.5);
            \shade[ball color = blue!10!white, opacity = 0.5] (1, 0) arc (0:360:1) -- cycle;
            \draw [thick]
                (0:1) arc (0:360:1) 
                (-90:0.5 and 1) arc (-90:90:0.5 and 1) 
                (180:1 and 0.5) arc (180:360:1 and 0.5);
        \end{scope}

        % Left tube
        \draw [thick]
            (90:4) arc (90:230:4) 
            (90:2) arc (90:230:2);

        % Left tube backframe
        \foreach\i in {0.55, 0.6, 0.65, 0.7, 0.75, 0.8, 0.85, 0.9, 0.95} { 
            \draw [thick, densely dotted, rotate = -50 + \i * 280] ({A\i}.center) ++(0:1 and 0.5) arc (0:180:1 and 0.3 - 0.875 * \i);
        }
        % Meridian backframe
        \draw [thick, densely dotted, rotate = -50 + 280, color = blue] ({A1}.center) ++(0:1 and 0.5) arc (0:180:1 and 0.3 - 0.875) node [pos = 0.6, red] {\tiny\textbullet};
        % Left core
        \draw [thick, orange]
            (90:3) arc (90:230:3) node {\tiny\textbullet};
        
        % Small ball shading
        \shade[ball color = blue!10!white, opacity = 0.5, yshift = 3cm] (1, 0) arc (0:360:1) -- cycle;
        
        % Transversal sphere top frame
        \draw [thick, color = darkgreen, yshift = 3cm] 
            (-90:0.1375 and 1) arc (-90:90:0.1375 and 1);

        % Last right tube topframe 
        \draw [thick, rotate = -50 + 0.45 * 280] ({A0.45}.center) ++(0:1 and 0.5) arc (0:180:1 and 0.3 - 0.875 * 0.45);

        % Left tube topframe
        \foreach\i in {0.55, 0.6, 0.65, 0.7, 0.75, 0.8, 0.85, 0.9, 0.95} { 
            \draw [thick, rotate = -50 + \i * 280] ({A\i}.center) ++(180:1 and 0.5) arc (180:360:1 and 0.3 - 0.875 * \i);
        }
        % Meridian topframe
        \draw [thick, rotate = -50 + 280, color = blue] ({A1}.center) ++(180:1 and 0.5) arc (180:360:1 and 0.3 - 0.875);
        
    \end{tikzpicture}
    \caption{Eine Vollkugel mit einem Henkel. Farblich markiert sind \textcolor{orange}{Kern}, \textcolor{darkgreen}{Kokern}, \textcolor{blue}{Meridian} und \textcolor{red}{\"Aquator}.}\label{fig:surgery}
\end{figure}
    \newpage
    \begin{lemma}\label{lem:smooth_man}
        Zu jeder Mannigfaltigkeit \(\mathcal{M}^n\) existiert eine Mannigfaltigkeit
        \[\mathcal{W}^{n+1}\subseteq\mathcal{M}\times\mathbb{I}\quad\text{mit}\quad\partial\mathcal{W}\cong\mathcal{M}\mathop{+}^{\partial\mathcal{M}}\mathcal{M}\,.\]
    \end{lemma}
    \begin{proof}
        Eine Kragenumgebung von \(\partial\mathcal{M}\) in \(\mathcal{M}\) kann zu einer topologischen Einbettung \(\partial\mathcal{M}\times\mathbb{R}_+^2\hookrightarrow\mathcal{M}\times\mathbb{I}\) fortgesetzt werden. Dann kann \(\mathbb{R}_+^2\) durch eine geeignete glatte Mannigfaltigkeit mit Rand ersetzt werden. Eine M\"oglichkeit diese zu definieren w\"are als jenes Gebiet \(Q\subseteq\mathbb{R}_+^2\), welches durch die Kurve \(\gamma\) abgegrenzt wird, die durch
        \[\gamma(t):=\left(1+t^2\right)\left(\cos(g(t)),\sin(g(t))\right)\quad\text{mit}\quad g(t)=\frac{h\left(t+\frac{1}{2}\right)}{h\left(t+\frac{1}{2}\right)+h\left(\frac{1}{2}-t\right)}\]
        und
        \[h(t)=\begin{cases}
            e^{-\frac{1}{x}} & x\geq0\\
            0 & \text{sonst}
        \end{cases}\]
        definiert ist. Siehe hierzu Abbildung \ref{fig:smooth_man}. Bezeichne die resultierende Mannigfaltigkeit durch \(\mathcal{W}\). Der Rand dieser Gl\"attung ist gerade eine Gl\"attung von
        \[\partial(\mathcal{M}\times\mathbb{I})=\partial\mathcal{M}\times\mathbb{I}\cup\mathcal{M}\times\partial\,\mathbb{I}\,.\]
        Aus der Definition l\"asst sich erkennen, dass diese zu \(\mathcal{M}+_{\partial\mathcal{M}}\mathcal{M}\) diffeomorph ist.
    \end{proof}
    \begin{figure}
        \centering
        \includegraphics[scale = 2]{Kapitel/Chirurgie/geogebra-export.png}
        \caption{Eine glatte Kurve, sodass das oben rechts berandete Gebiet eine glatte Mannigfaltigkeit mit Rand ist, die zum \(\mathbb{R}_+^2\) hom\"oomorph ist.}\label{fig:smooth_man}
    \end{figure}
    Sei \(\mathcal{W}\) gem\"a\ss{} Lemma \ref{lem:smooth_man}. Wird ein Henkel an dem Rand von \(\partial\mathcal{W}\) angebracht, ergibt sich
    \[\partial\left(\mathcal{W}\multimap\mathbb{D}^{i+1}\right)\cong\mathcal{M}\mathop{+}^{\partial\mathcal{M}}\left(\mathcal{M}\multimap\mathbb{S}^i\right)\,.\]
    Wenn \(\partial\mathcal{M}\) zus\"atzlich leer oder eine Homotopiesph\"are ist, folgt f\"ur \(0<i<n\) 
    \[H_i(\partial\mathcal{W})\cong H_i(\mathcal{M})\oplus H_i(\mathcal{M})\,.\]
    Diese Mannigfaltigkeit verh\"alt sich deshalb \"ahnlich genug zu \(\mathcal{M}\times\mathbb{I}\).

\subsection{Kombinatorische Chirurgie}
    Besonders im Rahmen homologischer \"Uberlegungen ist die glatte Struktur der Mannigfaltigkeit nicht vonn\"oten, sodass in diesem Fall eine topologisch \"aqui\-va\-len\-te und einfachere Notation genutzt werden kann. Sei eine glatte Einbettung \(\Phi\colon\mathbb{S}^i\times\mathbb{D}^{j+1}\hookrightarrow\mathring{\mathcal{M}}\) mit \(\mathcal{D}:=\im\Phi\) gegeben. Setze \(\mathcal{M}_0:=\mathcal{M}\setminus\mathring{\mathcal{D}}\) und betrachte das topologische Pushout
    \[\mathcal{M}^{\prime}:=\mathcal{M}_0\cup_{\partial\mathcal{D}}\left(\mathbb{D}^{i+1}\times\mathbb{S}^j\right)\,.\]
    Ebenso l\"asst sich das Anbringen eines Henkels an \(\mathcal{W}^{n+1}\) durch
    \[\mathcal{W}+H^i:=\mathcal{W}\cup_{\mathcal{D}}\left(\mathbb{D}^{i+1}\times\mathbb{D}^{j+1}\right)\]
    definieren. Die entstehende Mannigfaltigkeit sind hierbei hom\"oomorph zu ihren oberen glatten Definitionen (\cite{kosinski1992differential} Kapitel VI Proposition 8.1). 
    
    \subsubsection{Meridian und \"Aquator einer Chirurgie}
        Diese Definition legt nahe, warum \(\mathbb{S}^i\times\mathbb{S}^j\) bei dem Untersuchen von einer Chirurgie eine wichtige Rolle spielt. F\"ur ein \(x\in\mathbb{S}^i\) und ein \(y\in\mathbb{S}^j\) sei \(\Phi\left(\{x\}\times\mathbb{S}^j\right)\) ein \textbf{Meridian} und \(\Phi\left(\mathbb{S}^i\times\{y\}\right)\) ein \textbf{\"Aquator}. 

    \subsubsection{Kern und Kokern eines Henkels}
        Der Anteil von \(\mathcal{W}\multimap\mathbb{D}^{i+1}\), der mit \(\mathbb{D}^{i+1}\times0\) korrespondiert hei\ss e \textbf{Kern} und der Anteil \(0\times\mathbb{D}^{j+1}\) hei\ss e \textbf{Kokern}. Der Rand des Kernes ist gerade die Anklebesph\"are, der Rand des Kokerns hei\ss e transversale Sph\"are. Diese Begriffe sind insbesondere im Rahmen des H-Kobordismus-Satzes wichtig.


\section{Chirurgie unterhalb mittlerer Dimension}
    Der Effekt einer Chirurgie auf die niederen Homologiegruppen ist recht simpel. Sei \(n=i+j+1\), \(\mathcal{M}^n\) eine Mannigfaltigkeit, \(\mathcal{S}^i\hookrightarrow\mathcal{M}\) eine eingebettete Sph\"are mit trivialem Normalenb\"undel und \(\Phi\colon\underline{\mathbb{D}}^{j+1}\hookrightarrow\mathcal{M}\) eine zugeh\"orige Anklebeeinbettung. Setze \(\mathcal{D}:=\im\Phi\) und
\[\mathcal{M}_0:=\mathcal{M}\setminus\mathring{\mathcal{D}}\quad\text{sowie}\quad\mathcal{M}^{\prime}:=\mathcal{M}_0\cup_{\partial\mathcal{D}}\left(\mathbb{D}^{i+1}\times\mathbb{S}^j\right)\,.\] 
Dann lassen sich die langen exakten Folgen der Paare \((\mathcal{M},\mathcal{M}_0)\) und \((\mathcal{M}^{\prime},\mathcal{M}_0)\) zu folgendem Diagramm zusammensetzen:
\begin{center}
    \begin{tikzpicture}
        \draw
            (-3, -1.5) node (B) {\(H_{k+1}(\mathcal{M}^{\prime},\mathcal{M}_0)\)}
            (0, 0) node (C) {\(H_k(\mathcal{M}_0)\)}
            (2.5, 1.5) node (D) {\(H_k(\mathcal{M}^{\prime})\)}
            (5, 1.5) node (E) {\(H_k(\mathcal{M}^{\prime},\mathcal{M}_0)\)}
            (-3, 1.5) node (G) {\(H_{k+1}(\mathcal{M},\mathcal{M}_0)\)}
            (2.5, -1.5) node (H) {\(H_k(\mathcal{M})\)}
            (5, -1.5) node (I) {\(H_k(\mathcal{M},\mathcal{M}_0)\)}
            
            (B.east) edge [bend right = 25, -stealth] (C)
            (C) edge [bend left = 25, -stealth] (D.west)
            (D) edge [-stealth] (E)

            (G.east) edge [bend left = 25, -stealth] (C)
            (C) edge [bend right = 25, -stealth] (H.west)
            (H) edge [-stealth] (I)
            ;
    \end{tikzpicture}
\end{center}
Per Ausschneidung folgen
\[H_k(\mathcal{M},\mathcal{M}_0)\cong H_k(\mathbb{S}^i\times\mathbb{D}^{j+1},\mathbb{S}^i\times\mathbb{S}^j)\cong\begin{cases}
    \mathbb{Z} & k\in\{0,j+1\}\\
    0 & 0<k<j+1
\end{cases}\]
und 
\[H_k(\mathcal{M}^{\prime},\mathcal{M}_0)\cong H_k(\mathbb{D}^{i+1}\times\mathbb{S}^j,\mathbb{S}^i\times\mathbb{S}^j)\cong\begin{cases}
    \mathbb{Z} & k\in\{0,i+1\}\\
    0 & 0<k<i+1
\end{cases}\,.\]
F\"ur die Berechnung siehe Appendix. Die Erzeuger der \(\mathbb{Z}\)-Anteile in Dimension \(i+1\) und \(j+1\) sind gerade die Bilder der Erzeuger von \(H_{i+1}(\mathbb{D}^{i+1},\mathbb{S}^i)\) und \(H_{j+1}(\mathbb{D}^{j+1},\mathbb{S}^j)\) und stehen somit in Korrespondenz zu Fundamentalklassen eines \"Aquators \(e=\Phi(x\times\mathbb{S}^j)\) und einem Meridian \(m=\Phi(\mathbb{S}^i\times y)\). \"Aquator und Meridian sind deswegen interessant, da sie beide in \(\mathcal{M}_0\) liegen, und die Beziehungen
\[\eqcl{e\mathrel{|}\mathcal{M}}=\eqcl{\mathcal{S}\mathrel{|}\mathcal{M}},\quad\eqcl{m\mathrel{|}\mathcal{M}}=0\quad\text{sowie}\quad\eqcl{e\mathrel{|}\mathcal{M}^{\prime}}=0\]
erf\"ullen. Die Sph\"are hei\ss e \textbf{primitiv}, wenn ein \(g\in H_j(\mathcal{M})\) mit \(g\cdot\eqcl{\mathcal{S}\mathrel{|}\mathcal{M}}=1\) existiert.
\begin{lemma}
    Chirurgie an einer Sph\"are \(\mathcal{S}^i\hookrightarrow\mathcal{M}^n\) mit \(i<j\) eliminiert die Fundamentalklasse \(\eqcl{\mathcal{S}\mathrel{|}\mathcal{M}}\).
\end{lemma}
\begin{proof}
    Einerseits folgt f\"ur \(1<k<i\) direkt
    \[H_k(\mathcal{M}^{\prime})\cong H_k(\mathcal{M}_0)\cong H_k(\mathcal{M})\,.\]
    Das Diagramm nimmt f\"ur \(k=i\) die folgende Form an:
    \begin{center}
        \begin{tikzpicture}
            \draw
                (-2, -1.5) node (B) {\(\mathbb{Z}\)}
                (0, 0) node (C) {\(H_i(\mathcal{M}_0)\)}
                (2, 1.5) node (D) {\(H_i(\mathcal{M}^{\prime})\)}
                (4, 1.5) node (E) {\(0\)}
                (4, -1.5) node (F) {\(0\)}
                (-2, 1.5) node (G) {\(0\)}
                (2, -1.5) node (H) {\(H_i(\mathcal{M})\)}

                (B) edge [bend right = 25, -stealth] node [below = 0.17, pos = 0.3] {\tiny\(1\)} node [sloped, below, pos = 0.56] {\tiny\(\mapsto\eqcl{e\mathrel{|}\mathcal{M}_0}\)} (C)
                (C) edge [bend left = 25, -stealth] (D)
                (D) edge [bend left = 10, -stealth] (E)

                (B) edge [bend right = 20, -stealth] node [sloped, above, pos = 0.41] {\tiny\(\mapsto\eqcl{e\mathrel{|}\mathcal{M}}\)} (H)

                (G) edge [bend left = 25, -stealth] (C)
                (C) edge [bend right = 25, -stealth] (H)
                (H) edge [bend right = 10, -stealth] (F)
                ;
        \end{tikzpicture}
    \end{center}
    Somit gilt 
    \[H_i(\mathcal{M}^{\prime})\cong H_i(\mathcal{M}_0)/\langle\eqcl{e\mathrel{|}\mathcal{M}_0}\rangle\cong H_i(\mathcal{M})/\langle\eqcl{e\mathrel{|}\mathcal{M}}\rangle\,.\]
\end{proof}
Auf \"ahnliche Art und Weise kann der Effekt einer Chirurgie auf die Homotopiegruppen untersucht werden.
\begin{lemma}\label{lem:fund_smaller}
    F\"ur \(k<i<j\) existiert ein Normalteiler \(N\triangleleft\pi_i(\mathcal{M})\) mit \(\eqcl{\mathcal{S}}\in N\) und
    \[\pi_k(\mathcal{M}^{\prime})\cong\pi_k(\mathcal{M})\quad\text{und}\quad\pi_i\left(\mathcal{M}^{\prime}\right)\cong\pi_i(\mathcal{M})/N\,.\]
\end{lemma}
\begin{lemma}\label{lem:odd_surg_effect}
    Chirurgie an einer \textbf{primitiven} Sph\"are \(\mathcal{S}^i\hookrightarrow\mathcal{M}^{2i+1}\) eliminiert die Fundamentalklasse \(\eqcl{\mathcal{S}\mathrel{|}\mathcal{M}}\).
\end{lemma}
\begin{proof}
    Es gilt \(i=j\). Das Diagramm nimmt die folgende Form an:
    \begin{center}
        \begin{tikzpicture}
            \draw
                (-4.75, 1.5) node (A) {\(H_{i+1}(\mathcal{M})\)}
                (-2, -1.5) node (B) {\(\mathbb{Z}\)}
                (0, 0) node (C) {\(H_i(\mathcal{M}_0)\)}
                (2, 1.5) node (D) {\(H_i(\mathcal{M}^{\prime})\)}
                (4, 1.5) node (E) {\(0\)}
                (4, -1.5) node (F) {\(0\)}
                (-2, 1.5) node (G) {\(\mathbb{Z}\)}
                (2, -1.5) node (H) {\(H_i(\mathcal{M})\)}

                (A) edge [-stealth, bend left = 5] node [above] {\tiny\(x\mapsto x\cdot\eqcl{\mathcal{S}\mathrel{|}\mathcal{M}}\)} (G)
                (B) edge [bend right = 25, -stealth] node [below = 0.17, pos = 0.29] {\tiny\(1\)} node [sloped, below, pos = 0.56] {\tiny\(\mapsto\eqcl{e\mathrel{|}\mathcal{M}_0}\)} (C)
                (C) edge [bend left = 25, -stealth] (D)
                (D) edge [bend left = 10, -stealth] (E)

                (B) edge [bend right = 20, -stealth] node [sloped, above, pos = 0.39] {\tiny\(\mapsto\eqcl{e\mathrel{|}\mathcal{M}}\)} (H)
                (G) edge [bend left = 20, -stealth] node [sloped, below, pos = 0.45] {\tiny\(\mapsto\eqcl{m\mathrel{|}\mathcal{M}^{\prime}}\)} (D)

                (G) edge [bend left = 25, -stealth] node [above = 0.19, pos = 0.3] {\tiny\(1\)} node [sloped, above, pos = 0.59] {\tiny\(\mapsto\eqcl{m\mathrel{|}\mathcal{M}_0}\)} (C)
                (C) edge [bend right = 25, -stealth] (H)
                (H) edge [bend right = 10, -stealth] (F)
                ;
        \end{tikzpicture}
    \end{center}
    Da die Sph\"are primitiv ist, ist \(H_{i+1}(\mathcal{M})\to\mathbb{Z}\) surjektiv, also \(H_i(\mathcal{M})\cong H_i(\mathcal{M}_0)\). Es folgt 
    \[H_i(\mathcal{M}^{\prime})\cong H_i(\mathcal{M}_0)/\langle\eqcl{e\mathrel{|}\mathcal{M}_0}\rangle\cong H_i(\mathcal{M})/\langle\eqcl{e\mathrel{|}\mathcal{M}}\rangle\,.\]
    F\"ur \(k<i\) hat die Chirurgie keinen Effekt.
\end{proof}
Allgemein l\"asst sich erkennen, dass in diesem Fall 
\[H_i(\mathcal{M}^{\prime})/\langle\eqcl{m\mathrel{|}\mathcal{M}^{\prime}}\rangle\cong H_i(\mathcal{M}_0)/\langle\eqcl{m\mathrel{|}\mathcal{M}_0},\eqcl{e\mathrel{|}\mathcal{M}_0}\rangle\cong H_i(\mathcal{M})/\langle\eqcl{e\mathrel{|}\mathcal{M}}\rangle\]
gilt.
\begin{lemma}\label{lem:even_surg_effect}
    Chirurgie an einer \textbf{primitiven} Sph\"are \(\mathcal{S}^i\hookrightarrow\mathcal{M}^{2i}\) eliminiert die Fundamentalklasse \(\eqcl{\mathcal{S}\mathrel{|}\mathcal{M}}\).
\end{lemma}
\begin{proof}
    Es gilt \(i=j+1\). F\"ur \(k<i-1\) hat die Chirurgie keinen Effekt. Der interessante Anteil des Diagramms ist von der Form:
    \begin{center}
        \begin{tikzpicture}
            \draw
                (-2, -1.5) node (B) {\(\mathbb{Z}\)}
                (0, 0) node (C) {\(H_i(\mathcal{M}_0)\)}
                (2, 1.5) node (D) {\(H_i(\mathcal{M}^{\prime})\)}
                (4, 1.5) node (E) {\(0\)}
                (-2, 1.5) node (G) {\(0\)}
                (2, -1.5) node (H) {\(H_i(\mathcal{M})\)}
                (4, -1.5) node (I) {\(\mathbb{Z}\)}
                (6, 0) node (J) {\(H_{i-1}(\mathcal{M}_0)\)}
                (8, 1.5) node (K) {\(H_{i-1}(\mathcal{M})\)}
                (8, -1.5) node (L) {\(H_{i-1}(\mathcal{M}^{\prime})\)}
                (9.5, -1.5) node (M) {\(0\)}
                (9.5, 1.5) node (N) {\(0\)}
                
                (B) edge [bend right = 30, -stealth] node [below = 0.17, pos = 0.31] {\tiny\(1\)} node [sloped, below, pos = 0.57] {\tiny\(\mapsto\eqcl{e\mathrel{|}\mathcal{M}_0}\)} (C)
                (B) edge [bend right = 20, -stealth] node [sloped, above, pos = 0.42] {\tiny\(\mapsto\eqcl{e\mathrel{|}\mathcal{M}}\)} (H)
                (C) edge [bend left = 30, -stealth] (D)
                (D) edge [-stealth] (E)
                (E) edge [bend left = 30, -stealth] (J)
                (J) edge [bend right = 30, -stealth] (L)
                (L) edge [-stealth] (M)

                (G) edge [bend left = 30, -stealth] (C)
                (C) edge [bend right = 30, -stealth] (H)
                (H) edge [-stealth] node [above] {\(\phi\)} (I)
                (I) edge [bend right = 30, -stealth] node [below = 0.17, pos = 0.29] {\tiny\(1\)} node [sloped, below, pos = 0.57] {\tiny\(\mapsto\eqcl{m\mathrel{|}\mathcal{M}_0}\)}  (J)
                (I) edge [bend right = 20, -stealth] node [sloped, above, pos = 0.47] {\tiny\(\mapsto\eqcl{m\mathrel{|}\mathcal{M}^{\prime}}\)}  (L)
                (J) edge [bend left = 20, -stealth] (K)
                (K) edge [-stealth] (N)
                ;
        \end{tikzpicture}
    \end{center}
    Wegen der Primitivit\"at ist \(\phi\colon H_i(\mathcal{M})\to\mathbb{Z},\,x\mapsto x\cdot\eqcl{\mathcal{S}\mathrel{|}\mathcal{M}}\) ein Epimorphismus, also folgt
    \[H_{i-1}(\mathcal{M}^{\prime})\cong H_{i-1}(\mathcal{M}_0)\cong H_{i-1}(\mathcal{M})\,.\]
    Weiter gilt
    \[H_i(\mathcal{M}^{\prime})\cong H_i(\mathcal{M}_0)/\langle\eqcl{e\mathrel{|}\mathcal{M}_0}\rangle\cong\ker\phi/\langle\eqcl{e\mathrel{|}\mathcal{M}}\rangle\,.\]
    Wegen \(\ker\phi\subseteq H_i(\mathcal{M})\) wird \(\eqcl{\mathcal{S}\mathrel{|}\mathcal{M}}\) eliminiert, wo\-m\"og\-lich mehr.
\end{proof}

\section{Gerahmte Chirurgie}
    Sei \({n=i+j+1}\), \(\mathcal{M}^n\) eine \(\pi\)-Mannigfaltigkeit und \(\Psi\colon\underline{\mathbb{D}}^{j+1}\hookrightarrow\mathring{\mathcal{M}}\) eine beliebige Anklebeeinbettung. Bezeichne durch \(\mathcal{S}:=\Psi(\mathbb{S}^i\times\mathbf{0})\) die Anklebesph\"are. Die naive Chirurgie bez\"uglich dieser Daten kann unter Umst\"anden eine Mannigfaltigkeit liefern, die nicht stabil parallelisierbar ist. Um dem beizukommen, reicht es aus, die Anklebeeinbettungen mithilfe von einer glatten Funktion \(\gamma\colon\mathbb{S}^i\to\operatorname{SO}(j+1)\) informiert zu \textit{verdrehen}. 

Sei \(F\) eine stabile Rahmung von \(\mathcal{M}\). Eine stabile Rahmung \(\overline{F}\) auf \(\mathcal{M}\multimap\mathbb{S}^i\) induziert offenbar eine Rahmung von \(\mathcal{M}\setminus\mathcal{S}\). Ist diese induzierte stabile Rahmung zu \(F\) homotop, hei\ss e die Chirurgie gerahmt. 

Zu einer Folge gerahmter Mannigfaltigkeiten \((\mathcal{M}_i,F_i)\) mit \(1\leq i\leq m\), sodass \(\mathcal{M}_{i+1}\) durch gerahmte Chirurgie aus \(\mathcal{M}_i\) erhalten werden kann, hei\ss en \(\mathcal{M}_1\) und \(\mathcal{M}_m\) zueinander \textbf{\(\chi\)-\"aquivalent}. Diese Relation ist offenbar reflexiv und transitiv. Sie ist symmetrisch, denn wird an der transversalen Sph\"are einer \(i\)-Chirurgie eine \(j\)-Chirurgie durchgef\"uhrt, ergibt sich erneut die originale Mannigfaltigkeit. Derart ist also eine \"Aquivalenz\-re\-la\-tion definiert.

\subsection{Strategie der Berechnung von \texorpdfstring{\(\Theta_k\)}{TEXT}}
    Sei \(P^k\) die Menge der \(\chi\)-\"Aquivalenzklassen gerahmter Mannigfaltigkeiten, die Homotopiesph\"aren beranden. Diese sei mit der verbundenen Randsumme versehen, sodass \(P^k\) die Struktur eines kommutativen Monoiden erh\"alt. Da Chirurgie stets am Inneren vorgenommen wird, hat diese keinen Einfluss auf den Rand, sodass \(\chi\)-\"aquivalente, gerahmte Mannigfaltigkeiten stets diffeomorphe R\"ander besitzen. Es folgt, dass der Randoperator \(\partial\colon P_k\to\Theta_k\) ein wohldefinierter Homomorphismus ist. Die Berechnung dieser Untergruppe ist der erste Schritt in der Berechnung von \(\Theta_k\). Es existiert die offensichtliche exakte Folge
    \[0\to\ker\partial\to P^k\mathop{\longrightarrow}^{\partial}\partial P^k\to0\,.\]
    Der Kern besteht hierbei gerade aus all den \(\chi\)-\"Aquivalenzklassen gerahmter Mannigfaltigkeiten, deren Rand zu der Standardsph\"are diffeomorph ist. Um die Gruppe \(\partial P^k\subseteq\Theta_k\) zu berechnen wird eine Mannigfaltigkeit aus \(P^k\) nun so lange durch gerahmte Chirurgie bearbeitet, bis sie m\"oglichst simpel ist. 

\subsection{Rahmbarkeit und Reparametrisierung}
    Sei erneut \({n=i+j+1}\), \((\mathcal{M}^n,F)\) eine stabil gerahmte Mannigfaltigkeit und \({\Psi\colon\underline{\mathbb{D}}^{j+1}\hookrightarrow\mathring{\mathcal{M}}}\) eine Anklebeeinbettung. Es soll untersucht werden, wann eine Verklebung mit der Standard-Sph\"are \((\mathbb{S}^n,E\)) eine stabil gerahmte Mannigfaltigkeit ergibt. Zun\"achst ist \(F(p)\) eine Basis von \({T_p\mathcal{M}\oplus\mathbb{R}}\) und \(E(p)\) eine Basis von \({T_p\mathbb{S}^n\oplus\mathbb{R}}\). Wird eine Umgebung von \({\mathbb{S}^i\subseteq\mathbb{S}^n}\) mit dem Definitionsbereich von \(\Psi\) identifiziert, ist die Einh\"angung des Differentials von \(\Psi\) eine Funktion
    \[S(\dx_p\Psi)\colon T_p\mathbb{S}^n\oplus\mathbb{R}\to T_{\Psi(p)}\mathcal{M}\oplus\mathbb{R}\,.\]
    Derart wird \(E\) zu einer stabilen Rahmung von \(\im\Psi\subseteq\mathcal{M}\) transportiert. Bezeichne jene Rahmung durch \(\dx\Psi(E)\). Sei \(\sigma(F,\Psi)\) der Rahmenwechsel von \(\dx\Psi(E)\) zu \(F\) auf \(\Psi(\mathbb{S}^i\times\mathbf{0})\). Da der Basiswechsel der transportierten Basis zu \(F\) gerade die Abbildungsmatrix von \(S(\dx_p\Psi)\) bez\"uglich \(E\) und \(F\) ist, gilt mit \(q=\Psi(p,0)\)
    \[\sigma(F,\Psi)\colon\mathbb{S}^i\to\operatorname{GL}^+(n+1),\,p\mapsto M_{F(q)}^{E(p,0)}\left(S(\dx_{(p,0)}\Psi)\right)\,.\]
    Derart wird ein Element \({\mathfrak{o}(F,\Psi):=\eqcl{\sigma(F,\Psi)}\in\pi_i(\operatorname{SO}(n+1))}\) definiert. Wegen folgendem Lemma kann \(\mathfrak{o}(F,\Psi)\) als Obstruktion betrachtet werden.
    \begin{lemma}\label{lem:nullhom_frame}
        Sei \((\mathcal{M},F)\) eine gerahmte Mannigfaltigkeit und \(\Psi\colon\underline{\mathbb{D}}^{j+1}\hookrightarrow\mathring{\mathcal{M}}\) eine An\-kle\-be\-ein\-bet\-tung. Dann ist die Verklebung von \((\mathcal{M},F)\) mit \((\mathbb{S}^n,E)\) genau dann gerahmt, wenn \({\mathfrak{o}\left(F,\Psi\right)=0}\) gilt.
    \end{lemma}
    \begin{proof}
        Sei \({(\mathcal{M}+\mathbb{S}^n,\overline{F})}\) zun\"achst gerahmt, sodass \(\overline{F}\) und \(F\) auf \({\mathcal{M}\setminus\Psi(\mathbb{S}^i\times\mathbf{0})}\) homotop sind. Der Rahmenwechsel von \(\overline{F}\) zu \(\dx\Psi(E)\) auf \(\Psi(\mathbb{S}^i\times\mathbb{S}^j)\) ist offenbar konstant, also ist der Rahmenwechsel von \(F\) zu \(\dx\Psi(E)\) auf \(\Psi(\mathbb{S}^i\times\mathbb{S}^j)\) nullhomotop. Es l\"asst sich leicht zeigen, dass dies genau dann der Fall ist, wenn eine nullhomotope Fortsetzung auf \({\Psi(\mathbb{S}^i\times\mathbb{D}^{j+1})\simeq\Psi(\mathbb{S}^i\times\mathbf{0})}\) existiert. Es folgt \({\mathfrak{o}\left(F,\Psi\right)=0}\).

        Sei umgekehrt \({\mathfrak{o}\left(F,\Psi\right)=0}\), der Rahmenwechsel von \(F\) zu \(\dx\Psi(E)\) auf der Anklebesph\"are sei also nullhomotop. Da diese ein Deformationsretrakt von \(\im\Psi\) ist, ist auch der Rahmenwechsel auf \(\im\Psi\) nullhomotop. Eine Nullhomotopie \({\im\Psi\times\mathbb{I}\to\operatorname{GL}^+(n+1)}\) zu der Identit\"at kann mithilfe der Homotopie\-er\-wei\-te\-rungs\-ei\-gen\-schaft zu einer Homotopie \({H\colon\mathcal{M}\times\mathbb{I}\to\operatorname{GL}^+(n+1)}\) fortgesetzt werden. Dann ist durch \({\overline{F}(p):=H(p,1)\cdot F(p)}\) eine Rahmung von \(\mathcal{M}\) definiert, die einerseits zu \(F\) homotop ist, und andererseits auf \(\im\Psi\) mit \(\dx\Psi(E)\) \"uber\-ein\-stimmt. Zusammen mit \(E\) induziert \(\overline{F}\) eine Rahmung auf \(\mathcal{M}+\mathbb{S}^n\).
    \end{proof}
    Sei \({n=i+j+1}\), \(\mathcal{M}\) eine Mannigfaltigkeit und \(\Psi\colon\underline{\mathbb{D}}^{j+1}\hookrightarrow\mathring{\mathcal{M}}\) eine Anklebe\-ein\-bet\-tung. Zu einer glatten Funktion \(\gamma\colon\mathbb{S}^{i-1}\to\operatorname{SO}(j+1)\) l\"asst sich der Vektorb\"undel\-auto\-mor\-phis\-mus
    \[\Gamma\colon\underline{\mathbb{D}}^{j+1}\to\underline{\mathbb{D}}^{j+1},\,(x,y)\mapsto\left(x,\gamma(x)\cdot y\right)\]
    definieren. Dann l\"asst sich die reparametrisierte Chirurgie mithilfe von \(\Psi\Gamma\) bilden. Besonders interessant ist hierbei der Effekt einer Reparametrisierung auf die Obstruktion der stabilen Parallelisierbarkeit. In folgendem Lemma bezeichne \(S^k\colon\operatorname{SO}(m)\to\operatorname{SO}(m+k)\) die \(k\)-fach iterierte Einh\"angung.
    \begin{lemma}\label{lem:reparam_eq}
        Sei \({n=i+j+1}\), \((\mathcal{M}^n,F)\) eine gerahmte Mannigfaltigkeit, \(\Psi\colon\underline{\mathbb{D}}^{j+1}\hookrightarrow\mathring{\mathcal{M}}\) eine Anklebeeinbettung und \(\gamma\colon\mathbb{S}^i\to\operatorname{SO}(j+1)\) glatt. Dann gilt die Gleichung
        \begin{equation}\label{eq:reparam_eq}
            \mathfrak{o}\left(F,\Psi\Gamma\right)=\mathfrak{o}\left(F,\Psi\right)+S_*^{i+1}\eqcl{\gamma}\in\pi_i\left(\operatorname{SO}(n+1)\right)\,.
        \end{equation}
    \end{lemma}
    \begin{proof}
        F\"ur das Differential
        \[\dx_{(p,q)}\Gamma\colon T_{(p,q)}(\mathbb{S}^i\times\mathbb{D}^{j+1})\to T_{(p,q)}(\mathbb{S}^i\times\mathbb{D}^{j+1})\]
        bez\"uglich der Standardbasis \((e_k)_{1\leq k\leq n}=E(p,q)\) gilt
        \[J\Gamma(p,q)=M_{E(p,q)}^{E(p,q)}\left(\dx_{(p,q)}\Gamma\right)=\begin{pmatrix}
            I_i & J\gamma(p)\cdot q\\
            0 & \gamma(p)
        \end{pmatrix}\in\operatorname{GL}^+(n)\,.\]
        Folglich gilt \({J\Gamma(p,0)=S^i\gamma(p)\in\operatorname{SO}(n)}\). F\"ur \(p\in\mathbb{S}^i\) folgt
        \begin{align*}
            \sigma(F,\Psi\Gamma)(p)&\overeq{Def}M_{F(q)}^{E(p,0)}\left(S(\dx_{(p,0)}(\Psi\Gamma))\right)\\
            &=M_{F(q)}^{E(p,0)}\left(S(\dx_{\Gamma(p,0)}\Psi)\circ S(\dx_{(p,0)}\Gamma)\right)\\
            &=M_{F(q)}^{E(p,0)}\left(S(\dx_{(p,0)}\Psi)\right)\cdot M_{E(p,0)}^{E(p,0)}\left(S(\dx_{(p,0)}\Gamma)\right)\\
            &=\sigma(F,\Psi)(p)\cdot S\left(J\Gamma(p,0)\right)\\
            &=\sigma(F,\Psi)(p)\cdot S^{i+1}\gamma(p)\,.
        \end{align*}
        Dies zeigt \(\mathfrak{o}\left(F,\Psi\Gamma\right)=\mathfrak{o}\left(F,\Psi\right)+S_*^{i+1}\eqcl{\gamma}\).
    \end{proof}
    Insbesondere hat Reparametrisierung mit \(\eqcl{\gamma}\in\ker S_*^{i+1}\) keine Auswirkung auf stabile Parallelisierbarkeit.
    \newpage
    \begin{theorem}\label{thm:surg_framable}
        Sei \(n=i+j+1\), \((\mathcal{M}^n,F)\) eine gerahmte Mannigfaltigkeit und \(\mathcal{S}^i\) eine eingebettete Sph\"are mit trivialem Normalenb\"undel. Gilt \(i\leq j\), l\"asst sich an \(\mathcal{S}\) eine gerahmte Chirurgie durch\-f\"uh\-ren.
    \end{theorem}
    \begin{proof}
        Sei \(\Psi\colon\underline{\mathbb{D}}^{j+1}\to\mathring{\mathcal{M}}\) eine beliebige Anklebeabbildung, sodass die Obstruktion \(\mathfrak{o}(F,\Psi)\in\pi_i\left(\operatorname{SO}(n+1)\right)\) definiert ist. Aus der Annahme folgt, dass der von der \(i\)-fach iterierten Einh\"angung induzierte Homomorphismus
        \[S_*^{i+1}\colon\pi_i\left(\operatorname{SO}(j+1)\right)\to\pi_i\left(\operatorname{SO}(n+1)\right)\]
        surjektiv ist, es existiert also ein \(\eqcl{\gamma}\in\pi_i\left(\operatorname{SO}(j+1)\right)\) mit einem glatten Re\-pr\"a\-sen\-tan\-ten \(\gamma\), sodass f\"ur die Reparametrisierung \(\Gamma\) gem\"a\ss{} Lemma \ref{lem:reparam_eq}
        \[\mathfrak{o}\left(F,\Psi\Gamma\right)=\mathfrak{o}\left(F,\Psi\right)+S_*^{i+1}\eqcl{\gamma}=0\]
        gilt. Gem\"a\ss{} Lemma \ref{lem:nullhom_frame} ist eine Chirurgie an \(\mathcal{S}\) mittels \(\Psi\Gamma\) gerahmt.
    \end{proof}

    \begin{theorem}\label{thm:srg_conn}
        Eine gerahmte Mannigfaltigkeit \((\mathcal{M}^n,F)\) mit \({n\geq2k}\) ist zu einer \({(k-1)}\)-zu\-sam\-men\-h\"angen\-den gerahmten Mannigfaltigkeit \(\chi\)-\"aquivalent.
    \end{theorem}
    \begin{proof}
        Es wird Induktion \"uber \(k\) gef\"uhrt. Durch \(0\)-Chirurgie kann \(\mathcal{M}\) als zusammenh\"angend angenommen werden. Sei \(\mathcal{M}\) nun bereits \((k-2)\)-zusammen\-h\"angend. Dann kann ein beliebiges Element in \(\pi_{k-1}(\mathcal{M})\) durch eine Einbettung \(\mathbb{S}^{k-1}\hookrightarrow\mathcal{M}\) repr\"asentiert werden. Da die Bildsph\"are wegen Satz \ref{thm:vec_dim_triv} ein triviales Normalenb\"undel besitzt, kann Chirurgie an ihr durchgef\"uhrt werden. Mithilfe von Satz \ref{thm:surg_framable} ergibt dies erneut eine gerahmte Mannigfaltigkeit \(\mathcal{M}^{\prime}\). Wegen Lemma \ref{lem:fund_smaller} ist \(\pi(\mathcal{M}^{\prime})\) echt kleiner als \(\pi(\mathcal{M})\). Endlich viele Schritte f\"uhren zu einer \((k-1)\)-zusammen\-h\"angen\-den Mannigfaltigkeit.
    \end{proof}

\subsection{Gerahmte Henkel}
    Die vorangegangenen S\"atze gelten in \"ahnlicher Art und Weise, wenn mit Henkeln anstatt mit Chirurgie gearbeitet wird. Ein zu Lemma \ref{lem:nullhom_frame} nahezu analoger Beweis ergibt:
    \begin{lemma}
        Sei \((\mathcal{W},F)\) eine gerahmte Mannigfaltigkeit und \(\Psi\colon\underline{\mathbb{D}}^{j+1}\hookrightarrow\partial\mathcal{W}\) eine An\-kle\-be\-ein\-bet\-tung und \(\tilde{\Psi}\colon\underline{\mathbb{D}}^{j+1}\hookrightarrow\mathcal{W}\) eine Fortsetzung. Dann ist die Verklebung von \((\mathcal{W},F)\) mit \((\mathbb{D}^n,E)\) genau dann rahmbar, wenn \({\mathfrak{o}\left(F|_{\partial\mathcal{W}},\Psi\right)=0}\) gilt.
    \end{lemma}
    Die Obstruktionen der Rahmbarkeit von \(\mathcal{W}\multimap\mathbb{D}^{i+1}\) und \(\partial\mathcal{W}\multimap\mathbb{S}^i\) stimmen also \"uberein. Eine Reparametrisierung ist nun eine Abbildung
    \[\tilde{\Gamma}\colon\underline{\mathbb{D}}^j\times\mathbb{R}_+\to\underline{\mathbb{D}}^j\times\mathbb{R}_+,\,(x,y,t)\mapsto\left(x,\gamma(x)\cdot y,t\right)\,,\]
    und setzt damit die Reparametrisierung der Chirurgie an \(\partial\mathcal{W}\) trivial fort. F\"ur das folgende ist noch folgendes Lemma wichtig.
    \begin{lemma}\label{lem:smoothing_pi}
        Zu einer gerahmten Mannigfaltigkeit \(\mathcal{M}^n\) und \({i\leq j}\) existiert eine gerahmte Man\-nig\-faltig\-keit \(\mathcal{W}^{n+1}\) mit
        \[\partial\mathcal{W}\cong\mathcal{M}\mathop{+}^{\partial\mathcal{M}}\left(\mathcal{M}\multimap\mathbb{S}^i\right)\,.\]
    \end{lemma}
    \begin{proof}
        Sei \(\mathcal{W}\subseteq\mathcal{M}\times\mathbb{I}\) mit 
        \[\partial\mathcal{W}\cong\mathcal{M}\mathop{+}^{\partial\mathcal{M}}\mathcal{M}\,.\]
        gem\"a\ss{} Lemma \ref{lem:smooth_man}. Da die stabile Rahmung von \(\mathcal{M}\) eine Rahmung von \({\mathcal{M}\times\mathbb{R}}\) induziert, induziert diese auch eine Rahmung von \(\mathcal{W}\). Wegen \(i\leq j\) kann an \(\mathcal{W}\) derart ein gerahmter Henkel angebracht werden, dass die resultierende Mannigfaltigkeit dem Lemma  gen\"ugt.
    \end{proof}

\section{Probleme in gerader Dimension}
    Sei \(\mathcal{M}^{2k}\) eine \((k-1)\)-zusammenh\"angende \(\pi\)-Mannigfaltigkeit mit \(k\geq3\), deren Rand entweder leer oder eine Homotopiesph\"are ist. Um \(\mathcal{M}\) durch Chirurgie in eine \(k\)-zusammenh\"angende \(\pi\)-Mannigfaltigkeit zu \"uberf\"uhren m\"ussen einige Hindernisse beachtet werden. Einerseits muss das Normalenb\"undel einer eingebetteten \(k\)-Sph\"are nicht unbedingt trivial sein, andererseits ist nicht jede Sph\"are primitiv und vereinfacht somit nicht unbedingt die Homologie. In diesem Fall ist die mittlere Homologiegruppe \(H_k(\mathcal{M})\) jedoch frei, sodass keine Komplikationen durch Torsion auftreten. 
\begin{lemma}\label{lem:middle_free}
    Sei \(\mathcal{M}^{2k}\) mit \({H_k(\partial\mathcal{M})=H_{k-1}(\partial\mathcal{M})=0}\) derart, dass \(H_{k-1}(\mathcal{M})\) frei ist. Dann ist auch \(H_k(\mathcal{M})\) frei.
\end{lemma}
\begin{proof}
    Aus der Annahme an den Rand folgt, dass \(q^*\colon H_k(\mathcal{M})\to H_k(\mathcal{M},\partial\mathcal{M})\) ein Isomorphismus ist. Sei \(T_k\) die Torsionsuntergruppe von \(H_k(\mathcal{M})\). Es folgt
    \[H_k(\mathcal{M})\mathop{\cong}^{q^*}H_k(\mathcal{M},\partial\mathcal{M})\overcong{P.D.}H^k(\mathcal{M})\overcong{U.K.}\operatorname{Hom}(H_k(\mathcal{M}),\mathbb{Z})\cong H_k(\mathcal{M})/T_k\,.\]
    Somit muss \(T_k=0\), und \(H_k(\mathcal{M})\) frei sein.
\end{proof}

\subsection{Repr\"asentierende Sph\"aren}
    Sei \(\mathcal{M}^{2k}\) eine \((k-1)\)-zusammenh\"angende \(\pi\)-Mannigfaltigkeit und \(x\in H_k(\mathcal{M})\). Gem\"a\ss{} dem Satz von Hurewicz ist
    \[h\colon\pi_k(\mathcal{M})\to H_k(\mathcal{M}),\,\eqcl{\gamma}\mapsto\gamma_*\big(\eqcl{\mathbb{S}^k}\big)\]
    ein Isomorphimus. Es existiert also eine bis auf Homotopie eindeutig bestimmte Funktion \(\phi\colon\mathbb{S}^k\to\mathcal{M}\) mit \(\phi_*\eqcl{\mathbb{S}^k}=x\). Aufgrund eines Satzes von Haelfiger (\cite{haefliger1961differentiable} Satz 1) kann f\"ur \(k\geq3\) angenommen werden dass \(\phi\) eine Einbettung ist. Bezeichne die eingebettete Sph\"are durch \(\mathcal{S}(x)\). Diese ist f\"ur \(k\geq4\) bis auf Isotopie eindeutig bestimmt, sodass sich die Homotopieklasse der Kupplungsfunktion des Normalenb\"undels von \(\mathcal{S}(x)\) in \(\mathcal{M}\) betrachten l\"asst. F\"ur \(k\geq4\) ist also die Funktion
    \[\alpha\colon H_k(\mathcal{M})\to\pi_{k-1}(\operatorname{SO}(k)),\,x\mapsto\eqcl{\nu(\mathcal{S}(x))}\]
    wohldefiniert. Aufgrund von \(\pi_2(\operatorname{SO}(3))=0\) l\"asst sich \(\alpha\) auch f\"ur \(k=3\) definieren. Es folgt nun aus vorherigen \"Uberlegungen \ref{subsec:stable_normal_bundle}, das \(\alpha(x)\) ein ganzzahliges Vielfaches der Kupplungsfunktion des Normalenb\"undels ist, und
    \begin{equation}\label{eq:alpha_kernel}
        \alpha(x)\in\ker S_*\cong\begin{cases}
            \mathbb{Z} & k=2m\\
            \mathbb{Z}_2 & k=2m+1\notin\{3,7\}\\
            0 & k\in\{3,7\}
        \end{cases}
    \end{equation}
    gilt. Die Entscheidung, wann das Normalenb\"undel trivial ist, h\"angt also von der Parit\"at von \(k\) ab.

\subsection{Symplektische Basen}
    \begin{definition}[Schwach symplektische Basis]
        Sei \(M^{2k}\) ein freier \(R\)-Modul. Eine Basis \((e_i,f_i)_{1\leq i\leq k}\) von \(M\), hei\ss e schwach symplektisch bez\"uglich einer Bilinearform \(\beta\colon M\otimes M\to R\), falls
        \[\beta(e_i,f_j)=\delta_{ij}\quad\text{und}\quad\beta(e_i,e_j)=\beta(f_i,f_j)=0\quad\text{f\"ur alle}\quad1\leq i,j\leq k\,.\]
    \end{definition}
    Ist \(\beta\) schiefsymmetrisch, hei\ss e sie symplektisch. Die Signifikanz dieser Basen liegt unter anderem in folgendem Satz.
    \begin{theorem}\label{thm:even_symp_ann}
        Sei \(k\geq3\) und \(\mathcal{M}^{2k}\) eine \((k-1)\)-zusammenh\"angende, gerahmte Mannigfaltigkeit, deren Rand leer oder eine Homotopiesph\"are ist, sodass \(H_k(\mathcal{M})\) eine schwach symplektische Basis \(e_i,f_i\) mit \(\alpha(e_i)=0\) besitzt. Dann kann \(H_k(\mathcal{M})\) durch gerahmte Chirurgie vernichtet werden.
    \end{theorem}
    \begin{proof}
        Da \(e_1\) ein triviales Normalenb\"undel besitzt, kann Chirurgie an \(e_1\) durchgef\"uhrt werden. Sei \(\mathcal{S}\) die Anklebesph\"are und \(\mathcal{M}^{\prime}\) die resultierende Mannigfaltigkeit. Da die \(e_i,f_i\) eine schwach symplektische Basis bilden, ist \(e_1\) primitiv, und es gilt f\"ur
        \[\phi\colon H_i(\mathcal{M})\to\mathbb{Z},\,x\mapsto x\cdot e_1\]
        gerade \(\ker\phi=H_i(\mathcal{M})/\langle f_1\rangle\). Aus Satz \ref{lem:even_surg_effect} folgt
        \[H_i(\mathcal{M}^{\prime})\cong\ker\phi/\langle e_1\rangle\cong H_i(\mathcal{M})/\langle e_1,f_1\rangle\,.\]
        \"Uber Transversalit\"at kann angenommen werden, dass die Bilder \(e_i^{\prime},f_i^{\prime}\in H_i(\mathcal{M}^{\prime})\) durch Sph\"aren in \(\mathcal{M}\setminus\mathcal{S}\) repr\"asentiert werden, sodass sich die Schnittzahlen nicht \"andern und die \(e_i^{\prime},f_i^{\prime}\) erneut eine symplektische Basis mit \(\alpha(e_i^{\prime})=0\) bilden. Die Aussage folgt rekursiv.
    \end{proof}
    Es wird im Folgenden noch eine weitere symplektische Basis \({e_i,f_i\in H_k(\partial\mathcal{W})}\) ben\"otigt werden, wobei alle \(e_i\) im Kern der Inklusion \({\partial\mathcal{W}\hookrightarrow\mathcal{W}^{2k+1}}\) liegen. Hierbei sei stets an den Torus erinnert. Es gilt \({H_1(\mathbb{S}^1\times\mathbb{S}^1)\cong\mathbb{Z}\oplus\mathbb{Z}}\), wobei die Erzeuger eine symplektische Basis bilden. Eines der Elemente liegt im Kern der Inklusion \(\mathbb{S}^1\times\mathbb{S}^1\hookrightarrow\mathbb{D}^2\times\mathbb{S}^1\) des Torus als Rand des Volltorus. Es wird folgendes Lemma ben\"otigt. Im folgenden Satz bezeichne \(\operatorname{Ad}\) zu einem K\"orper \(\mathbb{K}\) die Abbildung
    \[\operatorname{Ad}\colon H_i(\partial\mathcal{W};\mathbb{K})\to\operatorname{Hom}(H_{n-i}(\partial\mathcal{W};\mathbb{K}),\mathbb{K}),\,x\mapsto x\cdot(\mathrel{-})\,.\]
    \begin{lemma}\label{lem:inter_annihilates}
        Sei \(\mathbb{K}\) ein K\"orper, \(\mathcal{W}^{2k+1}\) eine Mannigfaltigkeit, 
        \[A_i:=\ker\left(\iota_*\colon H_i(\partial\mathcal{W};\mathbb{K})\to H_i(\mathcal{W};\mathbb{K})\right)\]
        und \(R\colon\operatorname{Hom}(H_{n-i}(\partial\mathcal{W};\mathbb{K}),\mathbb{K})\to\operatorname{Hom}(A_{n-i},\mathbb{K})\) die Einschr\"ankung, dann ist der Kern von \(F:=R\circ\operatorname{Ad}\) gerade \(A_i\).
    \end{lemma}
    \begin{proof}
        Ein \(x\in H_i(\partial\mathcal{W};\mathbb{K})\) liegt genau dann in \(\ker F\), wenn \({(A_{n-i})\cdot x=0}\) ist. Betrachte das bis auf Vorzeichen kommutive Diagramm
        \begin{center}
            \begin{tikzpicture}
                \draw 
                    (-3, 0) node (A) {\(H^{n-i}(\mathcal{W};\mathbb{K})\)}
                    (0, 0) node (B) {\(H^{n-i}(\partial\mathcal{W};\mathbb{K})\)}
                    (3.5, 0) node (C) {\(H^{n-i+1}(\mathcal{W},\partial\mathcal{W};\mathbb{K})\)}
                    (0, -2) node (E) {\(H_i(\partial\mathcal{W};\mathbb{K})\)}
                    (3.5, -2) node (F) {\(H_i(\mathcal{W};\mathbb{K})\)}

                    (A) edge [-stealth] node [above] {\(\iota^*\)} (B)
                    (B) edge [-stealth] node [above] {\(\delta\)} (C)
                    (B) edge [-stealth] node [left] {\(\frown\eqcl{\partial\mathcal{W}}\)} (E)
                    (C) edge [-stealth] node [left] {\(\frown\eqcl{\mathcal{W}}\)} (F)
                    (E) edge [-stealth] node [above] {\(\iota_*\)} (F)
                    ;
            \end{tikzpicture}
        \end{center}
        F\"ur \(x\in H_i(\partial\mathcal{W};\mathbb{K})\) gilt
        \[(A_{n-i})\cdot x=\langle(\ker\iota_*)^*,x\rangle=\langle\im\iota^*,x\rangle=\langle H^{n-i}(\mathcal{W};\mathbb{K}),\iota_*x\rangle\,.\]
        Also liegt \(x\) genau dann im Kern von \(F\), wenn \(\langle H^{n-i}(\mathcal{W};\mathbb{K}),\iota_*x\rangle=0\) ist. Da \(\mathbb{K}\) ein K\"orper und die Kronecker-Paarung nicht degeneriert ist, ist dies zu \(x\in\ker\iota_*=A_i\) \"aquivalent.
    \end{proof}
    Die Beweisidee entstammt hierbei \cite{tomdieck2008algebraic} Proposition 18.7.5. Es ergibt sich somit \(\dim_{\mathbb{K}}A_{n-i}=\dim_{\mathbb{K}}H_i(\partial\mathcal{W};\mathbb{K})-\dim_{\mathbb{K}}A_i\), insbesondere also
    \begin{equation}\label{eq:ker_incl_dim}
        \dim_{\mathbb{K}}H_k(\partial\mathcal{W};\mathbb{K})=2\dim_{\mathbb{K}}A_k=2\dim_{\mathbb{K}}\ker\iota_*\,.
    \end{equation}
    Beachte, dass die Forderung, dass \(\mathbb{K}\) ein K\"orper n\"otig ist, da \(H_k(\mathcal{W})\) Torsion aufweisen k\"onnte. In diesem Fall k\"onnte die Kronecker-Paarung degeneriert sein. Wird Torsion herausgeteilt, gilt der vorherige Satz auch f\"ur \(\mathbb{Z}\)-Koeffizienten. Siehe etwa \cite{hatcher2002algebraic} Proposition 3.38. F\"ur die Abbildung
    \[\iota_*^{\prime}\colon H_k(\partial\mathcal{W})\xrightarrow{\iota_*}H_k(\mathcal{W})\xrightarrow{p}H_k(\mathcal{W})/T_k(\mathcal{W})\]
    gilt also gerade \(2\operatorname{Rang}\ker\iota_*^{\prime}=\operatorname{Rang}H_k(\partial\mathcal{W})\). Da das Herausteilen der Torsion keinen Einfluss auf den Rang hat, gilt \(\operatorname{Rang}\ker\iota_*=\operatorname{Rang}\ker\iota_*^{\prime}\) und
    \begin{equation}
        \operatorname{Rang}H_k(\partial\mathcal{W})=2\operatorname{Rang}\ker\iota_*\,.
    \end{equation}

    \begin{theorem}\label{thm:symp_base_ann}
        Sei \(k\) ungerade und \(\mathcal{M}^{2k}\) eine Mannigfaltigkeit mit \(H_k(\partial\mathcal{M})=H_{k-1}(\partial\mathcal{M})=0\). Dann existiert eine symplektische Basis von \(H_k(\mathcal{M})\) mit \(e_i\in\ker\iota_*\).
    \end{theorem}
    \begin{proof}
        Da \(k\) ungerade ist, besitzt \(H_k(\mathcal{M})\) geraden Rang, und es gilt \(x\cdot x=0\) f\"ur alle \(x\in H_k(\mathcal{M})\). Sei \(e_1\in\ker\iota_*\). Da die Schnittform unimodular ist, existiert ein \(f_1\in H_k(\mathcal{M})\) mit \(e_1\cdot f_1=1\). Da die Schnittform unimodular ist, spaltet \(f_i\) von \(H_k(\mathcal{M})\) und \(e_i\) von \(\ker\iota_*\) ab, sodass die Aussage rekursiv folgt.
    \end{proof}

    \chapter{Chirurgie in ungerader Dimension}
        Sei \(\mathcal{M}^{2k+1}\) eine \(\pi\)-Mannigfaltigkeit, deren Rand entweder leer oder eine Homotopiesph\"are sei. Das Normalenb\"undel einer jeden eingebetteten \(k\)-Sph\"are ist aus Dimensionsgr\"unden \ref{thm:vec_dim_triv} trivial, sodass stets Chirurgie durchgef\"uhrt werden kann. Leider ist \(H_k(\mathcal{M})\) in diesem Fall nicht zwingenderma\ss en frei. Der freie Anteil der Gruppe \(H_k(\mathcal{M})\) ist aufgrund des folgenden Lemmas kein Problem.
\begin{lemma}\label{lem:odd_dim_finite}
    Sei \(\mathcal{M}^{2k+1}\) eine gerahmte Mannigfaltigkeit, sodass \(H^k(\partial\mathcal{M})=0\) ist. Dann kann \(H_k(\mathcal{M})\) durch gerahmte Chirurgie auf ihre Torsionsgruppe reduziert werden.
\end{lemma}
\begin{proof}
    Wenn \(\operatorname{Rang}H_k(\mathcal{M})=0\) gilt, ist die Aussage trivial. Es existiere also ein Generator \(g\) eines direkten \(\mathbb{Z}\)-Summanden von \(H_k(\mathcal{M})\) und ein Homomorphismus \({f\colon H_k(\mathcal{M})\to\mathbb{Z}}\) mit \(f(g)=1\). Aus dem Satz \"uber universelle Koeffizienten folgt, dass \({H^k(\mathcal{M})\to\operatorname{Hom}(H_k(\mathcal{M}),\mathbb{Z})}\) surjektiv ist, sodass ein \(x\in H^k(\mathcal{M})\) mit \(f=\langle x,\mathrel{-}\rangle\) existiert. Aus der Exaktheit der Folge
    \[H^k(\mathcal{M},\partial\mathcal{M})\xrightarrow{q^*}H^k(\mathcal{M})\to H^k(\partial\mathcal{M})=0\]
    folgt, dass auch ein \({h^*\in H^k(\mathcal{M},\partial\mathcal{M})}\) mit \(q^*h^*=x\) existiert. F\"ur das Duale \(h\in H_{k+1}(\mathcal{M})\) folgt
    \[h\cdot g\mathop{=}^{\text{\tiny\eqref{eq:intersect_prop_2}}}\langle q^*h^*,g\rangle=\langle x,g\rangle=f(g)=1\,.\]
    Folglich ist \(g\) primitiv. Da \(g\) von einer eingebetteten Sph\"are mit trivialem Normalenb\"undel repr\"asentiert werden kann, f\"uhrt Chirurgie an \(g\) gem\"a\ss{} Lemma \ref{lem:odd_surg_effect} zu einer Vereinfachung der Homologie. Die Aussage folgt rekursiv.
\end{proof}
Es kann somit angenommen weiter angenommen werden, dass \(H_k(\mathcal{M})\) eine Torsionsgruppe ist. Ob eine Chirurgie die Homologie vereinfacht h\"angt nun von der Parit\"at von \(k\) ab. 

\section{Die Berechnung von \texorpdfstring{\(P^{4m+1}\)}{TEXT}}
    Der am einfachsten zu behandelnde Fall tritt ein, wenn \(k\) gerade ist. Dann ist der Effekt einer Chirurgie an einem Torsionselement, dass dieses durch ein zyklisch unendliches Element ersetzt wird. Somit wird die Gruppe zun\"achst zwar gr\"o\ss er, die Torsionsgruppe wird jedoch definitiv kleiner. Da zyklisch unendliche Elemente mithilfe von Lemma \ref{lem:odd_dim_finite} leicht vernichtet werden k\"onnen, kann auf diese Art und Weise immer eine \(k\)-zusammenh\"angende \(\pi\)-Mannigfaltigkeit erhalten werden. Sei \(\mathbb{K}\) ein K\"orper. Bezeichne
    \[b_i\left(\mathcal{W};\mathbb{K}\right):=\dim_{\mathbb{K}}H_i\left(\mathcal{W};\mathbb{K}\right)\quad\text{und}\quad b_i\left(\mathcal{W},\partial\mathcal{W};\mathbb{K}\right):=\dim_{\mathbb{K}}H_i\left(\mathcal{W},\partial\mathcal{W};\mathbb{K}\right)\,.\]
    F\"ur \(\mathbb{K}=\mathbb{Q}\) ist dies einfach die \(i\)-te Betti-Zahl von \(\mathcal{W}\). Beachte, dass aus der Poincar\'e-Dua\-li\-t\"at bereits \(b_i(\mathcal{W};\mathbb{K})=b_{2\ell-i}(\mathcal{W},\partial\mathcal{W};\mathbb{K})\) folgt. Betrachte zun\"achst folgenden Hilfssatz.
    \begin{lemma}\label{lem:inter_rank_mod}
        Die Schnittpaarung einer Mannigfaltigkeit \(\mathcal{W}^{2\ell}\) mit Werten in einem K\"orper \(\mathbb{K}\) besitzt Rang
        \begin{equation}\label{eq:inter_rank_mod}
            r\equiv\chi(\mathcal{W})+\sum_{i=0}^{\ell-1}b_i\left(\partial\mathcal{W};\mathbb{K}\right)\mod 2\,.
        \end{equation}
    \end{lemma}
    \begin{proof}
        Der Rang der Schnittpaarung ist gerade der Rang der Abbildung
        \[\operatorname{Ad}\colon H_{\ell}(\mathcal{W};\mathbb{K})\to\operatorname{Hom}(H_{\ell}(\mathcal{W};\mathbb{K}),\mathbb{K})\,,\]
        die gem\"a\ss{} Lemma \ref{lem:intersect_factor} gleich der Komposition
        \[H_{\ell}(\mathcal{W};\mathbb{K})\xrightarrow{q_*}H_{\ell}(\mathcal{W},\partial\mathcal{W};\mathbb{K})\xrightarrow{\text{\tiny P.D.}} H^{\ell}(\mathcal{W};\mathbb{K})\xrightarrow{\text{\tiny U.K.}}\operatorname{Hom}(H_{\ell}(\mathcal{W};\mathbb{K}),\mathbb{K})\]
        ist. Da die hinteren beiden Abbildungen Isomorphismen sind, ist der Rang von \(\operatorname{Ad}\) einfach der Rang von \(q_*\). Betrachte die lange exakte Folge
        \[\dots\to H_{\ell}(\mathcal{W};\mathbb{K})\xrightarrow{q_*}H_{\ell}(\mathcal{W},\partial\mathcal{W};\mathbb{K})\xrightarrow{\partial}H_{\ell-1}(\partial\mathcal{W};\mathbb{K})\xrightarrow{\iota_*}H_{\ell-1}(\mathcal{W};\mathbb{K})\to\dots\]
        von \(\mathbb{K}\)-Vektorr\"aumen. Aus dem Rangsatz folgt
        \[\operatorname{Rang}q_*\overeq{Def.}\dim_{\mathbb{K}}\im q_*=\dim_{\mathbb{K}}\ker\partial=\dim_{\mathbb{K}}H_{\ell}(\mathcal{W},\partial\mathcal{W};\mathbb{K})-\operatorname{Rang}\partial\,,\]
        sodass sich induktiv ergibt, dass \(\operatorname{Rang}q_*\) die alternierende Summe der Dimensionen aller Vektorr\"aume rechts von \(H_{\ell}(\mathcal{W},\partial\mathcal{W};\mathbb{K})\) ist. Modulo zwei ist also
        \[\operatorname{Rang}q_*\equiv\sum_{i=0}^{\ell-1}b_i\left(\mathcal{W};\mathbb{K}\right)+\sum_{i=0}^{\ell-1}b_i\left(\partial\mathcal{W};\mathbb{K}\right)+\sum_{i=0}^{\ell}b_i(\mathcal{W},\partial\mathcal{W};\mathbb{K})\mod2\,.\]
        Wegen \(b_i(\mathcal{W},\partial\mathcal{W};\mathbb{K})=b_{2\ell-i}\left(\mathcal{W};\mathbb{K}\right)\) folgt weiter
        \[\operatorname{Rang}q_*\equiv\sum_{i=0}^{2\ell}b_i\left(\mathcal{W};\mathbb{K}\right)+\sum_{i=0}^{\ell-1}b_i\left(\partial\mathcal{W};\mathbb{K}\right)\equiv\chi(\mathcal{W})+\sum_{i=0}^{\ell-1}b_i\left(\partial\mathcal{W};\mathbb{K}\right)\mod2\,.\]
    \end{proof}
    \begin{lemma}\label{lem:srg_chg_field_betti}
        Sei \(\mathbb{K}\) ein K\"orper, \(\mathcal{M}^{2k+1}\) eine \((k-1)\)-zusammenh\"angende, geschlossene Mannigfaltigkeit und \({\mathcal{W}^{2k+2}:=\mathcal{M}\times\mathbb{I}\multimap\mathbb{D}^{k+1}}\), sodass die Schnittform von \(\mathcal{W}\) geraden Rang besitzt. Dann gilt \(b_k(\mathcal{M};\mathbb{K})\not=b_k(\mathcal{M}\multimap\mathbb{S}^k;\mathbb{K})\).
    \end{lemma}
    \begin{proof}
        Setze \(\mathcal{M}^{\prime}:=\mathcal{M}\multimap\mathbb{S}^k\). Die Euler-Cha\-rak\-te\-ris\-tik einer kompakten Mannigfaltigkeit ungerade Dimension (mit oder ohne Rand) ist gleich null. Da \(\mathcal{W}\) homotopie\"aquivalent zu \(\mathcal{M}\) mit einer \((k+1)\)-Zelle ist, gilt also
        \begin{equation}\label{eq:w_euler}
            \chi(\mathcal{W})=\chi(\mathcal{M})+(-1)^{k+1}=(-1)^{k+1}\,.
        \end{equation}
        Da die Schnittpaarung auf \(\mathcal{W}\) geraden Rang besitzt, folgt
        \begin{equation}\label{eq:betti_cong}
            \begin{array}{rll}
                0\hspace{-2pt}&\displaystyle\hspace{-4pt}\mathop{\equiv}^{\text{\tiny\eqref{eq:inter_rank_mod}}}\chi(\mathcal{W})+\sum_{i=0}^k\big(b_i\left(\mathcal{M};\mathbb{K}\right)+b_i\left(\mathcal{M}^{\prime};\mathbb{K}\right)\big)&\mod2\\
                &\displaystyle\hspace{-4pt}\mathop{\equiv}^{\text{\tiny\eqref{eq:w_euler}}}1+\sum_{i=0}^kb_i\left(\mathcal{M};\mathbb{K}\right)+\sum_{i=0}^kb_i\left(\mathcal{M}^{\prime};\mathbb{K}\right)&\mod2\,.
            \end{array}
        \end{equation}
        Da \(\mathcal{M}\) und \(\mathcal{M}^{\prime}\) jeweils \((k-1)\)-zusammenh\"angend sind, zeigt dies
        \[b_k\left(\mathcal{M};\mathbb{K}\right)=\sum_{i=1}^kb_i\left(\mathcal{M};\mathbb{K}\right)\mathop{\neq}^{\text{\tiny\eqref{eq:betti_cong}}}\sum_{i=1}^kb_i\left(\mathcal{M}^{\prime};\mathbb{K}\right)=b_k\left(\mathcal{M}^{\prime};\mathbb{K}\right)\,.\]
    \end{proof}
    \begin{corollary}\label{cor:srg_chg_betti}
        Ist \(k\) gerade und \(\mathcal{M}^{2k+1}\) eine Mannigfaltigkeit, deren Rand leer oder eine Homotopiesph\"are ist, ver\"andert eine \(k\)-Chirurgie die \(k\)-te Betti-Zahl.
    \end{corollary}
    \begin{proof}
        Ist \(\mathcal{M}\) geschlossen, folgt dies aus Lemma \ref{lem:srg_chg_field_betti}, da die Schnittform auf \(\mathcal{M}\times\mathbb{I}\multimap\mathbb{D}^{k+1}\) schiefsymmetrisch ist, und deshalb geraden Rang besitzt. Sei nun \(\partial\mathcal{M}\) eine Homotopiesph\"are. Betrachte die geschlossene Mannigfaltigkeit
        \[\mathcal{N}:=\mathcal{M}\mathop{+}^{\partial\mathcal{M}}\mathcal{M}\,.\]
        Dann gelten
        \[H_k(\mathcal{N})\cong H_k(\mathcal{M})\oplus H_k(\mathcal{M})\quad\text{und}\quad H_k(\mathcal{N}\multimap\mathbb{S}^k)\cong H_k(\mathcal{M})\oplus H_k(\mathcal{M}\multimap\mathbb{S}^k)\,.\]
        Also \(b_k(\mathcal{N})=2b_k(\mathcal{M})\) und \(b_k(\mathcal{N}\multimap\mathbb{S}^k)=b_k(\mathcal{M})+b_k(\mathcal{M}^{\prime})\). Da der Satz f\"ur geschlossene Mannigfaltigkeiten bereits gezeigt wurde, folgt
        \[2b_k(\mathcal{M})\not=b_k(\mathcal{M})+b_k(\mathcal{M}^{\prime})\quad\text{also}\quad b_k(\mathcal{M})\not=b_k(\mathcal{M}^{\prime})\,.\]
    \end{proof}
    Damit sind bereits alle notwendigen Werkzeuge gegeben um \(P^{4m+1}\) zu berechnen.
    \begin{theorem}\label{thm:4m+1}
        Eine gerahmte Mannigfaltigkeit \(\mathcal{M}^{4m+1}\), deren Rand leer oder eine Homotopiesph\"are ist, ist zu einer \(k\)-zusammenh\"angenden gerahmten Mannigfaltigkeit \(\chi\)-\"aquivalent.
    \end{theorem}
    \begin{proof}
        Aus Lemma \ref{lem:odd_dim_finite} folgt, dass \(\mathcal{M}^{2k}\) als \((k-1)\)-zu\-sam\-men\-h\"ang\-end und \(H_k(\mathcal{M})\) als endlich angenommen werden kann, also Betti-Zahl null besitzt. Ist \(H_k(\mathcal{M})=0\) ist die Aussage trivial. Es existiere also ein nicht-triviales Element in \(H_k(\mathcal{M})\), welches durch eine eingebettete Sph\"are \(\mathcal{S}\) repr\"asentiert wird, die gem\"a\ss{} Satz \ref{thm:vec_dim_triv} ein triviales Normalenb\"undel besitzt. Eine Chirurgie an \(\mathcal{S}\) ergibt gem\"a\ss{} Korollar \ref{cor:srg_chg_betti} eine Mannigfaltigkeit \(\mathcal{M}^{\prime}\) mit Betti-Zahl ungleich null. Da 
        \[H_k(\mathcal{M})/\langle\eqcl{e\mathrel{|}\mathcal{M}}\rangle\cong H_k(\mathcal{M}^{\prime})/\langle\eqcl{m\mathrel{|}\mathcal{M}^{\prime}}\rangle\]
        gilt, die linke Gruppe endlich ist und \(H_k(\mathcal{M}^{\prime})\) mindestens einen direkten \(\mathbb{Z}\)-Sum\-man\-den besitzt, muss \(\langle\eqcl{m\mathrel{|}\mathcal{M}^{\prime}}\rangle\cong\mathbb{Z}\) sein. Dies zeigt, dass die Torsionsuntergruppe von \(H_k(\mathcal{M}^{\prime})\) zu einer Untergruppe von \(H_k(\mathcal{M})/\langle\eqcl{e\mathrel{|}\mathcal{M}}\rangle\) isomorph ist. Da \(\mathcal{M}^{\prime}\) erneut durch Chirurgie in eine Mannigfaltigkeit \"uberf\"uhrt werden kann, deren \(k\)-te Homologiegruppe die Torsionsuntergruppe von \(H_k(\mathcal{M}^{\prime})\) ist, kann \(H_k(\mathcal{M})\) durch zwei Chirurgien durch eine definitiv kleinere endliche Gruppe ersetzt werden. Die Aussage folgt rekursiv.
    \end{proof}

    \begin{corollary}\label{cor:double_odd_zero}
        Es gilt \(P^{4m+1}\cong0\).
    \end{corollary}

\section{Die Berechnung von \texorpdfstring{\(P^{4m+3}\)}{TEXT}}
    Sei \(k\) nun ungerade und \(\mathcal{M}^{2k}\) eine \((k-1)\)-zusammenh\"angende \(\pi\)-Man\-nig\-fal\-tig\-keit, sodass \(H_k(\mathcal{M})\) endlich ist. Sei \(x\in H_k(\mathcal{M})\) und \(\mathcal{S}\) eine repr\"asentierende Sph\"are. Aus Dimensionsgr\"unden \ref{thm:vec_dim_triv} ist das Normalenb\"undel von \(\mathcal{S}\subseteq\mathring{\mathcal{M}}\) trivial, sodass eine gerahmte Chirurgie an \(\mathcal{S}\) durchgef\"uhrt werden kann. Das Problem in dieser Dimension besteht darin, dass eine derartige Chirurgie die Homologie nicht zwingenderma\ss en vereinfacht. Um dem beizukommen ist eine weitere sorgsame Reparametrisierung vonn\"oten. 
    \subsubsection{Effekt einer Reparametrisierung auf \(H_k(\mathcal{M}_0)\)}
        Sei \({\Phi\colon\underline{\mathbb{D}}^{k+1}\hookrightarrow\mathring{\mathcal{M}}}\) eine Anklebeeinbettung, \({\mathcal{D}:=\im\Phi}\) und \({\mathcal{M}_0:=\mathcal{M}\setminus\mathring{\mathcal{D}}}\). Sei \(\mathcal{M}^{\prime}\) die Chirurgie und \(\mathcal{M}_{\gamma}^{\prime}\) die mit einem \(\gamma\colon\mathbb{S}^k\to\operatorname{SO}(k+1)\) reparametrisierte Chirurgie. Beachte, dass \({\partial\mathcal{D}\cong\mathbb{S}^k\times\mathbb{S}^k}\) gilt. Seien \(e,m\in H_k(\mathbb{S}^k\times\mathbb{S}^k)\) die Fundamentalklassen eines \"Aquators \({\mathbb{S}^k\times y}\) und eines Meridians \({x\times\mathbb{S}^k}\). Sei
        \[\overline{\Gamma}\colon\mathbb{S}^k\times\mathbb{S}^k\to\mathbb{S}^k\times\mathbb{S}^k,\,(x,y)\mapsto(x,\gamma(x)\cdot y)\]
        die zugeh\"orige Reparametrisierungsabbildung. Dann folgt \"uber den Satz von Hurewicz, dass in \(H_k(\mathbb{S}^k\times\mathbb{S}^k)\) die Gleichungen
        \begin{equation}\label{eq:torus_reparam}
            e^{\gamma}:=\overline{\Gamma}_*e=e+\psi_*(\eqcl{\gamma})\,m\quad\text{und}\quad\overline{\Gamma}_*m=m\,.
        \end{equation}
        gelten. Es folgt, dass dies auch f\"ur die Inkludierten \(e_0\), \(e_0^{\gamma}\) und \(m_0\) in \(H_k(\mathcal{M}_0)\) gilt. Wie zuvor kann stets eine Reparametrisierung mit einem Element des Kernes der iterierten Einh\"angung \(S_*^{k+1}\) vorgenommen werden. Aus Stabilisierungsgr\"unden ist dies gerade die rechte Abbildung in dem Diagramm
        \begin{center}
            \begin{tikzpicture}
                \draw
                    (-3, 0) node (A) {\(\pi_{k+1}\left(\mathbb{S}^{k+1}\right)\)}
                    (0, 0) node (B) {\(\pi_k\left(\operatorname{SO}(k+1)\right)\)}
                    (3.25, 0) node (C) {\(\pi_k\left(\operatorname{SO}(k+2)\right)\)}
                    (0, 1.5) node (D) {\(\pi_k\left(\operatorname{SO}(k)\right)\)}
                    (0, -1.5) node (E) {\(\pi_k\left(\mathbb{S}^k\right)\)}
                    
                    (A) edge [-stealth] node [above] {\(\partial\)} (B)
                    (B) edge [-stealth] node [above] {\(S_*\)} (C)
                    
                    (D) edge [-stealth] node [right] {\(S_*\)} (B)
                    (B) edge [-stealth] node [right] {\(\psi_*\)} (E)
                    ;
            \end{tikzpicture}
        \end{center}
        Erneut wird \(\ker S_*\) wird von dem Tangentialb\"undel der Sph\"are erzeugt, und ist f\"ur gerade \(k\) zyklisch unendlich. Diese wird unter \(\psi_*\) auf das Doppelte eines Erzeugers abgebildet. Somit kann \(\psi_*\eqcl{\gamma}\) als jedes beliebige Vielfache von \(2\) gew\"ahlt werden.

    \subsubsection{Effekt einer Reparametrisierung auf den Rang von \(H_k(\mathcal{M}^{\prime})\)}
        Seien \(e^{\prime}\in H_k(\mathcal{M})\), \(m^{\prime}\in H_k(\mathcal{M}^{\prime})\) und \(m_{\gamma}^{\prime}\in H_k(\mathcal{M}_{\gamma}^{\prime})\). Sei \(\ell\) der Rang von \(e^{\prime}\). Dann folgt aus der exakten Folge
        \[0\to\mathbb{Z}\xrightarrow{\lambda}H_k(\mathcal{M}_0)\to H_k(\mathcal{M})\to0\,,\]
        dass \(\ell e_0\in\im\lambda\) liegt. Dieses Bild besteht gerade aus den Vielfachen von \(m_0\), also existiert eine Abh\"angigkeit
        \[\ell e_0+\ell^{\prime}m_0=0\,.\]
        Hierbei ist \(\ell^{\prime}\) der Rang von \(m^{\prime}\). Wegen \(e_0^{\gamma}=e_0+\psi_*(\eqcl{\gamma})m_0\) folgt
        \[0=\ell e_0+\ell^{\prime}m_0=\ell\left(e_0^{\gamma}-\psi_*(\eqcl{\gamma})m_0\right)+\ell^{\prime}m_0=\ell e_0^{\gamma}+\left(\ell^{\prime}-\ell\psi_*(\eqcl{\gamma})\right)m_0\,.\]
        Folglich besitzt \(m_{\gamma}^{\prime}\) den Rang \(\abs{\ell^{\prime}-\ell\psi_*(\eqcl{\gamma})}\). Da weiterhin
        \[H_k(\mathcal{M}_{\gamma}^{\prime})/\langle m_{\gamma}^{\prime}\rangle\cong H_k\left(\mathcal{M}\right)/\langle e^{\prime}\rangle\]
        gilt, ist \(H_k(\mathcal{M}_{\gamma}^{\prime})\) kleiner als \(H_k\left(\mathcal{M}\right)\), wenn \(0\leq\operatorname{Rang}m_{\gamma}^{\prime}<\operatorname{Rang}e^{\prime}\) also
        \[0\leq\abs{\ell^{\prime}-\ell\psi_*(\eqcl{\gamma})}<\ell\]
        ist. Da \(\eqcl{\gamma}\in\ker S_*\) so gew\"ahlt werden kann, dass \(\psi_*(\eqcl{\gamma})\) eine beliebige gerade Zahl ist, kann dies, insofern \(\ell^{\prime}\) nicht von \(\ell\) geteilt wird, stets m\"oglich. Um das Teilungsverhalten von \(\ell^{\prime}\) und \(\ell\) zu untersuchen wird die Verschlingungszahl ben\"otigt.

    \subsubsection{Die Verschlingungszahl}
        Sei \(\mathcal{M}^n\) eine orientierte Mannigfaltigkeit und \(n=i+j+1\). Die kurze exakte Folge von Kettenkomplexen
        \[0\to C_*\left(\mathcal{M};\mathbb{Z}\right)\mathop{\rightarrowtail}^pC_*\left(\mathcal{M};\mathbb{Q}\right)\twoheadrightarrow C_*\left(\mathcal{M};\mathbb{Q}/\mathbb{Z}\right)\to0\]
        induziert eine lange exakte Folge
        \[H_{i+1}\left(\mathcal{M};\mathbb{Q}/\mathbb{Z}\right)\xrightarrow{\partial}H_i\left(\mathcal{M};\mathbb{Z}\right)\xrightarrow{p_*}H_i\left(\mathcal{M};\mathbb{Q}\right)\to H_i\left(\mathcal{M};\mathbb{Q}/\mathbb{Z}\right)\]
        Seien \(x\in H_i(\mathcal{M})\) und \(y\in H_j(\mathcal{M})\) endlichen Ranges. Dann gilt \(p_*x=0\), sodass \(x\) zu einem \(\mu\in H_{i+1}(\mathcal{M};\mathbb{Q}/\mathbb{Z})\) mit \(\partial\mu=x\) zur\"uckgezogen werden kann. Definiere die \textbf{Verschlingungszahl} (Linking number) von \(x\) und \(y\) als Schnittzahl repr\"a\-sen\-tie\-ren\-der Ketten von \(x\) mit \(\mu\). Die derart erhaltene Bilinearform ist nicht entartet, und es gilt
        \[L(x,y)+(-1)^{ij}L(y,x)\,.\]
        \begin{lemma}
            Sei \(k>1\) ungerade und \(\mathcal{M}^{2k+1}\) eine Mannigfaltigkeit, sodass \(H_k(\mathcal{M})\) endlich ist. Wird \(x\in H_k(\mathcal{M})\) vom Rang \(\ell\) durch Chirurgie an \(x\) durch ein Element vom Rang \(\ell^{\prime}\) ersetzt, gilt \(\ell^{\prime}/\ell=\pm L(x,x)\in\mathbb{Q}/\mathbb{Z}\).
        \end{lemma}
        \begin{proof}
            Sei \(z_1,z_2\in C_k(\mathcal{M}_0)\) Zyklen, die \(e_0\) und \(m_0\) repr\"asentieren. Wegen \(0=\ell e_0+\ell^{\prime}m_0\), existiert eine Kette \(c\in C_{k+1}(\mathcal{M}_0)\) mit
            \[\partial c=\ell z_1+\ell^{\prime}z_2\,.\]
            Sei \(c_1\in C_{k+1}(\mathcal{M})\) der durch \(\Phi(x\times\mathbb{D}^{k+1})\) definierte Zyklus mit \(\partial c_1=z_2\). Ist \(c_2\) jene durch \(\Phi(\mathbb{S}^k\times0)\) definierte Kette, die \(e\) repr\"asentiert, so gilt
            \[\partial(c-\ell^{\prime}c_1)/\ell=z_1\quad\text{und somit}\quad L(e,e)=(c-\ell^{\prime}c_1)/\ell\cdot c_2\,,\]
            Da \(c_2\) und \(c\) disjunkt sind gilt \(c\cdot c_2=0\), und da \(c_1\) und \(c_2\) genau einen Schnittpunkt in \((x,0)\) besitzen gilt \(c_1\cdot c_2=\mp1\). Es folgt \(L(e,e)=\mp\ell^{\prime}/\ell\).
        \end{proof}
        \begin{corollary}\label{cor:link_nzero_small}
            Sei \(k>1\) ungerade und \(\mathcal{M}^{2k+1}\) eine gerahmte Mannigfaltigkeit, sodass \(H_k(\mathcal{M})\) endlich ist. Existiert ein \(x\in H_k(\mathcal{M})\) mit \(L(x,x)\not=0\), kann \(H_k(\mathcal{M})\) durch eine gerahmte Chirurgie verkleinert werden. 
        \end{corollary}
        Auf diese Art und Weise kann \(H_k(\mathcal{M})\) so lange weiter verkleinert werden, bis keine \(x\) mit \(L(x,x)\not=0\) mehr existieren. Dann besitzt die Gruppe wegen folgendem Lemma jedoch eine besonders einfache Struktur.
        \newpage
        \begin{lemma}\label{lem:link_z2}
            Sei \(k>1\) ungerade und \(\mathcal{M}^{2k+1}\) eine Mannigfaltigkeit, sodass \(H_k(\mathcal{M})\) endlich ist. Gilt \(L(x,x)=0\) f\"ur alle \(x\in H_k(\mathcal{M})\), ist \(H_k(\mathcal{M})\cong\mathbb{Z}_2^{\oplus s}\).
        \end{lemma}
        \begin{proof}
            Aus \({L(\epsilon,\delta)+(-1)^kL(\delta,\epsilon)=0}\) folgt, dass die Verschlingungspaarung f\"ur \({i=j}\) symmetrisch ist. Da \(L(x,x)=0\) f\"ur alle \(x\in H_k(\mathcal{M})\) gilt, ergibt sich
            \[0=L(\epsilon+\delta,\epsilon+\delta)=L(\epsilon,\epsilon)+2L(\epsilon,\delta)+L(\delta,\delta)=2L(\epsilon,\delta)\,.\]
            Da die Verschlingungspaarung nicht entartet ist, folgt aus \(L(2\epsilon,\delta)=0\) f\"ur alle \(\delta\) bereits \(2\epsilon=0\).
        \end{proof}
        \begin{lemma}\label{lem:rep_group}
            Sei \(k>1\) ungerade. Ist \(\mathcal{M}^{2k+1}\) eine geschlossene \((k-1)\)-zu\-sam\-men\-h\"an\-gen\-de gerahmte Mannig\-faltig\-keit mit \(H_k(\mathcal{M})\cong\mathbb{Z}_2^{\oplus s}\), kann eine gerahmte Chirurgie an \(\mathcal{M}\) durchgef\"uhrt werden, sodass 
            \begin{equation}\label{eq:surg_poss}
                H_k(\mathcal{M}^{\prime})\cong\mathbb{Z}\oplus\mathbb{Z}_2^{\oplus(s-2)}\quad\text{oder}\quad H_k(\mathcal{M}^{\prime})\cong\mathbb{Z}_4\oplus\mathbb{Z}_2^{\oplus(s-2)}
            \end{equation}
            ist.
        \end{lemma}
        \begin{proof}
            Sei \(x\in H_k(\mathcal{M})\). Setze \({\mathcal{W}:=\mathcal{M}\times\mathbb{I}\multimap\mathbb{D}^{k+1}}\) und \({\mathcal{M}^{\prime}:=\mathcal{M}\multimap\mathbb{S}^k}\) sodass die Anklebesph\"are \(x\) repr\"asentiere. F\"ur das \(k\)-te Steenrod-Quadrat gilt
            \[\operatorname{Sq}^k\colon H^{k+1}(\mathcal{W},\partial\mathcal{W};\mathbb{F}_2)\to H^{2k+1}(\mathcal{W},\partial\mathcal{W};\mathbb{F}_2)\cong H_1(\mathcal{W};\mathbb{F}_2)=0\]
            sodass aus der Adem-Beziehung \({\operatorname{Sq}^{k+1}=\operatorname{Sq}^1\smile\operatorname{Sq}^k}\) bereits \(\operatorname{Sq}^{k+1}=0\) folgt. Dies zeigt f\"ur alle \(x\in H_{k+1}(\mathcal{W};\mathbb{F}_2)\) die Gleichung
            \[x\cdot x=\left\langle x^*\smile x^*,\eqcl{\mathcal{W}}\right\rangle=\big\langle\operatorname{Sq}^{k+1}(x^*),\eqcl{\mathcal{W}}\big\rangle=0\,,\]
            sodass die Schnittform auf \(H_{k+1}(\mathcal{W};\mathbb{F}_2)\) geraden Rang besitzt. Gem\"a\ss{} Lemma \ref{lem:srg_chg_field_betti} folgt, dass \(\dim_{\mathbb{F}_2}H_k(\mathcal{M}^{\prime};\mathbb{F}_2)\not=\dim_{\mathbb{F}_2}H_k(\mathcal{M};\mathbb{F}_2)\) gilt. Eine Chirurgie kann wie zuvor so reparametrisiert werden, dass \(x\) durch ein Element \(x^{\prime}\in H_k(\mathcal{M}^{\prime})\) geraden Ranges \(\ell^{\prime}\) mit \(0\leq\ell^{\prime}\leq2\) ersetzt wird. Somit ist \(\ell^{\prime}\) entweder null oder zwei. Aus
            \[H_k(\mathcal{M}^{\prime})/\langle x^{\prime}\rangle\cong H_k(\mathcal{M})/\langle x\rangle\cong\mathbb{Z}_2^{\oplus(s-1)}\]
            folgt die Existenz einer kurzen exakten Folge
            \[0\to\mathbb{Z}_{\ell^{\prime}}\rightarrowtail H_k(\mathcal{M}^{\prime})\twoheadrightarrow\mathbb{Z}_2^{\oplus(s-1)}\to0\,.\]
            Beachte \({\mathbb{Z}_0\cong\mathbb{Z}}\). W\"urde diese spalten, g\"olte \({H_k(\mathcal{M}^{\prime};\mathbb{F}_2)\cong\mathbb{Z}_2\oplus\mathbb{Z}_2^{\oplus(s-1)}}\), was ein Widerspruch gegen den Fakt w\"are, dass die Chirurgie die \(\mathbb{F}_2\)-Dimension ver\"andert. Folglich bleiben lediglich die M\"og\-lich\-kei\-ten aus Gleichung \ref{eq:surg_poss}.
        \end{proof}
        \newpage
        \begin{lemma}\label{lem:link_zero_small}
            Sei \(k>1\) ungerade und \(\mathcal{M}^{2k+1}\) eine \((k-1)\)-zu\-sam\-men\-h\"an\-gen\-de gerahmte Man\-nig\-fal\-tig\-keit, deren Rand leer oder eine Homotopiesph\"are ist sodass \(H_k(\mathcal{M})\) endlich ist. Existiert \textbf{kein} \(x\in H_k(\mathcal{M})\) mit \(L(x,x)\not=0\), kann \(H_k(\mathcal{M})\) trotzdem durch eine gerahmte Chirurgie verkleinert werden.
        \end{lemma}
        \begin{proof}
            Da \(L(x,x)=0\) f\"ur alle \(x\in H_k(\mathcal{M})\) gilt, folgt aus Lemma \ref{lem:link_z2} \(H_k(\mathcal{M})\cong\mathbb{Z}_2^{\oplus s}\). Dann ist 
            \[\mathcal{N}:=\mathcal{M}\mathop{+}^{\partial\mathcal{M}}\mathcal{M}\quad\text{mit}\quad{H_k(\mathcal{N})}\cong H_k(\mathcal{M})\oplus H_k(\mathcal{M})\cong\mathbb{Z}_2^{\oplus2s}\]
            eine geschlossene Mannigfaltigkeit. Gem\"a\ss{} Lemma \ref{lem:rep_group} kann an \(\mathcal{N}\) eine Chirurgie durchgef\"uhrt werden, sodass 
            \[\mathbb{Z}_2^{\oplus s}\oplus H_k(\mathcal{M}\multimap\mathbb{S}^k)\cong H_k(\mathcal{N}\multimap\mathbb{S}^k)\cong\mathbb{Z}_2^{\oplus s}\oplus\begin{cases}
                \mathbb{Z}\oplus\mathbb{Z}_2^{\oplus(s-2)}\\
                \mathbb{Z}_4\oplus\mathbb{Z}_2^{\oplus(s-2)}
            \end{cases}\]
            gilt. Da \(\mathbb{Z}_2^{\oplus s}\) gek\"urzt werden kann, ist \(H_k(\mathcal{M}\multimap\mathbb{S}^k)\) entweder zu \(\mathbb{Z}_4\oplus\mathbb{Z}_2^{\oplus(s-2)}\) oder zu \(\mathbb{Z}\oplus\mathbb{Z}_2^{\oplus(s-2)}\) isomorph. Im ersten Fall ergibt dies direkt eine kleinere Gruppe als \(H_k(\mathcal{M})\), im zweiten Fall muss vorher der direkte \(\mathbb{Z}\)-Summand mithilfe von Lemma \ref{lem:odd_dim_finite} eliminiert werden. 
        \end{proof}
        Aus Korollar \ref{cor:link_nzero_small} und Lemma \ref{lem:link_zero_small} folgt zusammen:
        \begin{corollary}\label{cor:single_odd_zero}
            Es gilt \(P^{4m+3}=0\).
        \end{corollary}

    \chapter{Die Berechnung von \texorpdfstring{\(P^{4m}\)}{TEXT}}
        Sei \(k\) gerade und \(\mathcal{M}^{2k}\) eine \(4m\)-dimensionale \(\pi\)-Mannigfaltigkeit. Dann existiert eine \textit{topologische} Kobordismus-Invariante die genutzt werden kann um zu entscheiden ob eine Chirurgie an einer \(k\)-Sph\"are die Homologie vereinfacht. Zudem existiert mit Satz \ref{thm:intersect_even} eine relativ einfache Methode um zu entscheiden, wann das Normalenb\"undel einer eingebetteten Sph\"are trivial ist. In Folge dessen, ist es relativ einfach \(P_{4m}\) zu berechnen. Im Gegensatz dazu steht die Berechnung von \(\partial P_{4m}\), die sich als deutlich aufw\"andiger erweist, und einige Kenntnisse zu \textit{fast parallelisierbaren} Mannigfaltigkeiten ben\"otigt. Die Anzahl der Elemente in \(\partial P^{4k}\) ist gerade
\[2^{2k-3}\left(2^{2k-1}-1\right)\left(3+(-1)^{k+1}\right)\,\text{\upshape Z\"ahler}\left(\frac{B_{2k}}{4k}\right)\,,\]
Dieser zun\"achst kurios erscheinende Term setzt sich aus einigen unterschiedlichen Ausdr\"ucken zusammen, wie zum Beispiel den Koeffizienten der Potenzreihe
\[\frac{\sqrt{t}}{\sinh\left(2\sqrt{t}\right)}=\sum_{k\geq0}\frac{2^{2k}(2^{2k-1}-1)B_{2k}}{(2k)!}t^k\]
und dem Rang des Bildes des \(J\)-Homomorphismus
\[\operatorname{Rang}\operatorname{Im}J_{4k-1}=\,\text{Nenner}\left(\frac{B_{2k}}{4k}\right)\,.\]
\newpage
\section{Die Schnittpaarung in vierfacher Dimension}
    Zun\"achst ist es vonn\"oten zu entscheiden, wann das Normalenb\"undel \(\nu\) einer in eine eingebetteten Sph\"are nicht nur stabil trivial, sondern bereits trivial ist. Da \(\nu\) stabil trivial ist, liegt es in \(\ker S_*\). Betrachte hierzu zun\"achst die Abbildung
\[\Phi\colon\ker S_*\to\mathbb{Z}\,,\eqcl{\xi}\mapsto\left\langle e\left(\xi\right),\eqcl{\mathbb{S}^k}\right\rangle\,.\]
Aus den Eigenschaften der Euler-Klasse folgt, dass dies ein wohldefinierter Homomorphismus ist. Dieser ist jedoch f\"ur ungerade \(k\) trivial, da in diesem Fall \(\ker S_*\cong\mathbb{Z}_2\) endlich ist.
\begin{lemma}\label{lem:phi_mono_even}
    F\"ur gerade \(k\) ist \(\Phi\) ein Monomorphismus, der lediglich gerade Werte annimmt.
\end{lemma}
\begin{proof}
    Der Kern \(\ker S_*\) wird von \(\eqcl{\tau_k}\) erzeugt, und ist, da \(k\) gerade ist, zyklisch unendlich. Folglich ist jedes \(\eqcl{\xi}\in\ker S_*\) ein Vielfaches von \(\eqcl{\tau_k}\), sei etwa \({\eqcl{\xi}=\beta\cdot\eqcl{\tau_k}}\). Da die Euler-Klasse von \(\tau_k\) das Doppelte eines Erzeugers \(g\in H^k(\mathbb{S}^k)\) ist (Siehe \cite{hatcher2017vector} Proposition 3.14), gilt
    \[\Phi(\eqcl{\xi})=\left\langle e(\xi),\eqcl{\mathbb{S}^k}\right\rangle=2\beta\cdot\left\langle g,\eqcl{\mathbb{S}^k}\right\rangle\,.\]
    Da \(\langle g,\eqcl{\mathbb{S}^k}\rangle\not=0\) ist, ist dies genau dann null, wenn \(\beta=0\), und somit \(\eqcl{\xi}=0\) ist.
\end{proof}
Sei \(\mathcal{M}^{2k}\) eine \(\pi\)-Mannigfaltigkeit, und \(\mathcal{S}^k\hookrightarrow\mathring{\mathcal{M}}\) eine eingebettete Sph\"are. Dann folgt aus dem vorangegangenen Lemma, dass das Normalenb\"undel \(\nu\) von \(\mathcal{S}\) genau dann trivial ist, wenn \(\Phi(\nu)=0\) gilt. Um dies zu entscheiden ist das folgende Lemma hilfreich.
\begin{lemma}\label{lem:euler_dual}
    Sei \(\mathcal{M}^{i+j}\) eine orientierte Mannigfaltigkeit und \(\iota\colon\mathcal{V}^i\hookrightarrow\mathring{\mathcal{M}}\) eine geschlossene Untermannigfaltigkeit mit Normalenb\"undel \(\nu\). Dann gilt
    \begin{equation}\label{eq:euler_dual}
        e(\nu)=\pm\iota^*q^*\eqcl{\mathcal{V}\mathrel{|}\mathcal{M}}^*\quad\text{also}\quad\langle e(\nu),\eqcl{\mathcal{V}}\rangle=\pm\eqcl{\mathcal{V}\mathrel{|}\mathcal{M}}\cdot\eqcl{\mathcal{V}\mathrel{|}\mathcal{M}}\,.
    \end{equation}
\end{lemma}
\begin{proof}
    Sei \(\mathcal{D}\subseteq\mathring{\mathcal{M}}\) eine geschlossene Tubenumgebung von \(\mathcal{V}\). Betrachte die kanonischen Inklusionen
    \[d\colon(\mathcal{D},\partial\mathcal{D})\hookrightarrow\big(\mathcal{M},\mathcal{M}\setminus\mathring{\mathcal{D}}\big)\quad\text{und}\quad f\colon(\mathcal{M},\partial\mathcal{M})\hookrightarrow\big(\mathcal{M},\mathcal{M}\setminus\mathring{\mathcal{D}}\big)\,.\]
    Sei \(X:=f_*\eqcl{\mathcal{M}}=d_*\eqcl{\mathcal{D}}\). Dann kommutiert das Diagramm
    \begin{center}
        \begin{tikzpicture}
            \draw
                (0, 0) node (A) {\(H_i\left(\mathcal{V}\right)\)}
                (2, -1.5) node (B) {\(H_i\left(\mathcal{D}\right)\)}
                (2, 0) node (C) {\(H_i\left(\mathcal{M}\right)\)}
                (2, 1.5) node (D) {\(H_i\left(\mathcal{M}\right)\)}
                (5, -1.5) node (E) {\(H^j\left(\mathcal{D},\partial\mathcal{D}\right)\)}
                (5, 0) node (F) {\(H^j\big(\mathcal{M},\mathcal{M}\setminus\mathring{\mathcal{D}}\big)\)}
                (5, 1.5) node (G) {\(H^j\left(\mathcal{M},\partial\mathcal{M}\right)\)}
                (8, -1.5) node (H) {\(H^j\left(\mathcal{D}\right)\)}
                (8, 0) node (I) {\(H^j\left(\mathcal{M}\right)\)}
                (8, 1.5) node (J) {\(H^j\left(\mathcal{M}\right)\)}
                (10, 0) node (K) {\(H^j\left(\mathcal{V}\right)\)}

                (A) edge [-stealth, bend right] node [sloped, below] {\(\cong\)} (B)
                (A) edge [-stealth] (C)
                (A) edge [-stealth, bend left] node [sloped, above] {\(\iota_*\)} (D)
                
                (E) edge [-stealth] node [below] {\(\cong\)} node [above] {\(\frown\eqcl{\mathcal{D}}\)} (B)
                (E) edge [-stealth] node [above] {\(q^*\)} (H)
                (H) edge [-stealth, bend right] node [sloped, below] {\(\cong\)} (K)
                
                (F) edge [-stealth] node [above] {\(\frown X\)} (C)
                (F) edge [-stealth] node [above] {\(q^*\)} (I)
                (I) edge [-stealth] (K)

                (G) edge [-stealth] node [below] {\(\cong\)} node [above] {\(\frown\eqcl{\mathcal{M}}\)} (D)
                (G) edge [-stealth] node [above] {\(q^*\)} (J)
                (J) edge [-stealth, bend left] node [sloped, above] {\(\iota^*\)}  (K)
                
                (B) edge [-stealth] node [left] {\(d_*\)} (C)
                (D) edge [-stealth] node [left] {\(\mathbbm{1}\)} (C)
                
                (F) edge [-stealth] node [left] {\(d^*\)} (E)
                (F) edge [-stealth] node [left] {\(f^*\)} (G)
                
                (I) edge [-stealth] node [left] {\(d^*\)} (H)
                (I) edge [-stealth] node [left] {\(\mathbbm{1}\)} (J)
                ;
        \end{tikzpicture}
    \end{center}
    Hierbei korrespondiert ein Generator von \(H^j\left(\mathcal{D},\partial\mathcal{D}\right)\) mit der Thom-Klasse von \(\nu\), und wird deshalb in \(H^j(\mathcal{V})\) auf die Euler-Klasse von \(\nu\) abgebildet. Durch eine Jagd der Fundamentalklasse \(\eqcl{\mathcal{V}}\in H_i(\mathcal{V})\) durch die obere und untere Zeile ergibt sich
    \[\pm\iota^*q^*\eqcl{\mathcal{V}\mathrel{|}\mathcal{M}}^*=e(\nu)\,.\]
    Es gilt also
    \begin{align*}
        \mp\left\langle e(\nu),\eqcl{\mathcal{S}}\right\rangle&=\left\langle\iota^*q^*\eqcl{\mathcal{S}\mathrel{|}\mathcal{M}}^*,\eqcl{\mathcal{S}}\right\rangle\\
        &=\left\langle q^*\eqcl{\mathcal{S}\mathrel{|}\mathcal{M}}^*,\eqcl{\mathcal{S}\mathrel{|}\mathcal{M}}\right\rangle\mathop{=}^{\text{\tiny\eqref{eq:intersect_prop_2}}}\eqcl{\mathcal{S}\mathrel{|}\mathcal{M}}\cdot\eqcl{\mathcal{S}\mathrel{|}\mathcal{M}}\,.
    \end{align*}
\end{proof}
Der Beweis orientiert sich an einem \"ahnlichen Lemma aus Lecture 11 von \cite{auroux2012algebraic}.

\begin{theorem}\label{thm:intersect_even}
    Sei \(k\geq3\) gerade und \(\mathcal{M}^{2k}\) eine \((k-1)\)-zusammenh\"angende \(\pi\)-Mannig\-faltig\-keit. Dann ist die Schnittpaarung gerade, und es gilt \({x\cdot x=0}\) genau dann, wenn \({\alpha(x)=0}\) ist.
\end{theorem}
\begin{proof}
    Zun\"achst gilt \(\alpha(x)\in\ker S_*\) gem\"a\ss{} Gleichung \ref{eq:alpha_kernel}. Sei \({\iota\colon\mathcal{S}\hookrightarrow\mathring{\mathcal{M}}}\) eine \(x\) repr\"asentierende Einbettung mit Normalenb\"undel \(\nu\). Dann gelten per Definitionem
    \(x=\eqcl{\mathcal{S}\mathrel{|}\mathcal{M}}\) und \(\alpha(x)=\eqcl{\nu}\). Zusammen ergibt dies
    \[x\cdot x=\eqcl{\mathcal{S}\mathrel{|}\mathcal{M}}\cdot\eqcl{\mathcal{S}\mathrel{|}\mathcal{M}}\mathop{=}^{\text{\tiny\eqref{eq:euler_dual}}}\mp\left\langle e(\nu),\eqcl{\mathcal{S}}\right\rangle=\mp\Phi(\alpha(x))\,.\]
    Dann ist die Aussage des Satzes eine direkte Konsequenz von Lemma \ref{lem:phi_mono_even}.
\end{proof}


\section{Die Signatur}
    Da \(k\) gerade und die Schnittform symmetrisch ist, kann diese \"uber \(\mathbb{R}\) diagonalisiert werden. Sei \(n_+\) die Anzahl der positiven und \(n_-\) die Anzahl der negativen Eintr\"age einer Diagonalisierung. Die \textbf{Signatur} von \(\mathcal{M}\) sei definiert als \(\sigma(\mathcal{M}):=n_+-n_-\). F\"ur ungerade \(k\), also wenn die Schnittform schiefsymmetrisch ist, sind alle Eigenwerte komplex, sodass keine Diagonalisierung \"uber \(\mathbb{R}\) existieren kann. 
\begin{theorem}\label{thm:sign_prop}
    F\"ur kompakte, orientierte topologische Mannigfaltigkeiten \(\mathcal{M}^n\), \(\mathcal{N}^n\) gilt
    \begin{itemize}
        \item[i] F\"ur eine Mannigfaltigkeit \(\mathcal{W}^{4m+1}\) gilt \(\sigma(\partial\mathcal{W})=0\)
        \item[ii] \(\sigma(\mathcal{M}\sqcup\mathcal{N})=\sigma(\mathcal{M}+\mathcal{N})=\sigma(\mathcal{M})+\sigma(\mathcal{N})\)
        \item[iii] Es gilt \(\sigma(\mathcal{M}\times\mathcal{N})=\sigma(\mathcal{M})\sigma(\mathcal{N})\)
        \item[iv] Die Signatur einer geraden, unimodularen Form teilt acht
    \end{itemize}
\end{theorem}
\begin{proof}
    \subsubsection{Behauptung \(i\)}
        Aus Gleichung \ref{eq:ker_incl_dim} folgt
        \[2\dim_{\mathbb{R}}\ker\iota_*=\dim_{\mathbb{R}}H_k(\partial\mathcal{W};\mathbb{R})\,.\]
        Weiter verschwindet \(Q(x):=x\cdot x\) auf \(\ker\iota_*\), also folgt
        \[\abs{\sigma(\partial\mathcal{W})}=\abs{\sigma(Q)}\leq\dim_{\mathbb{R}}H_k(\partial\mathcal{W};\mathbb{R})-2\dim_{\mathbb{R}}\ker\iota_*=0\]
        und somit \(\sigma(\partial\mathcal{W})=0\).
    \subsubsection{Behauptung \(ii\)}
        Das folgt, da die mittlere Homologiegruppe der disjunkten Vereinigung, der verbundene Summe und der verbundene Randsumme spaltet, und somit zu \(H_k(\mathcal{M})\oplus H_k(\mathcal{N})\) isomorph ist.
\end{proof}
Der Beweis von \(i\) folgt \cite{dold1980lectures} Proposition 9.6. F\"ur einen Beweis von iii siehe \cite{tomdieck2008algebraic} Proposition 18.7.3. Obwohl Satz \ref{thm:sign_prop} keine glatte Struktur fordert folgt aus ihr nur, dass die Signatur eine Chirurgie-Invariante f\"ur geschlossene Mannigfaltigkeiten ist. Ist \(\mathcal{M}\) eine Mannigfaltigkeit mit nicht-leerem Rand, ist \(\mathcal{M}\times\mathbb{I}\) zwar erneut eine Mannigfaltigkeit mit Rand, jedoch gilt
\[\partial\left(\mathcal{M}\times\mathbb{I}\right)=\partial\mathcal{M}\times\mathbb{I}\cup\mathcal{M}\times\partial\,\mathbb{I}\,.\]

\begin{lemma}\label{lem:4n_symp}
    Eine freie abelsche Gruppe mit einer unimodularen, geraden quadratischen Form \(\beta\) mit Signatur null besitzt eine schwach symplektische Basis.
\end{lemma}
\begin{proof}
    Da die Signatur null ist, ist die Form indefinit, und besitzt gem\"a\ss{} \cite{milnor1961procedure} Lemma 8 eine nicht-triviale Nullstelle \(e_1\). Da die Form unimodular ist, existiert ein \(\alpha\in H_k(\mathcal{M})\) mit \(\beta(\alpha,e_1)=1\). Da sie gerade ist, l\"asst sich
    \[f_1:=\alpha-\frac{\beta(\alpha,\alpha)}{2}e_1\]
    definieren. Es folgen
    \[\beta(e_1,f_1)=\beta(e_1,\alpha)-\frac{\beta(\alpha,\alpha)}{2}\beta(e_1,e_1)=1\]
    und
    \[\beta(f_1,f_1)=\beta(\alpha,\alpha)-\frac{2\beta(\alpha,\alpha)}{2}\beta(\alpha,e_1)+0=0\,.\]
    Aus der Unimodularit\"at folgt weiter, dass \(H\) in die direkte Summe der Untergruppe \(\langle e_1,f_1\rangle\) und des orthogonalen Komplements spaltet. Die Aussage folgt \"uber eine Induktion \"uber den Rang der Gruppe.
\end{proof}
\newpage
\begin{theorem}\label{thm:sign_inv}
    Die Signatur induziert einen Monomorphismus \(\sigma\colon P^{4m}\to\mathbb{Z}\).
\end{theorem}
\begin{proof}
    \subsubsection*{Wohldefiniertheit}
        Sei \({n=i+j+1}\). Es reicht aus \({\sigma(\mathcal{M})=\sigma(\mathcal{N})}\) f\"ur \({\mathcal{N}=\mathcal{M}\multimap\mathbb{S}^i}\) zu zeigen. Es kann zudem angenommen werden, dass \(i\leq j\) gilt, denn l\"asst sich \(\mathcal{N}\) durch gerahmte \(i\)-Chirurgie aus \(\mathcal{M}\) erhalten, l\"asst sich \(\mathcal{M}\) durch gerahmte \(j\)-Chirurgie aus \(\mathcal{N}\) erhalten. Gem\"a\ss{} Lemma \ref{lem:smoothing_pi} existiert eine gerahmte Mannigfaltigkeit \(\mathcal{W}^{n+1}\) mit 
        \[\partial\mathcal{W}\cong\mathcal{M}\mathop{+}^{\partial\mathcal{M}}\mathcal{N}\,.\]
        Aus Proposition \ref{thm:sign_prop} folgt \(\sigma(\partial\mathcal{W})=0\), da \(\partial\mathcal{M}\) eine Homotopiesph\"are ist, ist weiter
        \[H_k(\partial\mathcal{W})\cong H_k(\mathcal{M})\oplus H_k(\mathcal{N})\,,\]
        also \(0=\sigma(\partial\mathcal{W})=\sigma(\mathcal{M})-\sigma(\mathcal{N})\) und somit \(\sigma(\mathcal{M})=\sigma(\mathcal{N})\). 

    \subsubsection*{Injektivit\"at}
        Sei \(k\) gerade und \((\mathcal{M}^{2k},F)\) eine gerahmte Mannigfaltigkeit in \(\ker\sigma\). Da \(\partial\mathcal{M}\) eine Homotopiesph\"are ist, ist \(H_k(\mathcal{M})\) gem\"a\ss{} Lemma \ref{lem:middle_free} frei. Da die Schnittform symmetrisch, gerade \ref{thm:intersect_even} und unimodular \ref{crl:intersect_uni} ist und \(\sigma(\mathcal{M})=0\) gilt, besitzt \(H_k(\mathcal{M})\) gem\"a\ss{} Satz \ref{lem:4n_symp} eine schwach symplektische Basis \(e_i,f_i\) mit
        \[e_i\cdot e_j=f_i\cdot f_j=0\quad\text{und}\quad e_i\cdot f_j=\delta_{ij}\,.\]
        Aus Satz \ref{thm:intersect_even} folgt, dass \({\alpha(e_i)=0}\) gilt, sodass \(\mathcal{M}\) gem\"a\ss{} Satz \ref{thm:even_symp_ann} zu einer \(k\)-zusammenh\"angenden gerahmten Mannigfaltigkeit \(\chi\)-\"aquivalent ist und damit das triviale Element in \(P_{4m}\) repr\"asentiert. Die Homomorphismus-Eigenschaft folgt aus Lemma \ref{thm:sign_prop}.
\end{proof}

\begin{example}[Milnor-Mannigfaltigkeit]\label{ex:milnor_man}
    Es existiert eine \(4n\)-Mannigfaltigkeit \(M(4n)\), deren Schnittform durch die Matrix
    \[\Gamma_8:=\begin{pmatrix}
        2 & 1 & 0 & 0 & 0 & 0 & 0 & 0\\
        1 & 2 & 1 & 0 & 0 & 0 & 0 & 0\\
        0 & 1 & 2 & 1 & 0 & 0 & 0 & 0\\
        0 & 0 & 1 & 2 & 1 & 0 & 0 & 0\\
        0 & 0 & 0 & 1 & 2 & 1 & \mathcolor{red}{1} & 0\\
        0 & 0 & 0 & 0 & 1 & 2 & \mathcolor{red}{0} & 0\\
        0 & 0 & 0 & 0 & \mathcolor{red}{1} & \mathcolor{red}{0} & 2 & 1\\
        0 & 0 & 0 & 0 & 0 & 0 & 1 & 2
    \end{pmatrix}\]
    gegeben ist. Durch eine Rechnung folgt \(\det(\Gamma_8)=1\) und \(\sigma(\Gamma_8)=8\). Es gilt also \(\sigma(M(4n))=8\). Diese l\"asst sich konstruieren, indem an eine \(2k\)-Scheibe acht \(k\)-Henkel angebracht werden. Hierzu werden disjunkte Anklebeabbildungen \(\phi_i\colon\mathbb{S}^{k-1}\hookrightarrow\mathbb{D}^{2k}\), die sich f\"ur \(k\geq3\) zu disjunkten eingebetteten Scheiben fortsetzen lassen. In der Mannigfaltigkeit \(\mathcal{D}\), welche durch das Anbringen von Henkeln entlang der \(\phi\) entsteht, k\"onnen nun diese eingebetteten Scheiben als untere-, und die Kerne der Henkel als obere Hemisph\"aren aufgefasst werden. F\"ur jeden Henkel ergibt sich derart eine \(k\)-Sph\"are. Es l\"asst sich zeigen, dass sich die Anklebeabbildungen derart w\"ahlen lassen, dass zu gegebenen Elementen \(\eqcl{\gamma_i}\in\pi_{k-1}\left(\operatorname{SO}\left(k\right)\right)\) die Kupplungsfunktionen der Normalenb\"undel der Pr\"asentationssph\"aren gerade die \(\gamma_i\) sind. Wird nun acht mal \(\tau_k\) gew\"ahlt, ergibt sich die Milnor-Mannigfaltigkeit. Die Schnittform wird durch die Matrix \(\Gamma_8\) beschrieben, dass zugeh\"orige Gitter ist durch das au\ss erordentliche Dynkin-Diagramm \(E_8\) klassifiziert.
    \begin{figure}[!ht]
        \centering
        \begin{tikzpicture}
            \draw 
                (1, 0) node [state, minimum size = 5pt] (A) {}
                (2, 0) node [state, minimum size = 5pt] (B) {}
                (3, 0) node [state, minimum size = 5pt] (C) {}
                (4, 0) node [state, minimum size = 5pt] (D) {}
                (5, 0) node [state, minimum size = 5pt] (E) {}
                (5, 1) node [state, minimum size = 5pt] (F) {}
                (6, 0) node [state, minimum size = 5pt] (G) {}
                (7, 0) node [state, minimum size = 5pt] (H) {}

                (A) -- (B) -- (C) -- (D) -- (E) -- (G) -- (H)
                (F) -- (E)
            
            ;
        \end{tikzpicture}
    \end{figure}
\end{example}

\begin{theorem}\label{thm:sign_image}
    Es gilt \(P^{4m}\cong\operatorname{Im}\sigma=8\mathbb{Z}\).
\end{theorem}
\begin{proof}
    Sei \((\mathcal{M},F)\in P^{4m}\). Dann ist \(\sigma(\mathcal{M})\) gem\"a\ss{} \ref{thm:sign_prop} durch acht teilbar. Da \(\sigma\) gem\"a\ss{} \ref{thm:sign_inv} ein Monomorphismus ist, ist \(P^{4m}\) zu einer Untergruppe von \(8\mathbb{Z}\) isomorph. Die Aussage folgt, da die Milnor-Mannigfaltigkeit \(M(4m)\) in \(P^{4m}\) liegt und Signatur \(8\) besitzt \ref{ex:milnor_man}.
\end{proof}

\section{Der Signatursatz von Hirzebruch}
    Ein wichtiger Satz, welcher f\"ur die Berechnung von \(P_{4k}\) ben\"otigt wird, ist der Signatursatz von Hirzebruch. Er besagt, dass ein bestimmtes Polynom \(L_k\) existiert, sodass sich die Signatur durch die Pontrjagin-Klassen des Tangential\-b\"un\-dels \(p_i\) wie folgt darstellen l\"asst:
\[\sigma(\mathcal{M})=\langle L_k(p_1,\dots,p_k),\eqcl{\mathcal{M}}\rangle\,.\]
Um diese \(L\)-Polynome zu definieren, ist etwas Kombinatorik vonn\"oten. Das Folgende sei nur ein kurzer Anriss der eigentlichen Thematik, f\"ur eine pr\"azisere Abhandlung des Satzes siehe zum Beispiel \cite{milnor1974characteristic} \S19. 

\subsection{Multiplikative Folgen}
    Eine Folge \(K_i\in\Lambda\eqcl{X_1,\dots,X_i}\) homogener Polynome \(i\)-ten Grades hei\ss e multiplikativ, wenn \(K_0=1\) ist und f\"ur alle formalen Potenzreihen \(a,b,c\in\Lambda[[t]]\) mit Leitkoeffizient \(1\) und \(c=ab\) die Gleichung
    \[\sum_{i\geq0}K_i(c_1,\dots,c_i)t^i=\bigg(\sum_{i\geq0}K_i(a_1,\dots,a_i)t^i\bigg)\bigg(\sum_{i\geq0}K_i(b_1,\dots,b_i)t^i\bigg)\,,\]
    gilt. Schreibe auch \(K(ab)=K(a)K(b)\). Zu jeder multiplikativen Folge geh\"ort nun die \textbf{charakteristische Potenzreihe} \(K(1+t)\in\Lambda[[t]]\). Dass die Korrespondenz von multiplikativen Folge zu ihren charakteristischen Potenzreihen bijektiv ist, ist durch folgenden Satz gegeben.
    \begin{proposition}
        Zu jeder Potenzreihe \(f\in\Lambda[[t]]\) existiert genau eine multiplikative Folge \(K_i\) mit \(K(1+t)=f(t)\).
    \end{proposition}
    Die zu der Potenzreihe
    \[\frac{\sqrt{x}}{\tanh(\sqrt{x})}=\sum_{k\geq0}\frac{2^{2k}B_{2k}}{(2k)!}x^k\]
    geh\"orende Folge sei im Folgenden mit \(L_k\) bezeichnet, f\"ur eine \(4k\)-di\-men\-sio\-na\-le geschlossene und orientierte Manngifaltigkeit sei
    \[\mathcal{L}(\mathcal{M}):=\langle L_k(p_1,\dots,p_k),\eqcl{\mathcal{M}}\rangle\]
    der \textbf{L-Genus}, wobei \(p_j\) die \(j\)-te Pontrjagin-Klasse von \(T\mathcal{M}\) bezeichne. Die Wahl der charakteristischen Potenzreihe wirkt zun\"achst etwas willk\"urlich, kann jedoch als Normalisierung verstanden werden. Durch sie kann
    \[\mathcal{L}\left(\mathbb{C}P^{2m}\right)=1\]
    garantiert werden.

    \subsubsection{Der orientierte Kobordismusring}
        Zusammen mit der disjunkten Vereinigung bildet die Menge der orientierten Kobordismusklassen orientierter geschlossener Mannigfaltigkeiten \(\Omega_k^{\text{\tiny Or}}\) eine abelsche Gruppe. Diese wird von den Produkten \(\mathbb{C}P^{2n_1}\times\dots\times\mathbb{C}P^{2n_{\ell}}\) f\"ur alle Zerlegungen \(\sum n_i=k\) erzeugt. Dann erh\"alt
        \[\Omega_*^{\text{\tiny Or}}:=\bigoplus_{k\geq0}\Omega_{4k}^{\text{\tiny Or}}\,.\]
        mithilfe des kartesischen Produkts die Struktur eines graduierten Ringes. Insbesondere bilden die \(\mathbb{C}P^{2k}\) eine Basis des torsionsbefreiten Kobordismusrings \(\Omega_*^{\text{\tiny Or}}\otimes\mathbb{Q}\).
    \begin{proposition}
        Die Abbildung \(\mathcal{L}\colon\Omega_*^{\text{\tiny Or}}\otimes\mathbb{Q}\to\mathbb{Q},\,\mathcal{M}\mapsto\mathcal{L}\left(\mathcal{M}\right)\) ist ein Ringhomomorphismus.
    \end{proposition}
    Der Signatursatz von Hirzebruch ist nun eine direkte Konsequenz aus dem vorangegangenen Satz. Sowohl \(\sigma\) als auch \(\mathcal{L}\) definieren Ringhomomorphismen \(\Omega_*^{\text{\tiny Or}}\otimes\mathbb{Q}\to\mathbb{Q}\), und auf der Menge der Generatoren \(\mathbb{C}P^{2k}\) gilt
    \[\sigma\left(\mathbb{C}P^{2k}\right)=1=\mathcal{L}\left(\mathbb{C}P^{2k}\right)\,.\]
    Dies zeigt:
    \begin{proposition}[Signatursatz von Hirzebruch]\label{prop:signature_thm}
        Die Signatur \(\sigma(\mathcal{M})\) einer geschlossenen Mannigfaltigkeit \(\mathcal{M}\) ist gleich dem L-Genus \(\mathcal{L}(\mathcal{M})\).
    \end{proposition}
    F\"ur das Folgende ist noch ein weiterer Satz vonn\"oten.
    \begin{proposition}\label{prop:signature_coeff}
        Der Koeffizient von \(p_k\) in \(L_k(p_1,\dots,p_k)\) ist
        \[s_k=(-1)^k\frac{2^{2k}(2^{2k-1}-1)B_{2k}}{(2k)!}\,.\]
    \end{proposition}
    \begin{proof}
        Siehe \cite{hirzebruch1966topological} Lemma 1.4.1 und Abschnitt 1.5.
    \end{proof}
    Die in diesem Satz auftauchenden Koeffizienten berechnen sich hierbei aus den Koeffizienten der Potenzreihe
    \[\frac{x}{\sinh(2x)}=\sum_{k\geq0}\frac{2^{2k}(2^{2k-1}-1)B_{2k}}{(2k)!}x^{2k}\,.\]

\section{Fast parallelisierbare Mannigfaltigkeiten}
    Das Tangentialb\"undel einer \(n\)-Sph\"are ist f\"ur \(n\notin\{1,3,7\}\) nie trivial. Wird jedoch aus der Sph\"are eine kleine Scheibe entfernt, ergibt sich eine kontrahierbare Mannigfaltigkeit. \"Uber einem kontrahierbaren Raum ist hingegen jedes Vektorb\"undel trivial. Dies motiviert folgende Definition: 
\begin{definition}[Fast gerahmtes Vektorb\"undel]
    Ein Vektorb\"undel \(\xi\colon E\to\mathcal{M}\) \"uber einer geschlossenen Mannigfaltigkeit mit einer Einbettung \(\mathcal{D}\hookrightarrow\mathcal{M}\) und einer Rahmung der Einschr\"ankung von \(\xi\) auf \(\mathcal{M}\setminus\mathring{\mathcal{D}}\).
\end{definition}
Eine Mannigfaltigkeit mit einem fast gerahmten (stabilen) Tangentialb\"undel hei\ss e fast (stabil) gerahmt. Eine Mannigfaltigkeit, dessen Tangentialb\"undel fast (stabil) gerahmt werden kann hei\ss e fast (stabil) parallelisierbar. 
\begin{theorem}
    F\"ur \(n>1\) sind zusammenh\"angende und geschlossene \(\pi\)-Mannigfaltigkeiten fast parallelisierbar.
\end{theorem}
\begin{proof}
    Sei \(\mathcal{M}^n\) eine zusammenh\"angende, geschlossene \(\pi\)-Mannigfaltigkeit und \({\mathcal{D}\hookrightarrow\mathcal{M}}\) eine eingebettete Scheibe mit \(p\in\mathcal{D}\). Zun\"achst sind \({\mathcal{M}_0:=\mathcal{M}\setminus\mathring{\mathcal{D}}}\) und \(\mathcal{M}\setminus p\) homotopie\"aquivalent. Offenbar ist \(T\mathcal{M}|_{\mathcal{M}_0}\) stabil trivial. Weiter ist \(\mathcal{M}\setminus p\) eine nicht-kompakte zusammenh\"angende Mannigfaltigkeit, und enth\"alt somit einen \((n-1)\)-dimensionalen CW-Komplex \(X\) als Deformationsretrakt (siehe etwa \cite{napier2004elementary} Satz 2.2). Ist \(f\colon X\to\mathcal{M}_0\) eine Homotopie\"aquivalenz, folgt die Aussage, da \(f^*(T\mathcal{M}|_{\mathcal{M}_0})\) aufgrund von Satz \ref{thm:vec_dim_triv} trivial ist.
\end{proof}
Es stellt sich die Frage nach der Umkehrung dieses Satzes. Analog ergibt sich, dass eine Mannigfaltigkeit genau dann fast parallelisierbar ist, wenn sie fast stabil parallelisierbar ist. Folgendes folgt nahe \cite{kosinski1992differential} Kapitel IX Abschnitt 8.

\subsubsection{Die Gruppe \texorpdfstring{\(\Omega_n^{\text{\tiny Fast}}\)}{TEXT}}
    Die Bedeutung der fast stabil parallelisierbaren Mannigfaltigkeit liegt in der folgenden \"Uberlegung. Sei \((\mathcal{M},\mathcal{D},F)\) eine fast stabil gerahmte Mannigfaltigkeit. Dann kann an \(\mathcal{M}\) eine Chirurgie durchgef\"uhrt werden. Ist das Anklebegebiet der Chirurgie von \(\mathcal{D}\) disjunkt, und die zugeh\"orige Chirurgie an \(\mathcal{M}_0\) gerahmt, hei\ss e diese Chirurgie fast gerahmt. Zwei fast stabil gerahmte Mannigfaltigkeiten hei\ss en fast \(\chi\)-\"aquivalent, wenn eine durch eine endliche Folge von fast gerahmten Chirurgien aus der anderen erhalten werden kann. Die Menge der fast stabil gerahmten Mannigfaltigkeiten modulo fast-\(\chi\)-\"Aquivalenz sei durch \(\Omega_n^{\text{\tiny Fast}}\) bezeichnet. Dann ist die Abbildung
    \[c\colon\Omega_n^{\text{\tiny Fast}}\to P^n,\,\eqcl{(\mathcal{M},\mathcal{D},F)}\mapsto\eqcl{(\mathcal{M}\setminus\mathring{\mathcal{D}},F)}\]
    per Konstruktion wohldefiniert und injektiv. Folglich erbt \(\Omega_n^{\text{\tiny Fast}}\) eine kommutative Monoidstruktur von \(P^n\), sodass
    \begin{equation}
        0\longrightarrow\Omega_n^{\text{\tiny Fast}}\mathop{\rightarrowtail}^cP_n\mathop{\twoheadrightarrow}^{\partial}\partial P_n\longrightarrow0
    \end{equation}
    eine kurze exakte Folge ist. Es gilt also gerade \(\Omega_n^{\text{\tiny Fast}}\cong\ker\partial\). F\"ur \(n>5\) ist dies eine abelsche Gruppe.

\subsection{Von fast- zu stabil parallelisierbar}
    Sei \((\xi,\mathcal{D},F)\) ein fast gerahmtes Vektorb\"undel \"uber \(\mathcal{M}\). Wird \(F\) als Trivialisierung der Einschr\"ankung von \(\xi\) auf \(\mathcal{M}_0:=\mathcal{M}\setminus\mathring{\mathcal{D}}\) aufgefasst, l\"asst sich das kollabierte B\"undel \(\mu:=\xi/F\) \"uber dem Quotienten
    \[\mathcal{M}/\mathcal{M}_0\cong\mathcal{D}/\partial\mathcal{D}\cong\mathbb{S}^n\]
    bilden. F\"ur die Quotientenabbildung \(q\colon\mathcal{M}\to\mathcal{M}/\mathcal{M}_0\cong\mathbb{S}^n\) gilt \(\xi\cong q^*\mu\). Diese Abbildung besitzt offenbar Grad eins. Durch die Standardrahmung der Scheibe ist weiter eine Trivialisierung \(E\) der Einschr\"ankung von \(\xi\) auf \(\mathcal{D}\) gegeben. Es l\"asst sich erkennen, dass die Kupplungsfunktion von \(\mu\) gerade der Rahmenwechsel von \(F|_{\partial\mathcal{D}}\) zu \(E|_{\partial\mathcal{D}}\) ist. Besitzt \(\xi\) den Rang \(k\geq n\), so gilt
    \[S_*\eqcl{\mu}\in\pi_{n-1}(\operatorname{SO}(k+1))\cong\pi_{n-1}(\operatorname{SO})\,.\]
    Gem\"a\ss{} dem Periodizit\"atssatz von Bott gilt:
    \begin{center}
        \begin{tabular}{c|cccccccc}
            \(n\operatorname{mod}8\) & 0 & 1 & 2 & 3 & 4 & 5 & 6 & 7\\\hline
            \(\pi_{n-1}(\operatorname{SO})\)& \(\mathbb{Z}\) & \(\mathbb{Z}_2\) & \(\mathbb{Z}_2\) & \(0\) & \(\mathbb{Z}\) & \(0\) & \(0\) & \(0\)
        \end{tabular}
    \end{center}
    F\"ur \({n\operatorname{mod}8\in\{3,5,6,7\}}\) folgt direkt \({S_*\eqcl{\mu}=0}\). Wegen \({S\xi=f^*(S\mu)}\) ist \(\xi\) also bereits stabil trivial.
    \newpage
    \begin{theorem}
        Ist \(n\not=4k\), ist jede fast parallelisierbare Mannigfaltigkeit stabil parallelisierbar.
    \end{theorem}
    \begin{proof}
        Es muss gezeigt werden, dass das Normalenb\"undel \(\nu\) von \(\mathcal{M}^n\) in einem hinreichend gro\ss en \(\mathbb{R}^{n+k+1}\) trivial ist. Sei \(\nu=q^*\mu\). Dann ist \(J\mu\) die normal gerahmte Mannigfaltigkeit \((\mathbb{S}^n,G)\) im \(\mathbb{R}^{n+k}\), und \(\mathcal{M}_0\subseteq\mathbb{R}_+^{n+k+1}\) ein gerahmter Nullbordismus, sodass \(\mu\in\ker J_n^k\) folgt. Da der stabile \(J\)-Homomorphismus \(J_n\) f\"ur \(n\in\{1,2\}\) injektiv ist, folgt somit \(\mu=0\) also \(\nu=0\).
    \end{proof}

\subsection{4k-dimensionale Vektorbündel}
    Sei erneut \(\xi=q^*\mu\). Dann gilt f\"ur die Pontrjagin-Klassen \({p_i(\xi)\in H^{4i}(\mathcal{M})}\) und \({p_i(\mu)\in H^{4i}(\mathbb{S}^{4m})}\) 
    \[p_i(\xi)=p_i(q^*\mu)=q^*p_i(\mu)\,.\]
    Da f\"ur \(i<k\) stets \(H^{4i}(\mathbb{S}^{4m})=0\) gilt, m\"ussen also alle niederen Pontrjagin-Klassen von \(\xi\) null sein. Dann ist durch
    \[P\colon\pi_{4m-1}(\operatorname{SO})\to\mathbb{Z},\,\eta\mapsto\big\langle p_k(\eta),\eqcl{\mathbb{S}^{4m}}\big\rangle\]
    ein Homomorphismus definiert. Dass dies tats\"achlich ein Homomorphismus ist, folgt aus der Na\-t\"ur\-lich\-keit der Pontrjagin-Klassen. Gem\"a\ss{} \cite{kervaire1959obstructions} ist \(P\) ein Monomorphismus, und es gilt
    \begin{equation}\label{eq:pont_hom_multiple}
        p(x)\quad\text{ist ein Vielfaches von}\quad\frac{3+(-1)^{m+1}}{2}(2m-1)!\,.
    \end{equation}
    \begin{lemma}\label{lem:4m_stable_pont}
        Ein \(4m\)-dimensionales, fast parallelisierbares Vektorb\"undel \(\xi\) \"uber \(\mathcal{M}^{4m}\) ist genau dann stabil trivial, wenn \(p_m(\xi)=0\) ist.
    \end{lemma}
    \begin{proof}
        Ist \(\xi\) stabil trivial, ist \(p_m(\xi)=0\). Sei umgekehrt \(p_m(\xi)=0\) und \(\xi\cong q^*\mu\). Da \(q\) den Grad eins besitzt, muss auch \(p_m(\mu)=0\) sein. Dann ist
        \[P(\mu\oplus\underline{\mathbb{R}})=\langle p_m(\mu\oplus\underline{\mathbb{R}}),\eqcl{\mathbb{S}^n}\rangle=\langle p_m(\mu),\eqcl{\mathbb{S}^n}\rangle=0\,,\]
        da \(P\) ein Monomorphismus ist, ist \(\mu\) und somit auch \(\xi\) stabil trivial.
    \end{proof}
    \begin{theorem}
        Eine fast parallelisierbare Mannigfaltigkeit \(\mathcal{M}^{4m}\) ist genau dann stabil parallelisierbar, wenn \(\sigma(\mathcal{M})=0\) gilt.
    \end{theorem}
    \begin{proof}
        Sei \(p_i:=p_i(T\mathcal{M})\). Es sei daran erinnert, dass \(p_i=0\) f\"ur \(i<m\) ist. Der Signatursatz von Hirzebruch impliziert deshalb, dass die Signatur ein nicht-triviales Vielfaches von \(\langle p_m,\eqcl{\mathcal{M}}\rangle\) ist. Somit ist \(p_m=0\) genau dann, wenn \(\sigma(\mathcal{M})=0\) ist. Die Aussage folgt aus Lemma \ref{lem:4m_stable_pont}.
    \end{proof}

    \begin{corollary}\label{cor:hom_pi}
        Homotopiesph\"aren sind \(\pi\)-Mannigfaltigkeiten.
    \end{corollary}
    
    \newpage
    \begin{theorem}\label{thm:almost_sign_image}
        Das Bild von \(\sigma\colon\Omega_{4m}^{\text{\tiny Fast}}\to\mathbb{Z}\) besitzt ist gerade \(\abs{t_m}\cdot\mathbb{Z}\) mit
        \[t_m:=2^{2m}\left(2^{2m-1}-1\right)\left(3+(-1)^{m+1}\right)\,\text{\upshape Z\"ahler}\left(\frac{B_{2m}}{4m}\right)\,.\]
    \end{theorem}
    \begin{proof}
        Sei \((\mathcal{M},\mathcal{D},F)\in\Omega_{4m}^{\text{\tiny Fast}}\) eine fast gerahmte Mannigfaltigkeit. Sei \(\nu\) das Normalenb\"undel von \(\mathcal{M}\) in einem hinreichend gro\ss en \(\mathbb{R}^m\), und \(\nu=q^*\mu\) f\"ur ein Vektorb\"undel \(\mu\) \"uber \(\mathbb{S}^{4m}\). Erneut liegt die Kupplungsfunktion von \(\mu\) im Kern des stabilen \(J\)-Homomorphismus \(J_{4m-1}\). Sei \(g\in\pi_{4m}(\operatorname{SO})\cong\mathbb{Z}\) ein Generator. Da das Bild von \(J_{4m-1}\) eine zyklische Gruppe des Ranges \(j_{4m-1}\) ist, sind Elemente dieses Kernes Vielfache von
        \[j_{4m-1}\cdot g\mathop{=}^{\text{\tiny\ref{prop:adams_order}}}\text{Nenner}\left(\frac{B_{2m}}{4m}\right)\cdot g\,.\]
        Somit folgt zusammen mit \eqref{eq:pont_hom_multiple} bereits, dass
        \[P(\mu)\quad\text{ein Vielfaches von}\quad\text{Nenner}\left(\frac{B_{2m}}{4m}\right)\frac{3+(-1)^{m+1}}{2}(2m-1)!\]
        ist. Da \(q\) vom Grad eins ist gilt weiter
        \[\left\langle p_m(\nu),\eqcl{\mathcal{M}}\right\rangle=\left\langle q^*p_m(\mu),\eqcl{\mathcal{M}}\right\rangle=\left\langle p_m(\mu),q_*\eqcl{\mathcal{M}}\right\rangle=\left\langle p_m(\mu),\eqcl{\mathbb{S}^{4m}}\right\rangle=P(\mu)\,.\]
        Sei \(\tau\) das Tangentialb\"undel von \(\mathcal{M}\), so gilt
        \[0=p_m(\tau)=p_m(\nu)+p_m(\tau)\quad\text{also}\quad p_m(\tau)=\pm p_m(\nu)=\pm p_m(\mu)\,.\]
        Aus dem Signatursatz \ref{prop:signature_thm} zusammen mit Proposition \ref{prop:signature_coeff} folgt
        \[\sigma(\mathcal{M})=(-1)^m\frac{2^{2m}(2^{2m-1}-1)B_{2m}}{(2m)!}\langle p_m(\tau),\eqcl{\mathcal{M}}\rangle\]
        und ist deshalb ein Vielfaches von
        \begin{align*}
            &\frac{2^{2m}(2^{2m-1}-1)B_{2m}}{(2m)!}\text{Nenner}\left(\frac{B_{2m}}{4m}\right)\frac{3+(-1)^{m+1}}{2}(2m-1)!\\
            =\,&\frac{2^{2m}(2^{2m-1}-1)B_{2m}}{4m}\text{Nenner}\left(\frac{B_{2m}}{4m}\right)\left(3+(-1)^{m+1}\right)
        \end{align*}
        ist. Der Nenner l\"asst sich weiter vereinfachen, denn aus
        \[x=\frac{\text{\upshape Z\"ahler}(x)}{\operatorname{Nenner}(x)}\quad\text{folgt}\quad\frac{B_{2m}}{4m}\text{Nenner}\left(\frac{B_{2m}}{4m}\right)=\text{\upshape Z\"ahler}\left(\frac{B_{2m}}{4m}\right)\,.\]
        Zusammen ist also \(\sigma(\mathcal{M})\) ein Vielfaches von \(t_m\). Umgekehrt l\"asst sich eine Mannigfaltigkeit mit Signatur \(t_m\) konstruieren.
    \end{proof}

    \chapter{Die Berechnung von \texorpdfstring{\(P^{4m+2}\)}{TEXT}}
        \section{Die Kervaire-Invariante}
    \"Ahnlich der Signatur l\"asst sich eine Invariante gerahmter Kobordismen definieren, die \textbf{Kervaire-In\-va\-ri\-an\-te}, welche daf\"ur genutzt werden kann zu entscheiden, wann eine \(2k\)-Mannigfaltigkeit mit \(k\) ungerade zu einer \(k\)-zusammen\-h\"angen\-den Mannigfaltigkeit gerahmt kobordant ist. Die Definition dieser ist jedoch etwas komplizierter, als die der Signatur.

\subsection{Quadratische Verfeinerungen}
    Sei \(V\) ein \(\mathbb{K}\)-Vektorraum. Wenn \(\mathbb{K}\) von der Charakteristik \(\not=2\) ist, korrespondiert jede Bilinearform \(\beta\colon V\otimes V\to\mathbb{K}\) mit einer quadratischen Form \(q\colon V\to\mathbb{K}\) \"uber die Korrespondenz
    \[\beta(x,y):=\frac{q(x+y)-q(x)-q(y)}{2}\quad\text{und}\quad q(x):=\beta(x,x)\,.\]
    Dies \"andert sich, wenn \(\mathbb{K}\) die Charakteristik zwei besitzt. In diesem Fall muss eine quadratische Verfeinerung betrachtet werden.
    \begin{definition}[Quadratische Verfeinerung]
        Sei \(V\) ein \(\mathbb{F}_2\)-Vektorraum und \(\beta\colon V\otimes V\to\mathbb{F}_2\) eine Bilinearform. Eine Funktion \(q\colon V\to\mathbb{F}_2\) hei\ss e quadratische Verfeinerung, wenn f\"ur alle \(x,y\in V\) die Gleichung
        \[\beta(x,y)=q(x+y)+q(x)+q(y)\]
        gelte.
    \end{definition}
    Die Existenz einer solchen Verfeinerung ist nicht trivial.
    \newpage
    \begin{definition}[Arf-Invariante einer quadratischen Form]
        Sei \(V\) ein \(\mathbb{F}_2\)-Vektorraum mit einer symplektischen Basis \(e_i,f_i\in V\) bez\"uglich einer Bilinearform \(\beta\colon V\otimes V\to\mathbb{F}_2\) mit quadratischer Verfeinerung \(q\colon V\to\mathbb{F}_2\) besitzt. Setze
        \[\operatorname{Arf}(\beta,q):=\sum_{i=1}^kq(e_i)q(f_i)\,.\]
    \end{definition}
    Die Arf-Invariante h\"angt hierbei nicht von der gew\"ahlten symplektischen Basis ab. 

\subsection{Eine quadratische Verfeinerung der Schnittform}
    Sei \(\mathcal{M}\) eine \((k-1)\)-zusammenh\"angende \(\pi\)-Mannigfaltigkeit. \"Uber die Schnittform wird auf den \(\mathbb{F}_2\)-Vektorr\"aumen \(H_k(\mathcal{M};\mathbb{F}_2)\)
    \[H_k(\mathcal{M};\mathbb{F}_2)\otimes H_k(\mathcal{M};\mathbb{F}_2)\to\mathbb{F}_2,\,x\otimes y\mapsto x\cdot y\]
    eine unimodulare, schiefsymmetrische Bilinearform definiert. Es stellt sich die Frage nach einer geeigneten quadratischen Verfeinerung. Der wohl einfachste Kandidat hierzu wurde bereits zuvor definiert, und ist durch jene Abbildung \(\alpha\) gegeben, die einer eingebetteten Sph\"are, die \(x\in H_k(\mathcal{M})\) repr\"asentiert, die Kupplungsfunktion ihres Normalenb\"undels zuordnet. F\"ur \(k\in\{3,7\}\) ist bereits \({\alpha(x)\in\ker S_*=0}\), die Funktion \(\alpha\) ist also trivial.
    \begin{theorem}
        Sei \(\mathcal{M}^{2k}\) eine \((k-1)\)-zusammenh\"angende \(\pi\)-Mannigfaltigkeit und \(k\) ungerade. Dann gilt f\"ur \(x,y\in H_k(\mathcal{M})\) die Gleichung
        \begin{equation}
            \alpha(x+y)=\alpha(x)+\alpha(y)+\eqcl{\tau_k}\cdot(x\cdot y)\in\ker S_*\,.
        \end{equation}
    \end{theorem}
    \begin{proof}
        Seien \(f\colon\mathbb{S}^k\hookrightarrow\mathcal{M}\) und \(g\colon\mathbb{S}^k\hookrightarrow\mathcal{M}\) zwei Einbettungen, die \(x\) und \(y\) repr\"asentieren. Analog zu der Konstruktion der Addition der Homotopiegruppen ist eine stetige Funktion \(f+g\colon\mathbb{S}^k\to\mathcal{M}\) definiert. Siehe auch Abbildung \ref{fig:imm_sum}. Aus der Nat\"urlichkeit des Hurewicz-Homomorphismus folgt, dass \(\eqcl{f+g}=\eqcl{f}+\eqcl{g}\in\pi_k(\mathcal{M})\) die Klasse \(x+y\) repr\"asentiert. Es kann angenommen werden, dass \(f+g\) eine sich selbst transversal schneidende Immersion, nicht zwingenderma\ss en jedoch eine Einbettung, ist. Die Selbstschnittzahl \(\beta\) von \(f+g\) ist gerade die Summe von \(\pm1\) \"uber alle Doppelpunkte von \(f+g\). Da weder \(f\) noch \(g\) Doppelpunkte besitzen, muss
        \[\beta(f+g)=\eqcl{\mathcal{S}_1,\mathcal{S}_2}=x\cdot y\]
        gelten. Weiter ergibt sich aus der Konstruktion der Kupplungsfunktionen, dass \(\nu(f+g)=\nu(f)+\nu(g)\) gilt.

        \subsubsection{Der Fall \(x\cdot y=0\)}
            Es folgt \(\beta(f+g)=0\), sodass \(f+g\) gem\"a\ss{} Korollar \ref{cor:imm_reg_hom} regul\"ar homotop zu einer Einbettung \(\iota\) ist, die \(x+y\) repr\"asentiert. Dann gilt aber
            \[\alpha(x+y)=\eqcl{\nu(\iota)}=\eqcl{\nu(f+g)}=\eqcl{\nu(f)}+\eqcl{\nu(g)}=\alpha(x)+\alpha(y)\,.\]

        \subsubsection{Der Fall \(x\cdot y=1\)}
            Sei \(h\colon\mathbb{S}^k\looparrowright\mathcal{M}\) eine nullhomotope Immersion mit Selbstschnittzahl eins. Eine solche existiert im \(\mathbb{R}^{2k}\) gem\"a\ss{} \ref{prop:imm_inter_zero}, also \"uber eine Karte \(\mathbb{R}^{2k}\to\mathcal{M}\) auch in \(\mathcal{M}\). Es folgt, dass \(f+g+h\) Selbstschnittzahl null besitzt, regul\"ar homotop zu einer Einbettung \(\iota\) ist, und die, da \(h\) nullhomotop ist, \(x+y\) repr\"asentiert. Es folgt
            \begin{equation}\label{eq:alpha_odd}
                \alpha(x+y)=\eqcl{\nu(\iota)}=\eqcl{\nu(f+g+h)}=\alpha(x)+\alpha(y)+\eqcl{\nu(h)}\,.
            \end{equation}
        
        \subsubsection{Das Normalenb\"undel von \(h\)}
            Da \(h\) \"uber eine Karte definiert ist, ist \(\nu(h)\in\ker S_*\) nicht von \(\mathcal{M}\) abh\"angig, somit reicht es aus, \(\nu(h)=\tau_k\) f\"ur \(\mathcal{M}=\mathbb{S}^k\times\mathbb{S}^k\) zu zeigen. Es gilt \({H_k(\mathbb{S}^k\times\mathbb{S}^k)\cong\mathbb{Z}\oplus\mathbb{Z}}\). Seien \(x\) und \(y\) von den Einbettungen \({f=\mathbb{S}^k\times p}\) und \({g=p\times\mathbb{S}^k}\) repr\"asentiert. Diese besitzen triviale Normalenb\"undel und es gilt \(x\cdot y=1\). Bekannterweise ist \(f+g\) zu der Diagonalen in \(\mathbb{S}^k\times\mathbb{S}^k\) homotop. Diese besitzt gem\"a\ss{} \cite{milnor1974characteristic} Satz 11.5 Normalenb\"undel \(\tau_k\). Aus \(\alpha(x)=\alpha(y)=0\) folgt
            \[\eqcl{\nu(h)}\mathop{=}^{\text{\tiny\eqref{eq:alpha_odd}}}\alpha(x+y)=\eqcl{\nu(f+g)}=\eqcl{\tau_k}\,.\]
            Somit gilt allgemein
            \[\alpha(x+y)=\alpha(x)+\alpha(y)+\eqcl{\tau_k}\cdot(x\cdot y)\,.\]
    \end{proof}
    
    \begin{figure}
        \centering
        \begin{tikzpicture}[scale = 0.5]
            \draw (1, 0) arc (0:360:1);
            \draw[xshift = 5cm] (1, 0.5) arc (0:180:1) -- (-1, -0.5) arc (180:360:1) -- cycle;
            \draw[xshift = 10cm] (0, 0.5) arc (-90:270:1) node {\tiny\textbullet} -- (0, -0.5) arc (-270:90:1) node {\tiny\textbullet} -- cycle;

            \begin{scope}[scale = 0.8, xshift = 18cm]
                \draw
                    (1, -3) 
                    .. controls (0.5,-4) and (2, -4.5) .. (3, -4) node [pos = 0.7, above = 0.2] {\(\mathcal{S}_2\)}
                    .. controls (4, -3.5) and (3, -1.5) .. (1.5, -1.5) node (A) {\tiny\textbullet}
                    .. controls (0, -1.5) and (1.5, -2) .. (1, -3)
                    (1, 1) 
                    .. controls (2, 0) and (3, 2) .. (3, 4) node [pos = 0.18] (B) {\tiny\textbullet}
                    .. controls (3, 6) and (2, 6) .. (1, 5) node [pos = 0.75, below = 0.65] {\(\mathcal{S}_1\)}
                    .. controls (0, 4) and (0, 2) .. (1, 1);
                \draw (A.center) -- (B.center) node [pos = 0.5, right] {\(\gamma\)};
            \end{scope}
            
            \draw
                (1.5, 0) edge [-stealth] (3.5, 0)
                (6.5, 0) edge [-stealth] node [pos = 0.5, above] {\(q\)} (8.5, 0) 
                (11.5, 0) edge [-stealth] node [pos = 0.5, above] {\(f,\gamma,g\)} (14, 0)
                ;
            \draw (A.center) -- (B.center) node [pos = 0.5, right] {\(\gamma\)};
        \end{tikzpicture}
        \caption{Die Konstruktion von \(f+g\) \"uber eine Kurve \(\gamma\). Die resultierende immersierte Mannigfaltigkeit kann als verbundenen Summe \(\mathcal{S}_1+\mathcal{S}_2\) verstanden werden.}\label{fig:imm_sum}
    \end{figure}
    Dies erm\"oglicht nun f\"ur \(k\notin\{3,7\}\) endlich die Definition einer quadratischen Verfeinerung. Die Funktion
    \[\Psi\colon H_k(\mathcal{M};\mathbb{F}_2)\cong H_k(\mathcal{M})\otimes\mathbb{F}_2\xrightarrow{\alpha\otimes\mathbbm{1}}\mathbb{F}_2\]
    erf\"ullt offenbar erneut \(\Psi(x+y)=\Psi(x)+\Psi(y)+x\cdot y\in\mathbb{F}_2\). Die Arf-Invariante bez\"uglich dieser Verfeinerung hei\ss e \textbf{Kervaire-Invariante}, und h\"angt nicht von der Wahl einer Rahmung ab. Es l\"asst sich erkennen, dass f\"ur eine symplektische Basis \(e_i,f_i\) von \(H_k(\mathcal{M})\) nun
    \[\kappa(\mathcal{M})=\sum_{i=1}^{\ell}\alpha(e_i)\alpha(f_i)\mod2\]
    gilt.

\subsection{Eigenschaften der Kervaire-Invariante}
    Sei erneut \(k\notin\{3,7\}\) ungerade. Das vorherige Vorgehen definiert die Kervaire-Invariante f\"ur alle \((k-1)\)-zusammenh\"angenden \(\pi\)-Mannigfaltigkeiten. F\"ur derartige Mannigfaltigkeiten h\"angt die Invariante von keiner gew\"ahlten Rahmung ab. 
    \begin{lemma}
        Zwei \(\chi\)-\"aquivalente, \((k-1)\)-zusammenh\"angende, gerahmte Mannig\-faltig\-kei\-ten \(\mathcal{M}^{2k}\) und \(\mathcal{N}^{2k}\), deren R\"an\-der leer oder Homotopiesph\"aren sind, besitzen gleiche Ker\-vaire-In\-va\-ri\-an\-te.
    \end{lemma}
    \begin{proof}
        Sei \(n=i+j+1\). Es reicht aus den Satz f\"ur \(\mathcal{N}=\mathcal{M}\multimap\mathbb{S}^i\) mit \(i\leq j\) zu zeigen. Dann existiert gem\"a\ss{} Lemma \ref{lem:smoothing_pi} eine gerahmte Mannigfaltigkeit \(\mathcal{W}^{2k+1}\) mit 
        \[\partial\mathcal{W}\cong\mathcal{M}\mathop{+}^{\partial\mathcal{M}}\mathcal{N}\,.\]
        Diese kann durch endlich viele Chirurgien durch eine \((k-1)\)-zusammenh\"angende Mannigfaltigkeit mit gleichem Rand ersetzt werden. Sei \(\iota\colon\partial\mathcal{W}\hookrightarrow\mathcal{W}\) die kanonische Einbettung. W\"ahle gem\"a\ss{} Satz \ref{thm:symp_base_ann} eine symplektische Basis von \(H_k(\partial\mathcal{W})\) mit \(e_i\in\ker\iota_*\). Sei \(f\colon\mathbb{S}^k\hookrightarrow\partial\mathcal{W}\) eine \(e_i\) repr\"asentierende Einbettung. Aus dem Satz von Hurewicz folgt, dass eine stetige Fortsetzung \(\tilde{f}\colon\mathbb{D}^{k+1}\looparrowright\mathcal{W}\) existiert. Diese kann als Immersion mit wohldefiniertem Normalenb\"undel angenommen werden kann. Da \(\mathbb{D}^{k+1}\) kontrahierbar ist, m\"ussen \(\nu(\tilde{f})=0\) und \(\nu(f)=0\) gelten. Es folgt \(\alpha(e_i)=\eqcl{\nu(f)}=0\), also auch \(\kappa(\partial\mathcal{W})=0\). Da \(\partial\mathcal{M}\) leer oder eine Homotopiesph\"are ist, gilt
        \[H_k(\partial\mathcal{W})\cong H_k(\mathcal{M}\mathop{+}^{\partial\mathcal{M}}\mathcal{N})\cong H_k(\mathcal{M})\oplus H_k(\mathcal{N})\,,\]
        symplektische Basen von \(H_k(\mathcal{M})\) und \(H_k(\mathcal{N})\) bilden zusammen also eine Basis von \(H_k(\partial\mathcal{W})\). Dies zeigt \(\kappa(\mathcal{M})+\kappa(\mathcal{N})=0\), also \(\kappa(\mathcal{M})=\kappa(\mathcal{N})\).
    \end{proof}
    Sei \((\mathcal{M},F)\) eine gerahmte Mannigfaltigkeit, deren Rand leer ist oder eine Homotopiesph\"are berande. Diese ist zu einer \((k-1)\)-zu\-sam\-men\-h\"ang\-enden gerahmten Mannigfaltigkeiten \((\mathcal{M}^{\prime},F^{\prime})\) \(\chi\)-\"aquivalent, f\"ur welche die Kervaire-Invariante bereits definiert ist. Setze also \({\kappa(\mathcal{M},F):=\kappa(\mathcal{M}^{\prime})}\). Das vorangegangene Lemma garantiert, dass dies wohldefiniert ist.
    \newpage
    \begin{lemma}
        Die Kervaire-Invariante ist bez\"uglich der verbundenen Summe additiv.
    \end{lemma}
    \begin{proof}
        F\"ur \((k-1)\)-zusammenh\"angende \(\pi\)-Mannigfaltigkeiten folgt das aus
        \[H_k(\mathcal{M}+\mathcal{N})\cong H_k(\mathcal{M})\oplus H_k(\mathcal{N})\,,\]
        da die Vereinigung symplektischer Basen von \(H_k(\mathcal{M})\) und \(H_k(\mathcal{N})\) eine symplektische Basis von \(H_k(\mathcal{M}+\mathcal{N})\) ergibt.
    \end{proof}

    \begin{lemma}\label{lem:kerv_homo}
        Eine gerahmte Mannigfaltigkeit \((\mathcal{M},F)\), deren Rand leer oder eine Homotopiesph\"are ist, sodass \(\kappa(\mathcal{M},F)=0\) gilt, ist zu einer \(k\)-zusammenh\"angenden Mannigfaltigkeit \(\chi\)-\"aquivalent.
    \end{lemma}
    \begin{proof}
        Es kann angenommen werden, dass \(\mathcal{M}\) bereits \((k-1)\)-zu\-sam\-men\-h\"an\-gend ist. Sei \(e_i,f_i\) f\"ur \(1\leq i\leq\ell\) eine symplektische Basis von \(H_k(\mathcal{M})\). Dann gilt
        \[\kappa(\mathcal{M})=\sum_{i=1}^{\ell}\alpha(e_i)\alpha(f_i)\mod2\,.\]
        Wenn \(\alpha(e_i)\alpha(f_i)=0\) ist, setze
        \[e_i^{\prime}:=\begin{cases}
            e_i & \alpha(e_i)=0\\
            f_i & \alpha(e_i)=1
        \end{cases}\qquad\text{und}\qquad f_i^{\prime}:=\begin{cases}
            f_i & \alpha(e_i)=0\\
            e_i & \alpha(e_i)=1
        \end{cases}\,.\]
        Wegen \(\kappa(\mathcal{M})=0\) gilt \(\alpha(e_i)\alpha(f_i)=1\) f\"ur eine gerade Anzahl von \(i\). F\"ur ein Paar \(\alpha(e_i)\alpha(f_i)=\alpha(e_j)\alpha(f_j)=1\) kann durch die Substitutionen
        \[e_i^{\prime}:=e_i+e_j\qquad\text{und}\qquad e_j^{\prime}:=f_j-f_i\]
        und
        \[f_i^{\prime}:=f_i\qquad\text{und}\qquad f_j^{\prime}:=e_i\]
        \(\alpha(e_i^{\prime})=\alpha(e_j^{\prime})=0\) erreicht werden. Insgesamt bilden die \(e_i^{\prime},f_i^{\prime}\) eine symplektische Basis, in der \(\alpha(e_i^{\prime})=0\) f\"ur alle \(i\) gilt. Lemma \ref{thm:even_symp_ann} besagt nun, dass \(\mathcal{M}\) zu einer \(k\)-zusammenh\"angenden Mannigfaltigkeit \(\chi\)-\"aquivalent ist.
    \end{proof}

    \begin{theorem}\label{thm:kerv_iso}
        Die Kervaire-Invariante \(\kappa\colon P^{2k}\to\mathbb{Z}_2\) ist ein Isomorphismus.
    \end{theorem}
    \begin{proof}
        Sei \({(\mathcal{M},F)\in P^{2k}}\) mit \(\kappa(\mathcal{M},F)=0\). Gem\"a\ss{} Lemma \ref{lem:kerv_homo} ist \(\mathcal{M}\) zu einer \(k\)-zusammenh\"angenden, also einer kontrahierbaren, Mannigfaltigkeit \(\chi\)-\"aquivalent, und repr\"asentiert somit das Nullelement in \(P^{2k}\). Die Aussage folgt, da sich durch Klempern eine \(2k\)-Mannigfaltigkeit in \(P^{2k}\) mit Kervaire-Invariante eins konstruieren l\"asst.
    \end{proof}

    \chapter{Die Berechnung von \texorpdfstring{\(\Theta_n\)}{TEXT}}
        \section{Die Berechnung von \texorpdfstring{\(\partial P^n\)}{TEXT}}
    Betrachte erneut die exakte Folge
\[0\longrightarrow\Omega_n^{\text{\tiny Fast}}\mathop{\rightarrowtail}^cP^n\mathop{\twoheadrightarrow}^{\partial}\partial P^n\longrightarrow 0\,.\]
Den Korollaren \ref{cor:double_odd_zero} und \ref{cor:single_odd_zero} zufolge gilt \(P^{4m+1}=P^{4m+3}=0\), sodass sich direkt \(\partial P^{4m+1}=\partial P^{4m+3}=0\) ergibt.

\subsubsection{Die Berechnung von \(\partial P_{4m}\)}
    F\"ur \({4m\geq8}\) gelten \(P^{4m}\cong8\mathbb{Z}\) gem\"a\ss{} Satz \ref{thm:sign_image} und \(\Omega_{4m}^{\text{\tiny Fast}}\cong\abs{t_m}\cdot\mathbb{Z}\) gem\"a\ss{} Satz \ref{thm:almost_sign_image}. Somit existiert eine exakte Folge
    \[0\to\abs{t_m}\cdot\mathbb{Z}\rightarrowtail8\mathbb{Z}\twoheadrightarrow\partial P^{4m}\to0\,,\]
    und es ergibt sich
    \[\partial P^{4m}\cong8\mathbb{Z}/\abs{t_m}\cdot\mathbb{Z}\,.\]
    Folglich besitzt \(\partial P^{4m}\) f\"ur \(4m\geq8\) die Kardinalit\"at
    \[\frac{\abs{t_m}}{8}=2^{2m-3}\left(2^{2m-1}-1\right)\left(3+(-1)^{m+1}\right)\,\text{\upshape Z\"ahler}\left(\frac{\abs{B_{2m}}}{4m}\right)\,.\]
    Beispielsweise ist wegen \(B_4=-\nicefrac{1}{30}\) 
    \begin{equation}\label{eq:partialp8}
        \frac{\abs{t_8}}{8}=2^1\left(2^3-1\right)\left(3+(-1)^3\right)\,\text{\upshape Z\"ahler}\left(\frac{1}{4\cdot30}\right)=28\,,
    \end{equation}
    also \(\partial P^8\cong\mathbb{Z}_{28}\).

\subsubsection{Die Berechnung von \(\partial P_{4k+2}\)}
    Gem\"a\ss{} Satz \ref{thm:kerv_iso} gilt \({P^{4k+2}\cong\mathbb{Z}_2}\) f\"ur \(4m+2\notin\{6,14\}\), sodass in diesen F\"allen \({\partial P^{4k+2}\cong\mathbb{Z}_2}\) oder \({\partial P^{4k+2}=0}\) gelten muss. Eine Erweiterung der Definition der Kervaire-Invariante (siehe Levine) oder ein direkter Beweis (siehe Kosinski) zeigt \(\partial P^{4k+2}=0\) f\"ur \(4k+2\in\{2,6,14\}\). Es folgt \(\Omega_{4k+2}^{\text{\tiny Fast}}\subseteq\mathbb{Z}_2\). Die Frage nach der Kardinalit\"at von \(\Omega_n^{\text{\tiny Fast}}\) ist betr\"achtlich schwieriger. Browder zeigte 1969 \cite{browder1969general}, dass geschlossene Mannigfaltigkeiten mit Kervaire-Invariante eins lediglich in Dimensionen der Form \(n=2^k-2\), und zwar genau dann existieren, wenn ein gewisses Element der Adams-Spektralfolge ein \(\theta_{k-1}\in\Pi_{2^k-2}\) repr\"asentiert. Weiter konnte 2009 von Hill et al. \cite{hill2009nonexistence} gezeigt werden, dass hierbei auch alle \(k\geq8\) ausgeschlossen werden k\"onnen, sodass lediglich die F\"alle \(n\in\{30,62,126\}\) verblieben. Nun existiert in Dimensionen \(n=30\) eine explizite Konstruktion einer derartigen Mannigfaltigkeit von Jones \cite{jones1978extended}, und in Dimension \(n=62\) ein Existenzbeweis von \(\theta_5\) von Barratt et al. \cite{barratt1984relations}. In einem noch nicht verifizierten Pre-Print von Lin et al. aus dem Jahr 2025 \cite{lin2025last} wird die Existenz von \(\theta_6\) gezeigt. Werden diese S\"atze angenommen, folgt
    \[\partial P^{4k+2}\cong\begin{cases}
        0 & 4k+2\in\{2,6,14,30,62,126\}\\
        \mathbb{Z}_2 & \text{sonst}
    \end{cases}\,.\]

\section{Finale Ergebnisse}
    Sei \(p\colon\Omega_n^{\text{\tiny Fr}}\to\Pi_n\) der stabile Thom-Pontrjagin-Isomorphismus, und \(p(\mathcal{M})\subseteq\Pi_n\) die Menge jener stabilen Kollaps\-ab\-bil\-dung\-en, die mit verschiedenen normalen Rahmungen von \(\mathcal{M}\) in hinreichend gro\ss en \(\mathbb{R}^{n+k}\) korrespondieren. Sei \(\Sigma_n\subseteq\Pi_n\) die Menge der Abbildungen, die mit gerahmten Homotopiesph\"aren korrespondieren.
\begin{theorem}
    Die Menge \(p(\mathbb{S}^n)\subseteq\Sigma_n\) ist ein Normalteiler und
    \[\Phi\colon\Theta_n\to\Sigma_n/p(\mathbb{S}^n),\,\eqcl{\Sigma}\mapsto p(\Sigma)\]
    ein wohldefinierter Epimorphismus.
\end{theorem}
\begin{proof}
    Es gelten
    \begin{itemize}
        \item[i] \(\mathcal{M}+\mathbb{S}^n\cong\mathcal{M}\) f\"ur geschlossene Mannigfaltigkeiten \(\mathcal{M}\) und
        \item[ii]\(\Sigma+(-\Sigma)\cong\mathbb{S}^n\) f\"ur Homotopiesph\"aren \(\Sigma\).
    \end{itemize}
    Dass \(p(\mathbb{S}^n)\) ein Normalteiler ist, folgt aus \(\mathbb{S}^n+\mathbb{S}^n\cong\mathbb{S}^n\). F\"ur gerahmte Homotopiesph\"aren \((\Sigma,F)\) gilt einerseits
    \[-p(\Sigma,F)+p(\Sigma,F^{\prime})=p(\mathbb{S}^n,H)\quad\text{also}\quad p(\Sigma,F^{\prime})=p(\Sigma,F)+p(\mathbb{S}^n,H)\,,\]
    also \(p(\Sigma)\subseteq p(\Sigma,F)+p(\mathbb{S}^n)\). Andererseits gilt
    \[p(\Sigma,F)+p(\mathbb{S}^n,G)\in p(\Sigma)\]
    und somit \(p(\Sigma,F)+p(\mathbb{S}^n)\subseteq p(\Sigma)\). Folglich ist \(p(\Sigma)\) eine Nebenklasse von \(p(\mathbb{S}^n)\). Sind zwei Homotopiesph\"aren H-Kobordant, ergibt ein H-Kobordismus eine Homotopie \(p(\Sigma_1,F_1)\simeq p(\Sigma_2,F_2)\). Die Surjektivit\"at folgt, da jede Homotopiesph\"are gem\"a\ss{} Korollar \ref{cor:hom_pi} eine \(\pi\)-Mannigfaltigkeit ist, sodass ihr stabiles Normalenb\"undel eine Rahmung zul\"asst und deshalb im Bild von \(\Phi\) liegt.
\end{proof}
Beachte, dass \(p(\mathbb{S}^n)\) gerade das Bild des \(J\)-Homomorphismus \(J_n\) ist. Somit existiert eine Folge
\begin{equation}\label{eq:calc_theta_1}
    0\to\partial P^{n+1}\rightarrowtail\Theta_n\xrightarrow{\Phi}\Sigma_n/\im J_n\to0\,.
\end{equation}
Wenn \(\Phi(\Sigma)=0\) gilt, existiert eine Rahmung von \(\Sigma\), bez\"uglich welcher \((\Sigma,F)\) gerahmt nullbordant ist. Somit muss \(\Sigma\in\partial P^{n+1}\) liegen, also ist Folge \ref{eq:calc_theta_1} exakt. Da \(\Sigma_n/\im J_n\) der Quotient der Untergruppe \(\Sigma_n\subseteq\Pi_n\) und \(\Pi_n\) endlich ist, folgt:
\begin{corollary}
    Die Kervaire-Milnor-Gruppe \(\Theta_k\) ist f\"ur \(k\geq4\) endlich.
\end{corollary}
Zusammen mit der kurzen exakten Folge
\[0\to\Sigma_k/\im J_k\to\Pi_k/\im J_k\to\Pi_k/\Sigma_k\to0\,,\]
in welcher der mittlere Term gerade \(\operatorname{Coker} J_k\) ist, ergibt sich f\"ur eine Berechnung von \(\Theta_k\) die exakte Folge
\begin{equation}\label{eq:calc_theta_2}
    0\to\partial P^{k+1}\to\Theta_k\to\operatorname{Coker}J_k\to\Pi_k/\Sigma_k\to0\,.
\end{equation}
Es gilt (\cite{kosinski1992differential} Kapitel IX Satz 6.7)
\[\Pi_k/\Sigma_k\cong\begin{cases}
    \mathbb{Z}_2 & k\in\{2,6,14,30,62,126\}\\
    0 & \text{sonst}
\end{cases}\,.\]
Da das Bild von \(J_k\) bekannt ist, reduziert sich die Berechnung der Gruppe der Homotopiesph\"aren \(\Theta_k\) somit auf die stabilen Homotopiegruppe der Sph\"are \(\Pi_k\), die gr\"o\ss tenteils unbekannt sind. Beispielsweise ergibt dies f\"ur \(n=7\)
\[0\to\partial P^8\to\Theta_7\to\operatorname{Coker}J_7\to0\,.\]
Wegen \(\Pi_7\cong\im J_7\) folgt 
\[\Theta_7\cong\partial P^8\mathop{\cong}^{\text{\tiny\eqref{eq:partialp8}}}\mathbb{Z}_{28}\,,\]
also existieren genau \(28\) exotische Sph\"aren der Dimension \(7\).


    \appendix
    
    \chapter*{Appendix}
    \refstepcounter{chapter}
    \addcontentsline{toc}{chapter}{Appendix}
    
    \section{Relative Homologie einer Chirurgie}
    \begin{theorem}
        F\"ur \(i\leq j-1\) gelten
        \[H_k(\mathcal{M}^{\prime},\mathcal{M}_0)\cong\begin{cases}
            \mathbb{Z} & k\in\{0,i+1\}\\
            0 & 1<k<i+1
        \end{cases}\]
        und 
        \[H_k(\mathcal{M},\mathcal{M}_0)\cong\begin{cases}
            \mathbb{Z} & k\in\{0,j+1\}\\
            0 & 1<k<j+1
        \end{cases}\,.\]
        Erzeuger der \(\mathbb{Z}\)-Anteile in Dimension \(i+1\) und \(j+1\) sind der Kern \((\mathbb{D}^{i+1},\mathbb{S}^i)\) und der Kokern \((\mathbb{D}^{j+1},\mathbb{S}^j)\).
    \end{theorem}
    \begin{proof}
        Die lange exakte Folge des Tripels \((\mathbb{D}^{i+1}\times\mathbb{D}^{j+1},\mathbb{D}^{i+1}\times\mathbb{S}^j,\mathbb{S}^i\times\mathbb{S}^j)\) ist von der Form
        \begin{center}
            \begin{tikzpicture}
                \draw
                (-3, 1) node (A) {\(H_{k+1}(\mathbb{D}^{i+1}\times\mathbb{D}^{j+1},\mathbb{S}^i\times\mathbb{S}^j)\)}
                (-3, 0) node (B) {\(H_{k+1}(\mathbb{D}^{i+1}\times\mathbb{D}^{j+1},\mathbb{D}^{i+1}\times\mathbb{S}^j)\)}
                (-3, -1) node (C) {\(H_k(\mathbb{D}^{i+1}\times\mathbb{S}^j,\mathbb{S}^i\times\mathbb{S}^j)\)}
                (-3, -2) node (D) {\(H_k(\mathbb{D}^{i+1}\times\mathbb{D}^{j+1},\mathbb{S}^i\times\mathbb{S}^j)\)}
                (-3, -3) node (E) {\(H_k(\mathbb{D}^{i+1}\times\mathbb{D}^{j+1},\mathbb{D}^{i+1}\times\mathbb{S}^j)\)}
                
                (2, 1) node (F) {\(H_k(\mathbb{S}^i\times\mathbb{S}^j)\)}
                (2, 0) node (G) {\(H_k(\mathbb{S}^j)\)}
                (2, -1) node (H) {\(A_k\)}
                (2, -2) node (I) {\(H_{k-1}(\mathbb{S}^i\times\mathbb{S}^j)\)}
                (2, -3) node (J) {\(H_{k-1}(\mathbb{S}^j)\)}
    
                (A) edge [-stealth] (B)
                (B) edge [-stealth] (C)
                (C) edge [-stealth] (D)
                (D) edge [-stealth] (E)
                
                (F) edge [-stealth] (G)
                (G) edge [-stealth] (H)
                (H) edge [-stealth] (I)
                (I) edge [-stealth] (J)
                
                (A) edge [-stealth] node [above] {\(\cong\)} (F)
                (B) edge [-stealth] node [above] {\(\cong\)} (G)
                (C) edge [-stealth] node [above] {\(:=\)} (H)
                (D) edge [-stealth] node [above] {\(\cong\)} (I)
                (E) edge [-stealth] node [above] {\(\cong\)} (J)
                ;
            \end{tikzpicture}
        \end{center}
        Ebenso ist die lange exakte Folge f\"ur \((\mathbb{D}^{i+1}\times\mathbb{D}^{j+1},\mathbb{S}^i\times\mathbb{D}^{j+1},\mathbb{S}^i\times\mathbb{S}^j)\) von der Form
        \[H_k(\mathbb{S}^i\times\mathbb{S}^j)\to H_k(\mathbb{S}^i)\to B_k\to H_{k-1}(\mathbb{S}^i\times\mathbb{S}^j)\to H_{k-1}(\mathbb{S}^i)\,.\]
        Aus \(k\notin\{j,j+1\}\) folgt direkt 
        \begin{equation}
            A_k:=H_k(\mathbb{D}^{i+1}\times\mathbb{S}^j,\mathbb{S}^i\times\mathbb{S}^j)\cong H_{k-1}(\mathbb{S}^i\times\mathbb{S}^j)\,,
        \end{equation}
        aus \(k\notin\{i,i+1\}\)
        \begin{equation}
            B_k:=H_k(\mathbb{S}^i\times\mathbb{D}^{j+1},\mathbb{S}^i\times\mathbb{S}^j)\cong H_{k-1}(\mathbb{S}^i\times\mathbb{S}^j)\,.
        \end{equation}
        Dies impliziert \(A_k=0\) f\"ur \(1<k<j\) und \(B_k=0\) f\"ur \(1<k<i\). Interessant ist nun das Verhalten von \(\{i,i+1\}\cap\{j,j+1\}\). Es gibt drei F\"alle: \(i=j-1\), \(i=j\) und \(i=j+1\). Hierbei \"ubernehmen \(i\) und \(j\) f\"ur \(A_i\) und \(B_i\) duale Stellungen, sodass in den langen exakten Folgen von oben essenziell drei F\"alle eintreten k\"onnen.
        
        \subsubsection{Fall \(1\)}
            Es ergeben sich die exakten Folgen
            \[0\to A_{i+2}\rightarrowtail\mathbb{Z}\mathop{\longrightarrow}^{\mathbbm{1}}\mathbb{Z}\to A_{i+1}\twoheadrightarrow\mathbb{Z}\to0\quad\text{f\"ur}\quad i=j+1\,,\]
            \[0\to B_{j+2}\rightarrowtail\mathbb{Z}\mathop{\longrightarrow}^{\mathbbm{1}}\mathbb{Z}\to B_{j+1}\twoheadrightarrow\mathbb{Z}\to0\quad\text{f\"ur}\quad j=i+1\,,\]
            also folgt \(A_{i+2}\cong\ker\mathbbm{1}=0\). Da die Identit\"at surjektiv ist, ist der Morphismus \(\mathbb{Z}\to A_{i+1}\) gleich null, also folgt \(A_{i+1}\cong\mathbb{Z}\). Komplett analog folgen \(B_{j+2}=0\) und \(B_{j+1}\cong\mathbb{Z}\).
            
        \subsubsection{Fall \(2\)}
            Sei \(i=j\). Dann ergeben sich die exakten Folgen
            \[0\to A_{i+1}\rightarrowtail\mathbb{Z}\oplus\mathbb{Z}\mathop{\longrightarrow}^{\pi_2}\mathbb{Z}\twoheadrightarrow A_i\to0\]
            \[0\to B_{j+1}\rightarrowtail\mathbb{Z}\oplus\mathbb{Z}\mathop{\longrightarrow}^{\pi_1}\mathbb{Z}\twoheadrightarrow B_j\to0\]
            Der mittlere Homomorphismus ist surjektiv, also gelten
            \[A_{i+1}\cong\ker\pi_2=\mathbb{Z}\quad\text{und}\quad A_i\cong\operatorname{coker}\pi_2=0\,,\]
            und analog \(B_{j+1}\cong\mathbb{Z}\) und \(B_j=0\).
            
        \subsubsection{Fall \(3\)}
            Es ergeben sich die exakten Folgen
            \[0\to A_{i+1}\mathop{\to}^{\sim}\mathbb{Z}\to0\to A_i\rightarrowtail\mathbb{Z}\mathop{\longrightarrow}^{\mathbbm{1}}\mathbb{Z}\twoheadrightarrow A_{i-1}\to0\quad\text{f\"ur}\quad i=j-1\,.\]
            \[0\to B_{j+1}\mathop{\to}^{\sim}\mathbb{Z}\to0\to B_j\rightarrowtail\mathbb{Z}\mathop{\longrightarrow}^{\mathbbm{1}}\mathbb{Z}\twoheadrightarrow B_{j-1}\to0\quad\text{f\"ur}\quad j=i-1\,.\]
            Direkt folgen \(A_{i+1}\cong\mathbb{Z}\) und \(B_{j+1}\cong\mathbb{Z}\), sowie
            \[A_i\cong B_j\cong\ker\mathbbm{1}=0\quad\text{und}\quad A_{i-1}\cong B_{j-1}\cong\operatorname{coker}\mathbbm{1}=0\,.\]
    \end{proof}
    \printbibliography
\end{document}
